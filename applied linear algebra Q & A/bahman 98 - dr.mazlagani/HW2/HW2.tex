\documentclass{article}
\renewcommand{\labelenumii}{\alph{enumii}}
\usepackage{enumerate}
\usepackage{listings}

\usepackage{graphicx,comment,framed}
\usepackage{roundbox}
\usepackage{fancybox}
\usepackage{tikz}
\usepackage{color}
\usepackage[hidelinks]{hyperref}
\usepackage{framed}
\usepackage{amsthm,amssymb,amsmath}
%\usepackage[colorlinks,linkcolor=blue,citecolor=blue]{hyperref}
\definecolor{shadecolor}{cmyk}{0,0,0,0}
\usepackage{listings}
\usepackage{xepersian}
\usepackage[noend]{algpseudocode}


%--------------------- page settings ----------------------

\settextfont[Scale=1.1]{XB Niloofar}
\setdigitfont[Scale=1.1]{XB Niloofar}
\defpersianfont\sayeh[Scale=1.1]{XB Niloofar}
\addtolength{\textheight}{3.2cm}
\addtolength{\topmargin}{-22mm}
\addtolength{\textwidth}{3cm}
\addtolength{\oddsidemargin}{-1.5cm}


%------------------------ Environments ------------------------------------

\newtheorem{قضیه}{قضیه}
\newtheorem{لم}{لم}
\newtheorem{مشاهده}{مشاهده}
\newtheorem{تعریف}{تعریف}


%-------------------------- Notations ------------------------------------

\newcommand{\IR}{\ensuremath{\mathbb{R}}} 
\newcommand{\IZ}{\ensuremath{\mathbb{Z}}} 
\newcommand{\IN}{\ensuremath{\mathbb{N}}} 
\newcommand{\IS}{\ensuremath{\mathbb{S}}} 
\newcommand{\IC}{\ensuremath{\mathbb{C}}} 
\newcommand{\IB}{\ensuremath{\mathbb{B}}} 

\newcommand{\bR}{\mathbb{R}}
\newcommand{\cB}{\mathcal{B}}
\newcommand{\cO}{\mathcal{O}}
\newcommand{\cG}{\mathcal{G}}
\newcommand{\rM}{\mathrm{M}}
\newcommand{\rC}{\mathrm{C}}
\newcommand{\rV}{\mathrm{V}}

\newcommand{\lee}{\leqslant}
\newcommand{\gee}{\geqslant}
\newcommand{\ceil}[1]{{\left\lceil{#1}\right\rceil}}
\newcommand{\floor}[1]{{\left\lfloor{#1}\right\rfloor}}
\newcommand{\prob}[1]{{\mbox{\tt Pr}[#1]}}
\newcommand{\set}[1]{{\{ #1 \}}}
\newcommand{\seq}[1]{{\left< #1 \right>}}
\newcommand{\provided}{\,|\,}
\newcommand{\poly}{\mbox{\rm poly}}
\newcommand{\polylog}{\mbox{\rm \scriptsize polylog}\,}
\newcommand{\comb}[2] {\left(\!\!\begin{array}{c}{#1}\\{#2}\end{array}\!\!\right)}




\newcounter{probcnt}
\newcommand{\مسئله}[1]{\stepcounter{probcnt}\bigskip\bigskip{
 	\large \bf مسئله‌ی \arabic{probcnt}$\mbox{\bf{.}}$ \ #1} \bigskip}

\newcommand{\fqed}[1]{\leavevmode\unskip\nobreak\quad\hspace*{\fill}{\ensuremath{#1}}}

\newenvironment{اثبات}
	{\begin{trivlist}\item[\hskip\labelsep{\em اثبات.}]}
	{\fqed{\square}\end{trivlist}}

\newenvironment{حل}
	{\begin{trivlist}\item[\hskip\labelsep{\bf حل.}]}
	{\fqed{\blacktriangleright}\end{trivlist}}

\ifdefined\hidesols
	\newsavebox{\trashcan} % uncomment the following line to hide solutions
	\renewenvironment{حل}{\begin{latin}\begin{lrbox}{\trashcan}}{\end{lrbox}\end{latin}}
\fi


%------------------------- Header -----------------------------

%------------------------- Header -----------------------------

\newcommand{\سربرگ}[3]{
	\parindent=0em
	\begin{shaded}
		
		\rightline{ 
			\makebox[6em][c]{
				\includegraphics[height=1.6cm]{aut.png}
		}}
		\vspace{-.2em}
		{\scriptsize\bf دانشکده‌ی مهندسی کامپیوتر}
		\hfill {\small
			مدرس: دکتر امیرمزلقانی  \ 
		}\\[-5em]
		\leftline{\hfill\Large\bf 
			کاربرد های جبر خطی
		}\\[.7em]
		\leftline{\hfill\bf 
			نیم‌سال دوم ۹۶-۹۷
		}\\[1.7em]
		\hrule height .12em
		
		\normalsize
		\vspace{1mm} #1
		\hfill \small  #3
		\vspace{1mm} 
		\hrule height .1em
		
		\vspace{-0.5em} 
		\hfill {\sayeh\large #2} \hfill
	\end{shaded}
	\begin{large}

توجه:
\end{large}
\\
\\
\begin{itemize}
\item
این تمرین از مباحث مربوط به فصل 3 و 4 طراحی شده است که شامل 8 سوال اجباری و 2 سوال امتیازی است که نمره سوال های امتیازی فقط به نمرات تمرین شما کمک می کند.

\item 
اگه سوالی داشتین از طریق
\begin{latin}
\href{mailto:aut.la2018@gmail.com}{\[aut.la2018@gmail.com\]}
\end{latin}
 حتما بپرسید.

\item پاسخ های تمرین را در قالب یک فایل به صورت الگوی زیر آپلود کنید.
\begin{latin}
9531000\_Jonatan\_Vannieuwenhoven\_HW3.pdf
\end{latin}
\item \color{red}
مهلت تحویل جمعه 21 اردیبهشت 1397 ساعت 23:54:59
\end{itemize}

}




%\settextfont{Yas}
\begin{document}
\سربرگ{ فصل دوم (جبر ماتریسی)}{}{}
\clearpage

\مسئله{}  


الف) اگر $u = \begin{bmatrix}3 \\ 0\end{bmatrix}$ و $v = \begin{bmatrix}1\\ 2 \end{bmatrix}$ باشند. مساحت متواضی‌الاضلاع ساخته شده توسط $u$ و $v$ و $u + v$ و $0$ را بدست بیاورید.
\\\\
دترمینان $\begin{bmatrix}u & v\end{bmatrix}$ را بدست آورده و با مساحت متواضی‌الاضلاع مقایسه کنید. 
\\\\
درایه اول بردار $v$ را با یک مقدار دلخواه جایگزین کنید و تغیرات را مشاهده و تحیلل کنید.
\\\\
ب) نشان دهید معادله یک خط در ${\mathbb R}^2$ که از دو نقطه مشخص
$(x_1, y_1)$ و $(x_2, y_2)$میگذرد را میتوان به صورت زیر نوشت:

\[det(
\begin{bmatrix}
    1       & x        & y \\
    1       & x_1      & y_1 \\
    1       & x_2      & y_2 \\
\end{bmatrix}
)=0\]


\مسئله{}

معکوس ماتریس‌های زیر را به روش گوس جردن بدست بیاورید و مراحل آن را نیز بنویسید.
\\
\[A=
\begin{bmatrix}
    7       & 2      & 1 \\
    0       & 3      & -1 \\
    -3      & 4      & -2 \\
\end{bmatrix}
B=
\begin{bmatrix}
    1       & -2     & 1 \\
    4       & -7     & 3 \\
    -2      & 6      & -4 \\
\end{bmatrix}
\]

\مسئله{} 


درستی یا نادرستی گزاره های زیر را مشخص کنید و درصورت نادرست بودن مثال نقض ارائه دهید و در صورت درست بودن آن را اثبات کنید:
\begin{enumerate}
	\item اگر معکوس $A^2$ برابر $B$  باشد معکوس $A$ برابر $AB$ است.
\\
\item اگر $B$ و $C$ ماتریس‌هایی $m \times n$ باشند و $D$ معکوس پذیر باشد  و داشته باشیم 
$(B - C)D = 0$ ، $B = C$ است.
\\
\item اگر $A = BCD$ و داشته باشیم که $A$ معکوس‌پذیر است. $D$ و $C$ و $B$ هر سه معکوس پذیرند.
\\
\item اگر ماتریس $B$ یک ماتریس $3 \times 3$ باشد، $A = B^4 + 3B^2 + 7B + 3I_{3}$ معکوس پذیر است.

\end{enumerate}


\مسئله{}
به روش تجزیه $LU Factorization$ ماتریس زیر را تجزیه کرده و
ماتریس‌های $L$ و $U$ را به دست آورید. 
\\
\[A=
\begin{bmatrix}
    1       & 1      & 1 \\
    4       & 3      & -1 \\
    3       & 5      & 3 \\
\end{bmatrix}
\]

	\clearpage
{\LARGE{\textbf{سوالات شبیه سازی}}}

\مسئله

ماتریس متراکم (ماتریسی که بیشتر درایه های آن -مثلا بیش از نیمی از آنها- غیر صفر باشد) و معکوس پذیر
$A$
با ابعاد
$n \times n $
را در نظر بگیرید، روش استاندارد حل دستگاه معادله خطی 
$ Ax = b $
به صورت زیر است:
\begin{enumerate}
	\item 
	تجزیه
LU
	  ماتریس
A
	   را بیاید:
	$ A = LU$.
	
	\item 
	اگر 
	$\hat{x} := Ux$
	سیستم
	$L\hat{x} = b$
	(که در آن L یک ماتریس پایین مثلثی است) را از طریق جایگزینی پیشرو 
 $(forward substitution)$
حل کنید.
\item 
	سیستم بالا مثلثی 
$Ux= \hat{x} $
را(که در آن U یک ماتریس بالا مثلثی است) از طریق جایگزینی عقب گرد 
$(back substitution)$
حل کنید.
\end{enumerate}



\begin{latin}
	\begin{enumerate}[(a)]
		\begin{RTLitems}
			\item  \rl{
				تابعی بنویسید که تجزیه  LU ماتریس A را پیدا کند. فرض کنید که می توان ماتریس A را بدون استفاده از عمل جا به جایی دو سطر 
				$ row-interchange$
				از بین اعمال سطری مقدماتی
				به ماتریس بالا مثلثی
				$U$
				تبدیل کرد.
				تابع شما باید ماتریس A  را به عنوان ورودی بگیرد و ماتریس پایین مثلثی
				L
				و ماتریس بالا مثلثی
				U
				را باز گرداند.}
				\begin{latin}
				\begin{lstlisting}[language=Matlab]
Function [L, U] = lu_factor(A)
[n , n1] = size(A);
if n ~= n1 
	error ("A must be square")
end
L = eye (n)
U = zeros (n)

...

return;
				\end{lstlisting}
			\end{latin}
\rl{
				در کد بالا شما باید قسمت ....  را تکمیل کنید. برای این منظور تنها مقادیر بالای قطر اصلی ماتریس
				$U$
				که مقدار اولیه صفر گرفته است و مقادیر پایین قطر اصلی ماتریس L که برابر ماتریس همانی است را آپدیت کنید.	}
			

			\item  \rl{
			تابع دیگری بنویسید که معادله 
			Ax = b
			را از طریق مراحل ۱و ۲و ۳ را که در بالا ذکر شده است، حل کند. 
			تابع شما باید به شکل زیر باشد:}
			\begin{latin}
			\begin{lstlisting}[language=Matlab]

function x =  linear_sys_solver(A,b)

% compute the LU factorization of A
% Solve Ly = b for y by forward substitution
% Solve Ux = y by back substitution

return;
			\end{lstlisting}
		\end{latin}
			\rl{
	می توانید از کد خود مروبط به سوال ۸ تمرین اول در این بخش استفاده کنید	
}
			\item  \rl{
		تابع 
		myinverse
		را برای محاسبه وارون ماتریس 
		$A$
		با سایز 
		$n \times n $
		بنویسید. توجه کنید که باید از تابع 
		lu\_factor
		و توابع
		$forward substitution$
		و 
		$backward substitution$
		که در بخش های قبل یا تمرین قبل نوشته اید استفاده کنید.
		فرض کنید که 
		$ X = A^{-1}$
		پس
		$AX = I_n$
		از این نکته که 		
		$ Ax_i= e_i$ 
		به ازای
		$ i = 1 , ... , n$
		که در آن 
		$x_i$
		ستون 
		$i$
		ام ماتریس 
		$X$
		و 
		$e_{i}$
		ستون
		$i$
		ام ماتریس 
		$I_n$
		استفاده کنید.
		(توجه کنید که شما تنها یک بار می تواند از تجزیه 
		LU
		ماتریس A را محاسبه کنید)	
		}
	\item  \rl{
ماتریس هیلبرت یک ماتریس مربعی است به گونه ای که 
$H_{i,j} = \frac {1}{1+i+j}$
ماتریس هیلبرت A از مرتبه ۵ و ۱۰ و۱۵و ۲۰ را بسازید و ماتریس وارون آنها 
$(A^{-1})$
 را از طریق توابع آماده ( مثلا در متلب از طریق تابع 
inv
)
به دست آورید.
سپس ماتریس وارون آن را از طریق تابع 
myinverse 
 که در بخش قبل نوشته اید به دست آورید (آن را
 $A'^{-1}$
 بنامید)
 مقادیر 
 $AA^{-1}$
 و 
 $AA'^{-1}$
 را به دست آورید و نتایج را مقایسه و تحلیل کنید. (راهنمایی:‌ به ویژگی های ماتریس های 
 ill-conditioned
 توجه کنید
 )
}

		\end{RTLitems}    
	\end{enumerate}
\end{latin}



\end{document}
