
\documentclass{article}
\usepackage{amsmath}


\usepackage{graphicx,comment,framed}
\usepackage{roundbox}
\usepackage{fancybox}
\usepackage{tikz}
\usepackage{color}
\usepackage[hidelinks]{hyperref}
\usepackage{framed}
\usepackage{amsthm,amssymb,amsmath}
%\usepackage[colorlinks,linkcolor=blue,citecolor=blue]{hyperref}
\definecolor{shadecolor}{cmyk}{0,0,0,0}
\usepackage{listings}
\usepackage{xepersian}
\usepackage[noend]{algpseudocode}


%--------------------- page settings ----------------------

\settextfont[Scale=1.1]{XB Niloofar}
\setdigitfont[Scale=1.1]{XB Niloofar}
\defpersianfont\sayeh[Scale=1.1]{XB Niloofar}
\addtolength{\textheight}{3.2cm}
\addtolength{\topmargin}{-22mm}
\addtolength{\textwidth}{3cm}
\addtolength{\oddsidemargin}{-1.5cm}


%------------------------ Environments ------------------------------------

\newtheorem{قضیه}{قضیه}
\newtheorem{لم}{لم}
\newtheorem{مشاهده}{مشاهده}
\newtheorem{تعریف}{تعریف}


%-------------------------- Notations ------------------------------------

\newcommand{\IR}{\ensuremath{\mathbb{R}}} 
\newcommand{\IZ}{\ensuremath{\mathbb{Z}}} 
\newcommand{\IN}{\ensuremath{\mathbb{N}}} 
\newcommand{\IS}{\ensuremath{\mathbb{S}}} 
\newcommand{\IC}{\ensuremath{\mathbb{C}}} 
\newcommand{\IB}{\ensuremath{\mathbb{B}}} 

\newcommand{\bR}{\mathbb{R}}
\newcommand{\cB}{\mathcal{B}}
\newcommand{\cO}{\mathcal{O}}
\newcommand{\cG}{\mathcal{G}}
\newcommand{\rM}{\mathrm{M}}
\newcommand{\rC}{\mathrm{C}}
\newcommand{\rV}{\mathrm{V}}

\newcommand{\lee}{\leqslant}
\newcommand{\gee}{\geqslant}
\newcommand{\ceil}[1]{{\left\lceil{#1}\right\rceil}}
\newcommand{\floor}[1]{{\left\lfloor{#1}\right\rfloor}}
\newcommand{\prob}[1]{{\mbox{\tt Pr}[#1]}}
\newcommand{\set}[1]{{\{ #1 \}}}
\newcommand{\seq}[1]{{\left< #1 \right>}}
\newcommand{\provided}{\,|\,}
\newcommand{\poly}{\mbox{\rm poly}}
\newcommand{\polylog}{\mbox{\rm \scriptsize polylog}\,}
\newcommand{\comb}[2] {\left(\!\!\begin{array}{c}{#1}\\{#2}\end{array}\!\!\right)}




\newcounter{probcnt}
\newcommand{\مسئله}[1]{\stepcounter{probcnt}\bigskip\bigskip{
 	\large \bf مسئله‌ی \arabic{probcnt}$\mbox{\bf{.}}$ \ #1} \bigskip}

\newcommand{\fqed}[1]{\leavevmode\unskip\nobreak\quad\hspace*{\fill}{\ensuremath{#1}}}

\newenvironment{اثبات}
	{\begin{trivlist}\item[\hskip\labelsep{\em اثبات.}]}
	{\fqed{\square}\end{trivlist}}

\newenvironment{حل}
	{\begin{trivlist}\item[\hskip\labelsep{\bf حل.}]}
	{\fqed{\blacktriangleright}\end{trivlist}}

\ifdefined\hidesols
	\newsavebox{\trashcan} % uncomment the following line to hide solutions
	\renewenvironment{حل}{\begin{latin}\begin{lrbox}{\trashcan}}{\end{lrbox}\end{latin}}
\fi


%------------------------- Header -----------------------------

%------------------------- Header -----------------------------

\newcommand{\سربرگ}[3]{
	\parindent=0em
	\begin{shaded}
		
		\rightline{ 
			\makebox[6em][c]{
				\includegraphics[height=1.6cm]{aut.png}
		}}
		\vspace{-.2em}
		{\scriptsize\bf دانشکده‌ی مهندسی کامپیوتر}
		\hfill {\small
			مدرس: دکتر امیرمزلقانی  \ 
		}\\[-5em]
		\leftline{\hfill\Large\bf 
			کاربرد های جبر خطی
		}\\[.7em]
		\leftline{\hfill\bf 
			نیم‌سال دوم ۹۶-۹۷
		}\\[1.7em]
		\hrule height .12em
		
		\normalsize
		\vspace{1mm} #1
		\hfill \small  #3
		\vspace{1mm} 
		\hrule height .1em
		
		\vspace{-0.5em} 
		\hfill {\sayeh\large #2} \hfill
	\end{shaded}
	\begin{large}

توجه:
\end{large}
\\
\\
\begin{itemize}
\item
این تمرین از مباحث مربوط به فصل 3 و 4 طراحی شده است که شامل 8 سوال اجباری و 2 سوال امتیازی است که نمره سوال های امتیازی فقط به نمرات تمرین شما کمک می کند.

\item 
اگه سوالی داشتین از طریق
\begin{latin}
\href{mailto:aut.la2018@gmail.com}{\[aut.la2018@gmail.com\]}
\end{latin}
 حتما بپرسید.

\item پاسخ های تمرین را در قالب یک فایل به صورت الگوی زیر آپلود کنید.
\begin{latin}
9531000\_Jonatan\_Vannieuwenhoven\_HW3.pdf
\end{latin}
\item \color{red}
مهلت تحویل جمعه 21 اردیبهشت 1397 ساعت 23:54:59
\end{itemize}

}




%\settextfont{Yas}
\begin{document}
\سربرگ{تمرین آشنایی با متلب}{}{}
\clearpage
\مسئله{متغیر های اسکالر}

متغیر های زیر را ایجاد کنید:
\lr{
\begin{enumerate}
	\item $a = 10$
	\item $b = 1023 * 2.5$
	\item $c = 2+3j$ \text{عدد موهومی در اعداد مختلط است}
	\item $d = e^{j2pi}$  (use \textbf{exp}, \textbf{pi})
\end{enumerate}
}
j) عدد موهومی در اعداد مختلط است)

\مسئله{متغیر های برداری}
\begin{latin}
	\begin{enumerate}
		\item aٰVec = [3.14 15 9 26]
		
		\item bVec = $\begin{bmatrix}
			2.17 \\
			8 \\
			28 \\
			182
		\end{bmatrix}$
		
		\item cVec = [5 4.8 4.6  \dots  -4.8 -5] 
		
		(all the numbers from 5 to -5 in increments of -0.2)
		
		\item dVec = [$10^0$ $10^{0.01}$ $10^{0.02}$ \dots $10^{0.99}$ $10$ ] 
		
		(logarithmically spaced numbers between 1and 10, use \textbf{logspace}, make sure you get the length right!)
		
		\item eVec = Hello 
		
		( eVec is a string, which is a vector of characters)
	\end{enumerate}
\end{latin}

\مسئله{متغییر های ماتریسی}

\begin{latin}
	\begin{enumerate}
		\item
		$
		 aٰMat = \begin{bmatrix}
			2  & \dots &2 \\
			\vdots  & \ddots & \vdots \\
			2  & \dots &2 \\
		\end{bmatrix}
		$ 
		
		(a 9x9 matrix full of 2’s (use \textbf{ones} or \textbf{zeros}))
		\item
		$
		bMat = \begin{bmatrix}
			1  & \dots &\dots &\dots &0 \\
			\vdots  & \ddots &0&\ddots&\vdots \\
			\vdots  & 0&5&0&\vdots\\
			\vdots  & \ddots&0& \ddots&\vdots\\
			0  & \dots & \dots& \dots &1 \\
		\end{bmatrix}
		$ 
		
		(a 9x9 matrix of all zeros, but with the values [1 2 3 4 5 4 3 2 1] on the main diagonal (use \textbf{zeros},\textbf{diag}).)
		
		\item
		$
		cMat = \begin{bmatrix}
		1  & 11 &\dots &\dots &91 \\
		2  & 12 &\ddots&\ddots&92 \\
		\vdots  & \ddots&\ddots&\ddots&\vdots\\
		9  & 19&\ddots& \ddots&99\\
		10  & 20 & \dots& \dots &100 \\
		\end{bmatrix}
		$ 
		
		(a 10x10 matrix where the vector 1:100 runs down the columns (use \textbf{reshape}).)

	\end{enumerate}
\end{latin}
\مسئله{معادلات اسکالر}
\begin{latin}
	\begin{enumerate}
		\item x = {\LARGE${\frac{1}{1 + e^{-(a-15)/6}}}$}
		\item y = $(\sqrt{a} + \sqrt[21]{b}) ^{pi}$
	\end{enumerate}
\end{latin}

\مسئله{معادلات برداری}

در این قسمت از معادلات بخش 2 و عملگرهای عنصر نهاد
(\lr{element wise})
 ( مثل $.*$ و $./$ و...) استفاده کنید. (برای مثال، عناصر xVec حاصل محاسبه فرمول فوق بر روی تک تک عناصر cVec هستند)
\begin{latin}
	\begin{enumerate}
		\item xVec = {\LARGE${\frac{1}{\sqrt{2pi * 2.5}}}e^{{-cVec^2}/2}$}
		\item yVec = $\sqrt{(aVec)^T + bVec^2}$
	\end{enumerate}
\end{latin}
\مسئله{معادلات ماتریسی}

در این قسمت از عملگرهای ماتریسی استفاده کنید.
\begin{latin}
	\begin{enumerate}
		\item xMat = $(aVec * bVec) * aVec^T$
	\end{enumerate}
\end{latin}
\مسئله{ماتریس ستونی ای تشکیل دهید که میانگین هر سطر cMat را در خود ذخیره کند. همچنین میانگین کلی ماتریس را نیز پیدا کرده و در متغیری ذخیره و نمایش دهید. (‌از تابع \textbf{mean} استفاده کنید)}

\end{document}