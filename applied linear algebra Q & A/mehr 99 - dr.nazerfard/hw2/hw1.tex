
\documentclass{article}



\usepackage{graphicx,comment,framed}
\usepackage{roundbox}
\usepackage{fancybox}
\usepackage{tikz}
\usepackage{color}
\usepackage[hidelinks]{hyperref}
\usepackage{framed}
\usepackage{amsthm,amssymb,amsmath}
%\usepackage[colorlinks,linkcolor=blue,citecolor=blue]{hyperref}
\definecolor{shadecolor}{cmyk}{0,0,0,0}
\usepackage{listings}
\usepackage{xepersian}
\usepackage[noend]{algpseudocode}


%--------------------- page settings ----------------------

\settextfont[Scale=1.1]{XB Niloofar}
\setdigitfont[Scale=1.1]{XB Niloofar}
\defpersianfont\sayeh[Scale=1.1]{XB Niloofar}
\addtolength{\textheight}{3.2cm}
\addtolength{\topmargin}{-22mm}
\addtolength{\textwidth}{3cm}
\addtolength{\oddsidemargin}{-1.5cm}


%------------------------ Environments ------------------------------------

\newtheorem{قضیه}{قضیه}
\newtheorem{لم}{لم}
\newtheorem{مشاهده}{مشاهده}
\newtheorem{تعریف}{تعریف}


%-------------------------- Notations ------------------------------------
\renewcommand{\labelitemi}{$\bullet$}
\newcommand{\IR}{\ensuremath{\mathbb{R}}} 
\newcommand{\IZ}{\ensuremath{\mathbb{Z}}} 
\newcommand{\IN}{\ensuremath{\mathbb{N}}} 
\newcommand{\IS}{\ensuremath{\mathbb{S}}} 
\newcommand{\IC}{\ensuremath{\mathbb{C}}} 
\newcommand{\IB}{\ensuremath{\mathbb{B}}} 

\newcommand{\bR}{\mathbb{R}}
\newcommand{\cB}{\mathcal{B}}
\newcommand{\cO}{\mathcal{O}}
\newcommand{\cG}{\mathcal{G}}
\newcommand{\rM}{\mathrm{M}}
\newcommand{\rC}{\mathrm{C}}
\newcommand{\rV}{\mathrm{V}}

\newcommand{\lee}{\leqslant}
\newcommand{\gee}{\geqslant}
\newcommand{\ceil}[1]{{\left\lceil{#1}\right\rceil}}
\newcommand{\floor}[1]{{\left\lfloor{#1}\right\rfloor}}
\newcommand{\prob}[1]{{\mbox{\tt Pr}[#1]}}
\newcommand{\set}[1]{{\{ #1 \}}}
\newcommand{\seq}[1]{{\left< #1 \right>}}
\newcommand{\provided}{\,|\,}
\newcommand{\poly}{\mbox{\rm poly}}
\newcommand{\polylog}{\mbox{\rm \scriptsize polylog}\,}
\newcommand{\comb}[2] {\left(\!\!\begin{array}{c}{#1}\\{#2}\end{array}\!\!\right)}




\newcounter{probcnt}
\newcommand{\مسئله}[1]{\stepcounter{probcnt}{
 	\bf \arabic{probcnt}$\mbox{\bf{.}}$ \ #1}}

\newcommand{\fqed}[1]{\leavevmode\unskip\nobreak\quad\hspace*{\fill}{\ensuremath{#1}}}

\newenvironment{اثبات}
	{\begin{trivlist}\item[\hskip\labelsep{\em اثبات.}]}
	{\fqed{\square}\end{trivlist}}

\newenvironment{حل}
	{\begin{trivlist}\item[\hskip\labelsep{\bf حل.}]}
	{\fqed{\blacktriangleright}\end{trivlist}}

\ifdefined\hidesols
	\newsavebox{\trashcan} % uncomment the following line to hide solutions
	\renewenvironment{حل}{\begin{latin}\begin{lrbox}{\trashcan}}{\end{lrbox}\end{latin}}
\fi


%------------------------- Header -----------------------------

%------------------------- Header -----------------------------

\newcommand{\سربرگ}[3]{
	\parindent=0em
	
	
	\begin{shaded}
		
		\rightline{ 
			\makebox[8em][c]{
				\includegraphics[height=3cm]{aut.png}
		}} \ \
	\\[-3em] 
	\centerline{ مدرس :دکتر ناظر فرد } 
	\\[-6.5em]
	\centerline{\large \bf جبرخطی کاربردی }
	\\[0.05em]
	\centerline {\bf نیمسال دوم 96-97}	
	\\[-6.4em]
		\leftline{ 
		\makebox[8em][c]{
			\includegraphics[height=3.2cm]{ceit.jpg}
	}}
	
	
		

		\hrule height .12em
		
		\normalsize
		\vspace{1mm} #1
		\hfill \small  #3
		\vspace{1mm} 
		\hrule height .1em
		
		\vspace{-0.5em} 
		\hfill {\sayeh\large #2} \hfill
	\end{shaded}
	\begin{large}
	

\bf{توجه!!! :}
\end{large}
\\
\begin{itemize}
\item
سری دوم تمرینات با موضوع جبر ماتریسی و دترمینان را در زیر مشاهده می کنید. 
\item
این سری تمرین شامل 11 سوال نظری  است که سوالات شبیه سازی نیز به زودی در اختیار شما قرار خواهد گرفت.


\item 
پس از حل مسائل آن ها را به صورت یک فایل 
\lr{pdf}
 در قسمت مورد نظر آپلود کنید همچنین تمرینات عملی و شبیه سازی را نیز در یک پوشه قرار دهید و در قسمت در نظر گرفته شده با توجه به اصول ارسال تمارین که در کانال و مودل قرار گرفته است ارسال کنید. 
\item
تمرینات نظری را به شکل:\\
\lr{\[9531000\_T\_Giovanni\ \  van Bronckhorst\_HW2.pdf\]}
\\
و تمرینات عملی و شبیه سازی را به شکل:\\
\lr{\[9531000\_S\_Giovanni\ \  van Bronckhorst\_HW2.pdf\]}
ارسال فرمایید.
\item
\color{red}
مهلت تحویل تمارین ساعت 23:55 روز جمعه 97/2/28 خواهد بود.
\end{itemize}
{\bf تمارین:}
\\
\\
}





\begin{document}
\سربرگ{مجموعه سوالات فصل 2و3 (جبر ماتریسی و دترمینان)}{}{}


\مسئله{} 
ابتدا دترمینان ماتریس های زیر را بیابید سپس با استفاده از دو روش استفاده از ماتریس های مقدماتی و ماتریس الحاقی (adjugate) وارون آن هارا در صورت وجود بیابید.
 
$$\begin{bmatrix}
1&1&3\\
-2&2&1\\
0&1&1
\end{bmatrix}\quad 
\begin{bmatrix}
-2&-7&-9\\
2&5&6\\
1&3&4\\
\end{bmatrix}\quad 
\begin{bmatrix}
1&-2&1\\
4&-7&3\\
-2&6&-4\\
\end{bmatrix}\quad
\begin{bmatrix} 
0&1&1&\cdots&1\\
1&0&1&\cdots&1\\
1&1&0&\cdots&1\\
\vdots&\vdots&\vdots&\ddots&\vdots\\
1&1&1&\cdots&0
\end{bmatrix}_{n\times n}$$$$
\begin{bmatrix} 
0&a_1&1&\cdots&0\\
0&0&a_2&\cdots&0\\
\vdots&\vdots&\ddots&\ddots&0\\
0&0&0&\cdots&a_{n-1}\\
a_n&0&0&\cdots&0
\end{bmatrix}_{n\times n}
$$
$a_i$
ها پارامتر و غیر صفر هستند.


\مسئله{}
درستی یا نادرستی گزاره های زیر را مشخص کنید برای گزاره های نادرست مثال نقض و برای گزاره های درست اثبات ارائخ دهید
\begin{enumerate}
\item 
اگر 
$A,B$
ماتریس های 
$n\times n $
و 
$AB-BA=A$
آنگاه 
$A$
معکوس پذیر نیست.
\item 
	ماتریسی را پاد متقارن گوییم که برابر قرینه ترانهاده اس باشد ،اگر 
	$A$
	ماتریسی 
	$n\times n$
	باشد آنگاه اگر 
	$n$
	زوج باشد
	$det A=0$.
	\item 
	اگر 
	$A$
	یک ماتریس بالا مثلثی باشد اگر هیچ یک از درایه های روی قطر اصلی آن برابر صفر نباشد آنگاه هم ارز سطری 
	$I_n$
	است.
\end{enumerate}
\مسئله{}
فرض کنید 
$A$
یک ماتریس 
$n\times n$
باشد که درایه های آن 
$-1,1$
تشکیل شده باشد ،ثابت کنید 
$2^{n-1}$
عاد می کند 
$det A$
را .

\مسئله {}
اگر یک ماتریس 
$A$
یک ماتریس 
$n\times n $
باشد 
گزاره های زیر را ثابت کنید:
\begin{enumerate}
\item 
	$adj(A^t)=(adj A)^t$
\item 
$adj  A^{-1}=(adj A)^{-1}$
\item 
اگر 
$A$
ماتریس قطری باشد انگاه 
$adj A$
نیز قطری است.
\item 
نشان دهید اگر 
$I-AB$
معکوس پذیر باشد آنگاه 
$I-BA$
نیز معکوس پذیر است.
\item 
نشان دهید :
$|adj(adj(A))|=|A|^{(n-1)^2}$
	
\end{enumerate}
\مسئله {}
فرض کنید 
$A$
یک ماتریس 
$n\times n$
باشد و 
$A=a_{(ij)}$
به طوری که 
به ازای هر 
$1\leq i\leq n $
داشته باشیم 
\\
$\sum_{j=1}^{n}a_{ij}=a$
که 
$a$
مستقل از درایه های ماتریس است.اگر 
$A^2=I$
مقدار 
$a$
را محاسبه کنید.

\مسئله {}
فرض کنید 
$A,B,C$
ماتریس هایی 
$n\times n$
 باشند ،به ازای چه 
 $n$
 هایی رابطه زیر برقرار است :
 $$(AB-BA)^2C=C(AB-BA)^2$$
 \مسئله {}
 دترمینان ماتریس های زیر را بیابید:
 $$\begin{bmatrix}
 a+b&ab&0&\cdots&0&0\\
 1&a+b&ab&\cdots&0&0\\
 0&1&a+b&\cdots&0&0\\
 \cdots&\cdots&\cdots&\cdots&\cdots&\cdots\\
 0&0&0&\cdots&a+b&ab\\
 0&0&0&\cdots&1&a+b
 \end{bmatrix}_{n\times n}\quad
 \begin{bmatrix}
 1&1&1&\cdots&1\\
 a_1&a_2&a_3&\cdots&a_n\\
 a_1^2&a_2^2&a_3^2&\cdots&a_n^2\\
 \cdots&\cdots&\cdots&\cdots&\cdots\\
 a_1^{n-1}&a_2^{n-1}&a_3^{n-1}&\cdots&a_n^{n-1}
\end{bmatrix}_{n\times n}\\
 $$
 $$
 \begin{bmatrix}
 1&1&1&\cdots&1\\
 1&2&2&\cdots&2\\
 1&2&3&\cdots&3\\
 \vdots&\vdots&\vdots&\ddots&\vdots\\
 1&2&3&\cdots&n
 \end{bmatrix}_{n\times n}\quad
  $$
 \مسئله {}فرض کنید تمامی ماتریس های زیر وارون پذیر باشند نشان دهید :
 $$(A-B)^{-1}=A^{-1}+A^{-1}(B^{-1}-A^{-1})^{-1}A^{-1}$$
 همچنین در حالت خاص نشان دهید :
 $$(I+A)^{-1}=I-(A^{-1}+I)^{-1}$$
 و نشان دهید:
 $$|(I+A)^{-1}+(I+A^{-1})^{-1}|=1$$
 \مسئله{}
جواب دستگاه معادلات زیر را به روش کرامر بیابید:
\begin{equation*}
\left\{
\begin{array}{rl}
x_1+x_2&=3\\
-3x_1+2x_3&=0\\
x_2-2x_3&=2
\end{array} \right.\qquad
\left\{
\begin{array}{rl}
x_1+3x_2+x_3&=4\\
-x_1+2x_3&=2\\
3x_1+x_2&=2
\end{array} \right.
\end{equation*}
\مسئله {}دستگاه معادلات  زیر را با استفاده از تجزیه 
$LU$
حل کنید همچینین وارون ماتریس افزوده را با تجزیه 
$LU$
به دست آورید.
\begin{equation*}
\left\{
\begin{array}{rl}
	x_1+3x_2+3x_4&=8\\
	2x_1+X_2-2x_3+x_4&=7\\
	3x_1-x_2+x_3+2x_4&=14\\
	-x_1+2x_2+3x_3-x_4&=-7
\end{array} \right.
\end{equation*}
\مسئله{}
فرض کنید 
$T:\mathbb{R}^3\longrightarrow \mathbb{R}^3$
یک تبدیل خطی با ماتریس 
$\begin{bmatrix}
a&0&0\\
0&b&0\\
0&0&c
\end{bmatrix}$
باشد که 
$a,b,c$
مقادیری مثبت باشند،
$S$
را کره واحد در نظر بگیرید که سطح آن با معادله 
$x_1^2+x_2^2+x_3^2=1$
محدود شده است،
\begin{enumerate}
	\item 
نشان دهید 
$T(S)$
با بیضی به معادله 
$\frac{x_1^2}{a^2}+\frac{x_2^2}{b^2}+\frac{x_3^2}{c^2}=1$
محدود شده است.
\item 
با این فرض که حجم کره واحد 
$4\pi/3$
است حجم بیضی مطرح شده در قسمت 
$1$
را بیابید.
\end{enumerate}
\end{document}