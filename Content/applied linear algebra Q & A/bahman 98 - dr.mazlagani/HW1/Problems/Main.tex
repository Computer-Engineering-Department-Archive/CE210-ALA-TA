
\documentclass{article}



\usepackage{graphicx,comment,framed}
\usepackage{roundbox}
\usepackage{fancybox}
\usepackage{tikz}
\usepackage{color}
\usepackage[hidelinks]{hyperref}
\usepackage{framed}
\usepackage{amsthm,amssymb,amsmath}
%\usepackage[colorlinks,linkcolor=blue,citecolor=blue]{hyperref}
\definecolor{shadecolor}{cmyk}{0,0,0,0}
\usepackage{listings}
\usepackage{xepersian}
\usepackage[noend]{algpseudocode}


%--------------------- page settings ----------------------

\settextfont[Scale=1.1]{XB Niloofar}
\setdigitfont[Scale=1.1]{XB Niloofar}
\defpersianfont\sayeh[Scale=1.1]{XB Niloofar}
\addtolength{\textheight}{3.2cm}
\addtolength{\topmargin}{-22mm}
\addtolength{\textwidth}{3cm}
\addtolength{\oddsidemargin}{-1.5cm}


%------------------------ Environments ------------------------------------

\newtheorem{قضیه}{قضیه}
\newtheorem{لم}{لم}
\newtheorem{مشاهده}{مشاهده}
\newtheorem{تعریف}{تعریف}


%-------------------------- Notations ------------------------------------

\newcommand{\IR}{\ensuremath{\mathbb{R}}} 
\newcommand{\IZ}{\ensuremath{\mathbb{Z}}} 
\newcommand{\IN}{\ensuremath{\mathbb{N}}} 
\newcommand{\IS}{\ensuremath{\mathbb{S}}} 
\newcommand{\IC}{\ensuremath{\mathbb{C}}} 
\newcommand{\IB}{\ensuremath{\mathbb{B}}} 

\newcommand{\bR}{\mathbb{R}}
\newcommand{\cB}{\mathcal{B}}
\newcommand{\cO}{\mathcal{O}}
\newcommand{\cG}{\mathcal{G}}
\newcommand{\rM}{\mathrm{M}}
\newcommand{\rC}{\mathrm{C}}
\newcommand{\rV}{\mathrm{V}}

\newcommand{\lee}{\leqslant}
\newcommand{\gee}{\geqslant}
\newcommand{\ceil}[1]{{\left\lceil{#1}\right\rceil}}
\newcommand{\floor}[1]{{\left\lfloor{#1}\right\rfloor}}
\newcommand{\prob}[1]{{\mbox{\tt Pr}[#1]}}
\newcommand{\set}[1]{{\{ #1 \}}}
\newcommand{\seq}[1]{{\left< #1 \right>}}
\newcommand{\provided}{\,|\,}
\newcommand{\poly}{\mbox{\rm poly}}
\newcommand{\polylog}{\mbox{\rm \scriptsize polylog}\,}
\newcommand{\comb}[2] {\left(\!\!\begin{array}{c}{#1}\\{#2}\end{array}\!\!\right)}




\newcounter{probcnt}
\newcommand{\مسئله}[1]{\stepcounter{probcnt}\bigskip\bigskip{
 	\large \bf مسئله‌ی \arabic{probcnt}$\mbox{\bf{.}}$ \ #1} \bigskip}

\newcommand{\fqed}[1]{\leavevmode\unskip\nobreak\quad\hspace*{\fill}{\ensuremath{#1}}}

\newenvironment{اثبات}
	{\begin{trivlist}\item[\hskip\labelsep{\em اثبات.}]}
	{\fqed{\square}\end{trivlist}}

\newenvironment{حل}
	{\begin{trivlist}\item[\hskip\labelsep{\bf حل.}]}
	{\fqed{\blacktriangleright}\end{trivlist}}

\ifdefined\hidesols
	\newsavebox{\trashcan} % uncomment the following line to hide solutions
	\renewenvironment{حل}{\begin{latin}\begin{lrbox}{\trashcan}}{\end{lrbox}\end{latin}}
\fi


%------------------------- Header -----------------------------

%------------------------- Header -----------------------------

\newcommand{\سربرگ}[3]{
	\parindent=0em
	\begin{shaded}
		
		\rightline{ 
			\makebox[6em][c]{
				\includegraphics[height=1.6cm]{aut.png}
		}}
		\vspace{-.2em}
		{\scriptsize\bf دانشکده‌ی مهندسی کامپیوتر}
		\hfill {\small
			مدرس: دکتر امیرمزلقانی  \ 
		}\\[-5em]
		\leftline{\hfill\Large\bf 
			کاربرد های جبر خطی
		}\\[.7em]
		\leftline{\hfill\bf 
			نیم‌سال دوم ۹۶-۹۷
		}\\[1.7em]
		\hrule height .12em
		
		\normalsize
		\vspace{1mm} #1
		\hfill \small  #3
		\vspace{1mm} 
		\hrule height .1em
		
		\vspace{-0.5em} 
		\hfill {\sayeh\large #2} \hfill
	\end{shaded}
	\begin{large}

توجه:
\end{large}
\\
\\
\begin{itemize}
\item
این تمرین از مباحث مربوط به فصل 3 و 4 طراحی شده است که شامل 8 سوال اجباری و 2 سوال امتیازی است که نمره سوال های امتیازی فقط به نمرات تمرین شما کمک می کند.

\item 
اگه سوالی داشتین از طریق
\begin{latin}
\href{mailto:aut.la2018@gmail.com}{\[aut.la2018@gmail.com\]}
\end{latin}
 حتما بپرسید.

\item پاسخ های تمرین را در قالب یک فایل به صورت الگوی زیر آپلود کنید.
\begin{latin}
9531000\_Jonatan\_Vannieuwenhoven\_HW3.pdf
\end{latin}
\item \color{red}
مهلت تحویل جمعه 21 اردیبهشت 1397 ساعت 23:54:59
\end{itemize}

}




%\settextfont{Yas}
\begin{document}
\سربرگ{تمرین اول}{}{}
\clearpage
\مسئله{}  ماتریس  های زیر متعلق ماتریس افزوده سه دستگاه معادله خطی است،در هرمرحله پس از مشخص  کردن جایگاه(درایه) و ستون محوری و با استفاده از روش حذفی گاوس جردن ماتریس ها را به شکل کاهش یافته سطری در بیاورید و سپس در مورد جواب دستگاه ها بحث کنید.(در صورت داشتن داشتن جواب عمومی،جواب ها را به صورت پارامتریک بنویسید.)
\\
الف)
${\begin{bmatrix}
		1&2&3&4&1\\
		0&1&2&3&1\\
		0&0&0&1&1\\
		0&1&0&1&1
		\end{bmatrix}}$
	\qquad
	ب)
$\begin{bmatrix}
1&-7&0&6&5\\
0&0&1&-2&-3\\
-1&7&-4&2&7
\end{bmatrix}
$
\qquad
ج)
$\begin{bmatrix}
3&-4&2&0\\
-9&12&-6&0\\
-6&8&-4&0
\end{bmatrix}$

\مسئله{}
مقدار 
		$\lambda$
		را طوری تعیین کنید که 
		\\
		الف) دستگاه معادلات زیر جواب ناصفر داشته باشد:
	\begin{equation*}
	 \left\{
	\begin{array}{rl}
	 \lambda x+3y=0\\
	 2x+4y=0
	\end{array} \right.
	\end{equation*}
	
	(بیش از آنکه جواب نهایی شما مهم باشد روش حل و نوع نگاه شما به آن مهم است.روشی که قابل تعمیم به مسائل دیگر باشد)
	\\
	ب) دستگاه معادلات زیر فقط جواب صفر داشته باشد :
		\begin{equation*}
	\left\{
	\begin{array}{rl}
	\lambda x_1+x_2+x_3=0\\
	x_1+\lambda x_2+x_3=0\\
	x_1+x_2+x_3=0
	\end{array} \right.
	\end{equation*}
	
\مسئله{} درستی یا نادرستی گزاره های زیر را مشخص کنید و درصورت نادرست بودن مثال نقض ارائه دهید و در صورت درست بودن آن را اثبات کنید:
\\
الف) اگر 
$\forall i\ \ v_i\in {\Bbb R}^n$
و 
$\{v_1,v_2,\ldots,v_n \}$
یک مجموعه وابسته خطی باشد ،هریک از $v_i$ ها را می توان به صورت یک ترکیب خطی از بقیه اعضا نوشت.
\\
ب) اگر $s_1 $ و $s_2 $  زیر مجموعه هایی از بردار های 
${\Bbb R}^n$
باشند که 
$span(s_1)=span(s_2)$
آنگاه 
$s_1=s_2$
.
\\
ج) اگر
 $A\subset {\Bbb R}^n$
 و $َA$ مستقل خطی باشد آنگاه :
 $span(A)={\Bbb R}^n$.
 \\
 د) اگر 
 $span(A)={\Bbb R}^n$
 آنگاه $A$ یک مجموعه مستقل خطی از
 ${\Bbb R}^n$
 است.
 \\
 ه) اگر 
 $w$
 یک ترکیب خطی از بردار های 
 $v_1,v_2,\ldots,v_n$
 باشد،آنگاه 
 $\{w,v_1,v_2,\ldots,v_n\}$
 مستقل خطی است.
 \\
 ی)اگر 
 $S \subseteq {\Bbb R}^n$
 مسنقل خطی باشد و 
 $v \in ({{\Bbb R}^n}-span(S))$
 آنگاه 
 $S\cup \{v\}$
 نیز مستقل خطی است.
 
 \مسئله{} مانریس سودوکو یک ماتریس 
 $9\times 9$
 که اعداد 
 $1,2,\ldots,9$
در هر سطر در هر ستون و هر بلوک 
$9\times 9$
آن ظاهر شده اند ، اگر $S$یک ماتریس سودوکو باشد به سوالات زیر پاسخ دهید:
\\
الف) حاصلضرب 
$S {\left(\begin{array}{ccc}
	1\\
	1\\
	\vdots\\
	1
\end{array}\right)_{9\times 1}}$
 را بیابید.
 \\
 ب)  کدام یک از اعمال سطری روی 
 $S$
 ،مجددا یک ماتریس سودوکو به ما می دهد؟
 
 \مسئله {} فرض کنید 
 $ََA$
یک ماتریس 
$m\times n$
است که:
\\
الف) برای هر $b$  در 
${\Bbb R}^n$
 معادله 
 $Ax=b$
 حداکثر یک جواب دارد،ثابت کنید ستون های ماتریس 
 $A$
 باید مستقل خطی باشند.
 \\
 ب) که $n$ تا از ستون های آن محوری هستند،ثابت کنید برای هر $b$  در 
 ${\Bbb R}^n$
 معادله ${\Bbb R}^n$ حداکثر یک جواب دارد.
 \newpage
 \مسئله {} مشخص کنید هریک از تبدیلات زیر خطی هستند یا نه،در صورتی که خطی باشند ماتریس استاندارد آن ها را نیز بیابید.\\
 الف)
 {\setlength\arraycolsep{0.1em}
 \begin{eqnarray*}
 	f:{\Bbb R}^2&\longrightarrow& {\Bbb R}^3\\
 	 (x,y)&\longmapsto&(x^2,2y)
 \end{eqnarray*}}
\\
ب) 
{\setlength\arraycolsep{0.1em}
	\begin{eqnarray*}
		f:{\Bbb R}^2\ \ &\longrightarrow&\ \ {\Bbb R}^2\\
		(x,y)&\longmapsto&(2x+y,-y)
\end{eqnarray*}}\\
ج) اگر تبدیل خطی زیر به شکل:
{\setlength\arraycolsep{0.1em}
	\begin{eqnarray*}
		f:{\Bbb R}^2\ \ &\longrightarrow&\ \ {\Bbb R}^2\\
		(v_1,v_2)&\longmapsto&(\frac{v_1+v_2}{2},\frac{v_1+v_2}{2})
\end{eqnarray*}}
آنگاه در مورد 
$f(f(v_1,v_2))$
چه می توان گفت؟(مشخص کنید خطی هست یا نه و درصورت خطی بودن ماتریس استاندارد آن را بیابید.)
\end{document}
