
\documentclass{article}



\usepackage{graphicx,comment,framed}
\usepackage{roundbox}
\usepackage{fancybox}
\usepackage{tikz}
\usepackage{color}
\usepackage[hidelinks]{hyperref}
\usepackage{framed}
\usepackage{amsthm,amssymb,amsmath}
%\usepackage[colorlinks,linkcolor=blue,citecolor=blue]{hyperref}
\definecolor{shadecolor}{cmyk}{0,0,0,0}
\usepackage{listings}
\usepackage{xepersian}
\usepackage[noend]{algpseudocode}


%--------------------- page settings ----------------------

\settextfont[Scale=1.1]{XB Niloofar}
\setdigitfont[Scale=1.1]{XB Niloofar}
\defpersianfont\sayeh[Scale=1.1]{XB Niloofar}
\addtolength{\textheight}{3.2cm}
\addtolength{\topmargin}{-22mm}
\addtolength{\textwidth}{3cm}
\addtolength{\oddsidemargin}{-1.5cm}


%------------------------ Environments ------------------------------------

\newtheorem{قضیه}{قضیه}
\newtheorem{لم}{لم}
\newtheorem{مشاهده}{مشاهده}
\newtheorem{تعریف}{تعریف}


%-------------------------- Notations ------------------------------------

\newcommand{\IR}{\ensuremath{\mathbb{R}}} 
\newcommand{\IZ}{\ensuremath{\mathbb{Z}}} 
\newcommand{\IN}{\ensuremath{\mathbb{N}}} 
\newcommand{\IS}{\ensuremath{\mathbb{S}}} 
\newcommand{\IC}{\ensuremath{\mathbb{C}}} 
\newcommand{\IB}{\ensuremath{\mathbb{B}}} 

\newcommand{\bR}{\mathbb{R}}
\newcommand{\cB}{\mathcal{B}}
\newcommand{\cO}{\mathcal{O}}
\newcommand{\cG}{\mathcal{G}}
\newcommand{\rM}{\mathrm{M}}
\newcommand{\rC}{\mathrm{C}}
\newcommand{\rV}{\mathrm{V}}

\newcommand{\lee}{\leqslant}
\newcommand{\gee}{\geqslant}
\newcommand{\ceil}[1]{{\left\lceil{#1}\right\rceil}}
\newcommand{\floor}[1]{{\left\lfloor{#1}\right\rfloor}}
\newcommand{\prob}[1]{{\mbox{\tt Pr}[#1]}}
\newcommand{\set}[1]{{\{ #1 \}}}
\newcommand{\seq}[1]{{\left< #1 \right>}}
\newcommand{\provided}{\,|\,}
\newcommand{\poly}{\mbox{\rm poly}}
\newcommand{\polylog}{\mbox{\rm \scriptsize polylog}\,}
\newcommand{\comb}[2] {\left(\!\!\begin{array}{c}{#1}\\{#2}\end{array}\!\!\right)}




\newcounter{probcnt}
\newcommand{\مسئله}[1]{\stepcounter{probcnt}\bigskip\bigskip{
 	\large \bf مسئله‌ی \arabic{probcnt}$\mbox{\bf{.}}$ \ #1} \bigskip}

\newcommand{\fqed}[1]{\leavevmode\unskip\nobreak\quad\hspace*{\fill}{\ensuremath{#1}}}

\newenvironment{اثبات}
	{\begin{trivlist}\item[\hskip\labelsep{\em اثبات.}]}
	{\fqed{\square}\end{trivlist}}

\newenvironment{حل}
	{\begin{trivlist}\item[\hskip\labelsep{\bf حل.}]}
	{\fqed{\blacktriangleright}\end{trivlist}}

\ifdefined\hidesols
	\newsavebox{\trashcan} % uncomment the following line to hide solutions
	\renewenvironment{حل}{\begin{latin}\begin{lrbox}{\trashcan}}{\end{lrbox}\end{latin}}
\fi


%------------------------- Header -----------------------------

%------------------------- Header -----------------------------

\newcommand{\سربرگ}[3]{
	\parindent=0em
	\begin{shaded}
		
		\rightline{ 
			\makebox[6em][c]{
				\includegraphics[height=1.6cm]{aut.png}
		}}
		\vspace{-.2em}
		{\scriptsize\bf دانشکده‌ی مهندسی کامپیوتر}
		\hfill {\small
			مدرس: دکتر امیرمزلقانی  \ 
		}\\[-5em]
		\leftline{\hfill\Large\bf 
			کاربرد های جبر خطی
		}\\[.7em]
		\leftline{\hfill\bf 
			نیم‌سال دوم ۹۶-۹۷
		}\\[1.7em]
		\hrule height .12em
		
		\normalsize
		\vspace{1mm} #1
		\hfill \small  #3
		\vspace{1mm} 
		\hrule height .1em
		
		\vspace{-0.5em} 
		\hfill {\sayeh\large #2} \hfill
	\end{shaded}
	\begin{large}

توجه:
\end{large}
\\
\\
\begin{itemize}
\item
این تمرین از مباحث مربوط به فصل 3 و 4 طراحی شده است که شامل 8 سوال اجباری و 2 سوال امتیازی است که نمره سوال های امتیازی فقط به نمرات تمرین شما کمک می کند.

\item 
اگه سوالی داشتین از طریق
\begin{latin}
\href{mailto:aut.la2018@gmail.com}{\[aut.la2018@gmail.com\]}
\end{latin}
 حتما بپرسید.

\item پاسخ های تمرین را در قالب یک فایل به صورت الگوی زیر آپلود کنید.
\begin{latin}
9531000\_Jonatan\_Vannieuwenhoven\_HW3.pdf
\end{latin}
\item \color{red}
مهلت تحویل جمعه 21 اردیبهشت 1397 ساعت 23:54:59
\end{itemize}

}




%\settextfont{Yas}
\begin{document}
\سربرگ{پاسخ تمرین های سری اول}{}{}
\clearpage


\حل{مسئله{\large\bf 1}} 
\\
الف) 

\begin{gather*}	
\begin{bmatrix}
1&2&3&4&1\\	
0&1&2&3&1\\
0&0&0&1&1\\
0&1&0&1&1
\end{bmatrix}\xrightarrow[R_1-R_2\to R_1]{R_4-R_1\to R_4}
\begin{bmatrix}
1&0&-1&-2&-1\\
0&1&2&3&1\\
0&0&0&1&1\\
0&0&-2&-2&10
\end{bmatrix}\xrightarrow{R_3\leftrightarrow R_4}
\begin{bmatrix}
1&0&-1&-2&-1\\
0&1&2&3&1\\
0&0&-2&-2&0\\
0&0&0&1&1
\end{bmatrix}\\
\xrightarrow[R_3+R_2\to R_2]{R_1-\frac{1}{2}R_3\to R_1}
 \begin{bmatrix}
 1&0&0&-1&-1\\
 0&1&0&1&1\\
 0&0&-2&-2&0\\
 0&0&0&1&1
 \end{bmatrix}\xrightarrow[2R_4+R_3\to R_3]{R_2-R_4\to R_2\& R_1-R_4\to R_1}
\begin{bmatrix}
\underline{1}&0&0&0&0\\
0&\underline{1}&0&0&0\\
0&0&\underline{-2}&0&2\\
0&0&0&\underline{1}&1
\end{bmatrix}\xrightarrow{-\frac{1}{2}R_3}
\begin{bmatrix}
\underline{1}&0&0&0&0\\
0&\underline{1}&0&0&0\\
0&0&\underline{1}&0&-1\\
0&0&0&\underline{1}&1
\end{bmatrix}
\end{gather*}


همانطور که در بالا مشاهده می کنید این دستگاه معادلات 4 نقطه محوری و 4 ستون محوری دارد و تنها یک جواب دارد که برابر است با:
$x_1=0,x_2=0,x_3=-1,x_4=1$\\
\\
ب)

\begin{gather*}
\begin{bmatrix}
1&-7&0&6&5\\
0&0&1&-2&-3&\\
-1&7&-4&2&7
\end{bmatrix}\xrightarrow{R_1+R_3\to R_1}
\begin{bmatrix}
1&-7&0&6&5\\
0&0&1&-2&-3\\
0&0&-4&8&12
\end{bmatrix}\xrightarrow{R_4+4R_2\to R_4}
\begin{bmatrix}
\underline{1}&-7&0&6&5\\
0&0&\underline{1}&-2&-3\\
0&0&0&0&0
\end{bmatrix}
\end{gather*}	

حال با توجه به نقاط محوری ستون های 1و3 نیز ستون های محوری هستند و معادله بی نهایت جواب دارد و جواب های آن به شکل زیر است:
\begin{equation*}
\left\{
\begin{array}{rl}
&x_1=5+7x_2-6x_4\\
&x_3=-3+2x_4\\
&x_2,x_4 free
\end{array} \right.
\end{equation*}\\ \\
ج) 

\begin{gather*}
\begin{bmatrix}
3&-4&2&0\\
9&12&-6&0\\
-6&8&-4&0
\end{bmatrix}\xrightarrow[R_3+2R_1\to R_2]{R_3+2R_1\to R_3}
\begin{bmatrix}
\underline{3}&-4&2&0\\
0&0&0&0\\
0&0&0&0
\end{bmatrix}\xrightarrow{-\frac{1}{3}R_1}
\begin{bmatrix}
\underline{1}&-\frac{4}{3}&\frac{2}{3}&0\\
0&0&0&0\\
0&0&0&0
\end{bmatrix}
\end{gather*}	
این دستگاه معادلات فقط یک نقطه ویک ستون محوری دارد که ستون اول است و بی شمار جواب دارد که جواب های آن به شکل زیر هستند.
\begin{equation*}
\left\{
\begin{array}{rl}
&x_1=\frac{4x_2-2x_3}{3}\\
&x_2,x_3 free
\end{array} \right.
\end{equation*}
\حل{مسئله{\large\bf 2}}\\
الف)برای حل این دستگاه ابتدا ماتریس افزوده آن را تشکیل می دهیم:
\begin{gather*}
\begin{bmatrix}
\lambda&3&0\\
2&4&0
\end{bmatrix}\xrightarrow{R_1\leftrightarrow R_2}
\begin{bmatrix}
2&4&0\\
\lambda&3&0
\end{bmatrix}\xrightarrow{R_2-\frac{\lambda}{2}R_1\to R_2}
\begin{bmatrix}
2&4&0\\
0&3-2\lambda&0
\end{bmatrix}
\xrightarrow{R_1-\frac{4}{3-2\lambda}\to R_1}
\begin{bmatrix}
\underline{2}&0&0\\
0&\underline{3-2\lambda}&0
\end{bmatrix}
\end{gather*}
با توجه به اینکه این دستگاه معالات 2 نقطه وستون محوری دارد و یک دستگاه همگن است برای اینکه جوابی جز جوابی بدیهی صفر داشته باشد باید یکی از سطر های آن صفر باشد در نتیجه:
$\lambda=\frac{2}{3}$

ب) برای حل ابتدا باید ماتریس افزوده دستگاه معادلات را تشکیل بدهیم:
\begin{gather*}
\begin{bmatrix}
\lambda&1&1&0\\
1&\lambda&1&0\\
1&1&1&0
\end{bmatrix}\xrightarrow{R_1\leftrightarrow R_3}
\begin{bmatrix}
1&1&1&0\\
1&\lambda&1&0\\
\lambda&1&1&0
\end{bmatrix}\xrightarrow[R_3-\lambda R_1\to R_3]{R_2-R_1\to R_2}
\begin{bmatrix}
1&1&1&0\\
0&\lambda-1&0&0\\
0&1-\lambda&1-\lambda&0
\end{bmatrix}\\
\xrightarrow[R_1-R_2/\lambda-1\to R_1]{R_3-R_2\to R_3}
\begin{bmatrix}
1&0&1&0\\
0&\lambda-1&0&0\\
0&0&1-\lambda&0
\end{bmatrix}\xrightarrow{R_1-R_2/1-\lambda\to R_1}
\begin{bmatrix}
\underline{1}&0&0&0\\
0&\underline{\lambda-1}&0&0\\
0&0&\underline{1-\lambda}&0
\end{bmatrix}
\end{gather*}
حال یک دستگاه معادلات همگن داریم و می دانیم که صفر جواب بدیهی هر دستگاه معادله همگن است حال در دستگاه معادلات بالا سه نقطه محوری داریم و اگر هر کدام از این نقاط صفر شوند آنگاه دستگاه بی شمار جواب خواهد شد پس باید هیچکدام از نقاط محوری صفر نشوند پس باید :
\begin{equation*}
\left\{
\begin{array}{rl}
&\lambda-1\neq0\\
&1-\lambda\neq0
\end{array} \right.
\end{equation*}
در نتیجه به ازای تمامی مقادیر 
$\lambda$
جز صفر دستگاه فقط یک جواب دارد.
\حل{مسئله{\large\bf 3}}\\
لازم است قبل از پرداختن به حل این سوال دانشجویان با نوع دیگری از نمایش بردار ها نیز آشنا شوند فرض کنید 
$v=\begin{bmatrix}
v_1\\
v_2\\
\vdots\\
v_n
\end{bmatrix}$
باشد آنگاه 
$v$
را اینگونه هم نشان می دهند :
$v=(v_1,v_2,\ldots,v_n)$
که ما در حل این سوال از این نوع نمایش استفاده خواهیم کرد.
\begin{enumerate}

\item
 برای این قسمت مثال نقضی در 
$\mathbb{R}^3$
می زنیم،فرض کنید 
$A=\{(1,0,0),(0,1,0),(0,0,1),(0,2,2)\}$	
که $َA$ یک مجموعه وابسته خطی است زیرا :
$(0,2,2)=2(0,1,0)+2(0,0,1)$
اما 
$(1,0,0)$
را نمی توان به صورت ترکیب خطی بقیه بردار ها نوشت.
 \item 
 برای این قسمت مثال نقضی در 
 $\mathbb{R}^2$
 می زنیم فرض کنید:
 $$S_1=\{(1,0),(0,1),(2,2)\}\qquad S_2=\{(1,0),(0,1),(3,3)\}$$
 و واضح است 
 $\mathbb{R}^2=span(s_1)=span(s_2)=\mathbb{R}^2$
 اما 
 $S_1\neq S_2$
 .
 \item 
 برای این گزاره مثال نقضی در 
 $\mathbb{R}^2$
 می زنیم،به وضوح 
 $A=\{(1,0)\}$
 یک مجموعه مستقل خطی از بردار ها است ،اما 
 $span(A)\neq \mathbb{R}^2$
 بلکه 
 $span(A)$
 تمامی نقاط واقع در محور $y$ ها را شامل می شود و این تناقض است.
 \item 
 مثل نقض برای این گزاره  در
 $\mathbb{R}^3$
 همان مجموعه 
 $S_1$
 در قسمت 1 است که مستقل خطی نیست اما 
 $span(s_1)=\mathbb{R}^3$
 .
 \item
 چون 
 $w$
 ترکیب خطی از بقیه بردار ها است پس می توان نوشت:
 $$\alpha_1v_1+\alpha_2v_2+\ldots+\alpha_nv_n=w $$
 که 
$\exists \alpha_i \ \ \alpha_i\neq0$
پس می توان نتیجه گرفت:
$$\alpha_1v_1+\alpha_2v_2+\ldots+\alpha_nv_n-w=0$$ 
پس ما توانستیم ترکیب خطی از 
$\{w,v_1,v_2,\ldots v_n\}$
بیابیم که حاصل ترکیب خطی آن ها صفر مبشود اما همه ضرایب آن ها صفر نیست و این متناقض با استقلال خطی این بردار ها است . 
\item 
به برهان خلف فرض کنیم که 
$S\cup \{v\}$
مستقل خطی نباشد و 
$\forall i \ \ w_i\in S$
آنگاه:
$$\exists \alpha_i \ \ \alpha_i\neq0 \ \ \alpha_1w_1+\alpha_2w_2+\ldots+\alpha_nw_n+\beta v=0$$
حالا دو حالت پیش می آید اگر 
$\beta=0$
باشد آنگاه 
$$\exists \alpha_i \ \ \alpha_i\neq0 \ \ \alpha_1w_1+\alpha_2w_2+\ldots+\alpha_nw_n=0$$
که این با مستقل خطی بودن $ُS$ در تناقض است زیرا توانستیم ترکیب خطی از بردار های $S$ را بیابیم که مساوی صفر باشد اما ضرایب آن ها صفر نباشد.پس در این حالت فرض خلف باطل و حکم درست است.

در حالت دیگر فرض کنیم که 
$\beta\neq0$
آنگاه می نویسیم 
$$ \alpha_1w_1+\alpha_2w_2+\ldots+\alpha_nw_n=-\beta v$$
در نتیجه :
$$v=\frac{\alpha_1w_1+\alpha_2w_2+\ldots+\alpha_nw_n}{-\beta}$$
پس از اینجا نتیجه می شود 
$v\in span(s)$
آنگاه 
$v\notin(\mathbb{R}^n-span(S))$
خواهد بود که با فرض سوال در تناقض است در نتیجه فرض خلف باطل و حکم درست است.
\end{enumerate}
\حل{مسئله{\large\bf 4}}\\
الف)می دانیم در هر ستون ماتریس سودوکو تمامی اعداد 1 تا 9 قرار دارند پس :
$$َA=S\begin{bmatrix}
1\\
1\\
\vdots\\
1\end{bmatrix}_{9\times1}=\underbrace{1\times \begin{bmatrix}
		v_{11}\\
		v_{21}\\
		\vdots\\
	v_{91}\end{bmatrix}+\ldots1\times\begin{bmatrix}
	v_{19}\\
	v_{29}\\
	\vdots\\
	v_{99}\end{bmatrix} }_\text{9بار }$$
ار طرف دیگر می دانیم :
$$\forall i,j\in \{1,2\ldots9\} \ \ \sum_{i=1}^{9}v_{ij}=\frac{9(9+1)}{2}=45$$
پس نتیجه می شود :
$$A=\begin{bmatrix}
45\\
45\\
\vdots\\
45
\end{bmatrix}_{9\times1}$$\\
\\
ب) یادآوری می شود اعمال سطری شامل سه عمل می شوند و باید بررسی کنیم کدام یک از این اعمال سطری شکل سودوکو را حفظ می کنند .
\begin{enumerate}
\item {\bf ضرب سطر در عددی حقیقی:}\\
اگر سطری در عدی ضرب کنیم انگاه مقادیر آن دیگر اعداد 1 تا 9 نخواهند بود و این متناقض با سودوکو بودن است
\item {\bf اضافه کردن ضریبی از یک سطر به سطر دیگر:}

می دانیم مجموع ارقام ماتریس سودوکو 
$9\times45$
است که اگر ماتریس دیگری سودوکو باشد باید این شرط را حفظ کند حال اگر ضریبی از یک سطر را به سطر دیگر اضافه کنیم آنگاه مجموع اعداد سودوکو به شکل 
$9\times45+k(45)$
خواهد بود که
$k$
همان ضریب مورد نظر است پس مجموع اعداد ماتریس جدید بیشتر یا کمتر از
$9\times45$
خواهد بود و این متناقض با سودوکو بودن است.
\item{\bf جابه جایی دو سطر:}\\
اما عملی که نیاز به بررسی بیشتر دارد عمل جابه جایی است اگر سطری را با سطر دیگر جا به جا کنیم آنگاه شرط اینکه در هر سطر و در هر ستون اعداد 1 تا9 باشند همچنان حفظ می شود ولی شرطی که برای سودوکو بودن نقض می شود قرار گرفتن  تمام اعداد 1 تا 9 در مربع های 
$3\times3$
است و حالا اثبات می کنیم که این موضوع برقرار نمی باشد به برهان خلف فرض کنیم که با جابه جایی هر دو سطری قرار گرفتن اعداد 1 تا 9 در مربع های 
$3\times3$
همچنان برقرار بماند در آن صورت مثلا 3 درایه اول سطر اول با 3 درایه تمامی سطر های دیگر یکی است فقط ترتیب قرار گرفتن این 3 عدد متفاوت است چون اعداد قرار گرفته در این قسمت ها بیشتر از 3 نیس بنابر اصل لانه کبوتری وجود دارد درایه ای در یکی از این سه ستون که با هم برابرند پس شرط سودوکو بودن نقض می شود و فرض خلف باطل و عمل جابه جایی سودوکو بودن را حفظ نمی کند.   

\end{enumerate}
\حل{مسئله{\large\bf 5}}
\begin{enumerate}
\item
به برهان خلف فرض کنیم که ستون های ماتریس $َA$ مستقل خطی نباشند آنگاه می توان نوشت:
$$\exists x_i\neq0 x_1v_1+x_2v_2+\ldots x_nv_n=0$$
که 
$v_i$
ها همان ستون های ماتریس $َA$ و 
$x_i$
ها درایه های $x$ هستند ،اگر شرایط بالا برقرار باشد ما برای 
$b=0$
جوابی غیر از جواب بدیهی  صفر یافته ایم و این متناقض با فرض اینکه معادله حداکثر یک جواب دارد هست پس فرض خلف باطل و حکم درست است.

\item
ابتدا فرض کنیم 
$n>m$
 باتوجه به اینکه هر نقطه محوری در یک سطر و ستون خاص قرار دارد در نتیجه اگر 
 $n$
 ستون محوری داشته باشیم آنگاه در واقع $n$ نقطه محوری داریم پس $n$ سطر محوری هم خواهیم داست در حالی که 
 $m<n$
 و این ممکن نیست 
 پس 
 $m>n$
 ،حال اگر $n$  ستون محوری داشته باشیم یعنی تمامی ستون ها محوری هستند ،پس $n$ سطر نیز محوری است و بقیه سطر ها صفر هستند در نتیجه اگر این سطر ها مقدار ناصفر داشته باشند معدله جواب ندارد و اگر مقدارشان صفر باشد آنگاه معادله دقیقا یک جواب دارد .	
\end{enumerate}
\حل{مسئله{\large\bf 6}}
در هر کدام تبدیلات زیر ابتدا خطی بودن را بررسی می کنیم،برای این موضوع باید دو شرط:
\begin{enumerate}
	\item $T(0)=0$
	\item $T(c{\bf u}+d{\bf v})=cT({\bf u})+dT({\bf v})$
\end{enumerate}

	برقرار باشند ،و در صورت خطی بودن بایستی ماتریس استاندارد تبدیل خطی را بیابیم.
	برای یا فتن ماتریس استاندارد تبدیل خطی، ماتریس
	$$A= [T({\bf e_1}) \ldots T({\bf e_n})]$$
	را می یابیم که 
	$e_j$ 
	،
	$j$امین ستون ماتریس همانی است.
	 
الف) برای این قسمت خطی بودن را بررسی می کنیم:
ّ$$f(0,0)=(0^2,2(0))=(0,0)$$
\begin{eqnarray*}
 f(c(x,y)+d(u,v))&=&f(cx+du,cy+dv)=((cx+du)^2,2(cy+dv))\\
	&=&c^2x^2+2cxdu+d^2u^2,2cy+2dv\\
	\neq cf(x,y)+df(u,v)&=&c(x^2,2y)+d(u^2,2v)\\
	&=&(cx^2+du^2,2cy+2dv) 
	 \end{eqnarray*}
 با توجه به اینکه تساوی برقرار نشد پس تبدیل خطی نیست.
 \\
ب) ابتدا بایستی خطی بودن رابررسی کنیم :
$$f(0,0)=(2(0)+0,-(0))=(0,0)$$
\begin{eqnarray*}
	f(c(x,y)+d(u,v))&=&f(cx+du,cy+dv)=(2(cx+du)+cy+dv,-(cy+dv))\\
	&=&(2cx+2du+cy+dv,-cy-dv)\\
	= cf(x,y)+df(u,v)&=&c(2x+y,-y)+d(2u+v,-v)\\
	&=&(2cx+2y+2du+2dv,-cy-dv) 
\end{eqnarray*}
پس تساوی برقرار است واز آنجاییکه تبدیل خطی است  باید ماتریس استاندارد آن را بیابیم.
ابتدا 
$f(e_1)$
را می یابیم:
$$f(e_1)=f(1,0)= \begin{bmatrix}
1\\
0
\end{bmatrix}$$
و سپس 
$f(e_2)$
را :
$$f(e_2)=f(0,1)=\begin{bmatrix}
1\\
-1
\end{bmatrix}$$
حال ماتریس استاندارد را تشکیل می دهیم: 
$$A=[f(e_1),f(e_2)]=\begin{bmatrix}
1&1\\
0&-1
\end{bmatrix}$$
ج)این قسمت ابتدا تبدیلی معرفی شده سپس آن تبدیل با خودش ترکیب شده و تبدیل جدیدی را به وجود آورده است لازم است به این نکته توجه شود که اگر تبدیلی خطی باشد ترکیب آن با تبدیل خطی دیگری نیز همچنان خطی است. 

پس برای اثبات خطی بودن 
$ّّf(f(v_1,v_2))$
کافی است خطی بودن 
$f$
را اثبات کنیم.که این قسمت نیز به سادگی همانند مثال قبل اثبات می شود اما برای یافتن ماتریس استاندارد تبدیل ،باید ضابطه آن را بیابیم: 
$$f(v_1,v_2)=(\frac{v_1+v_2}{2},\frac{v_1+v_2}{2})\to f(f(v_1,v_2))=f(\frac{v_1+v_2}{2},\frac{v_1+v_2}{2})=(v_1+v_2,v_1+v_2)$$
در نتیجه:
$$f(e_1)=f(1,0)=\begin{bmatrix}
1\\
1
\end{bmatrix}
$$ $$f(e_2)=f(0,1)=\begin{bmatrix}
	1\\
	1
\end{bmatrix}$$
پس ماتریس استاندار این تبدیل خطی برابراست با:
$$A=[f(e_1),f(e_2)]=\begin{bmatrix}
1&1\\
1&1
\end{bmatrix}$$
\end{document}