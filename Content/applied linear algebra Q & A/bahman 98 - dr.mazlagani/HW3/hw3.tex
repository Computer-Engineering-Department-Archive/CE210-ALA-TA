
\documentclass{article}
\usepackage{enumitem}


\usepackage{graphicx,comment,framed}
\usepackage{roundbox}
\usepackage{fancybox}
\usepackage{tikz}
\usepackage{color}
\usepackage[hidelinks]{hyperref}
\usepackage{framed}
\usepackage{amsthm,amssymb,amsmath}
%\usepackage[colorlinks,linkcolor=blue,citecolor=blue]{hyperref}
\definecolor{shadecolor}{cmyk}{0,0,0,0}
\usepackage{listings}
\usepackage{xepersian}
\usepackage[noend]{algpseudocode}


%--------------------- page settings ----------------------

\settextfont[Scale=1.1]{XB Niloofar}
\setdigitfont[Scale=1.1]{XB Niloofar}
\defpersianfont\sayeh[Scale=1.1]{XB Niloofar}
\addtolength{\textheight}{3.2cm}
\addtolength{\topmargin}{-22mm}
\addtolength{\textwidth}{3cm}
\addtolength{\oddsidemargin}{-1.5cm}


%------------------------ Environments ------------------------------------

\newtheorem{قضیه}{قضیه}
\newtheorem{لم}{لم}
\newtheorem{مشاهده}{مشاهده}
\newtheorem{تعریف}{تعریف}


%-------------------------- Notations ------------------------------------

\newcommand{\IR}{\ensuremath{\mathbb{R}}} 
\newcommand{\IZ}{\ensuremath{\mathbb{Z}}} 
\newcommand{\IN}{\ensuremath{\mathbb{N}}} 
\newcommand{\IS}{\ensuremath{\mathbb{S}}} 
\newcommand{\IC}{\ensuremath{\mathbb{C}}} 
\newcommand{\IB}{\ensuremath{\mathbb{B}}} 

\newcommand{\bR}{\mathbb{R}}
\newcommand{\cB}{\mathcal{B}}
\newcommand{\cO}{\mathcal{O}}
\newcommand{\cG}{\mathcal{G}}
\newcommand{\rM}{\mathrm{M}}
\newcommand{\rC}{\mathrm{C}}
\newcommand{\rV}{\mathrm{V}}

\newcommand{\lee}{\leqslant}
\newcommand{\gee}{\geqslant}
\newcommand{\ceil}[1]{{\left\lceil{#1}\right\rceil}}
\newcommand{\floor}[1]{{\left\lfloor{#1}\right\rfloor}}
\newcommand{\prob}[1]{{\mbox{\tt Pr}[#1]}}
\newcommand{\set}[1]{{\{ #1 \}}}
\newcommand{\seq}[1]{{\left< #1 \right>}}
\newcommand{\provided}{\,|\,}
\newcommand{\poly}{\mbox{\rm poly}}
\newcommand{\polylog}{\mbox{\rm \scriptsize polylog}\,}
\newcommand{\comb}[2] {\left(\!\!\begin{array}{c}{#1}\\{#2}\end{array}\!\!\right)}




\newcounter{probcnt}
\newcommand{\مسئله}[1]{\stepcounter{probcnt}\bigskip\bigskip{
 	\large \bf مسئله‌ی \arabic{probcnt}$\mbox{\bf{.}}$ \ #1} \bigskip}

\newcommand{\fqed}[1]{\leavevmode\unskip\nobreak\quad\hspace*{\fill}{\ensuremath{#1}}}

\newenvironment{اثبات}
	{\begin{trivlist}\item[\hskip\labelsep{\em اثبات.}]}
	{\fqed{\square}\end{trivlist}}

\newenvironment{حل}
	{\begin{trivlist}\item[\hskip\labelsep{\bf حل.}]}
	{\fqed{\blacktriangleright}\end{trivlist}}

\ifdefined\hidesols
	\newsavebox{\trashcan} % uncomment the following line to hide solutions
	\renewenvironment{حل}{\begin{latin}\begin{lrbox}{\trashcan}}{\end{lrbox}\end{latin}}
\fi


%------------------------- Header -----------------------------

%------------------------- Header -----------------------------

\newcommand{\سربرگ}[3]{
	\parindent=0em
	\begin{shaded}
		
		\rightline{ 
			\makebox[6em][c]{
				\includegraphics[height=1.6cm]{aut.png}
		}}
		\vspace{-.2em}
		{\scriptsize\bf دانشکده‌ی مهندسی کامپیوتر}
		\hfill {\small
			مدرس: دکتر امیرمزلقانی  \ 
		}\\[-5em]
		\leftline{\hfill\Large\bf 
			کاربرد های جبر خطی
		}\\[.7em]
		\leftline{\hfill\bf 
			نیم‌سال دوم ۹۶-۹۷
		}\\[1.7em]
		\hrule height .12em
		
		\normalsize
		\vspace{1mm} #1
		\hfill \small  #3
		\vspace{1mm} 
		\hrule height .1em
		
		\vspace{-0.5em} 
		\hfill {\sayeh\large #2} \hfill
	\end{shaded}
	\begin{large}

توجه:
\end{large}
\\
\\
\begin{itemize}
\item
این تمرین از مباحث مربوط به فصل 3 و 4 طراحی شده است که شامل 8 سوال اجباری و 2 سوال امتیازی است که نمره سوال های امتیازی فقط به نمرات تمرین شما کمک می کند.

\item 
اگه سوالی داشتین از طریق
\begin{latin}
\href{mailto:aut.la2018@gmail.com}{\[aut.la2018@gmail.com\]}
\end{latin}
 حتما بپرسید.

\item پاسخ های تمرین را در قالب یک فایل به صورت الگوی زیر آپلود کنید.
\begin{latin}
9531000\_Jonatan\_Vannieuwenhoven\_HW3.pdf
\end{latin}
\item \color{red}
مهلت تحویل جمعه 21 اردیبهشت 1397 ساعت 23:54:59
\end{itemize}

}




%\settextfont{Yas}
\begin{document}
\سربرگ{تمرین فصل3و4}{}{}
\clearpage
\مسئله{}درستی یا نادرستی عبارت های زیر را مشخص کنید و دلیل آن را بیان کنید.
\\
(
از بین 21 مورد  15 مورد را به اختیار انتخاب کرده و مشخص کنید.)

\begin{enumerate}[label=\alph*)]
	\item 
	دترمینان 
	$S^{-1}AS$
	 برابر
	 $det(A)$ 	
	 می باشد.
	\item 
	اگر A یک ماتریس مربعی $4 \times 4 $ باشد، دترمینان 4A برابر 
	$4det(A)$
	\item
	ماتریس های AB و BA دترمینان برابری دارند
	\item
	دترمینان AB-BA برابر صفر است
	\item
	اگر A معکوس پذیر نباشد، AB نیز معکوس پذیر نیست
	\item 
	اگر 
	$AA^T = I$
	 باشد آنگاه
	 $ det(A) = \pm 1$
	 \item 
	 دترمینان هر ماتریس پادمتقارن
	  $(A^T = -A)$
	   برابر صفر است
	   \item 
	    اگر A ماتریسی
	     $n\times n $  
	      با درایه هایی از مجموعه اعداد صحیح باشد به گونه ای که
	     $det(A) = 1$
	     آنگاه درایه های وارون ماتریس A نیز عضو مجموعه اعداد صحیح می باشند.
	     \item 
	     برای هر ماتریس A که 
	     $2\times 2 $ 
	       باشد
	       ، طبق محاسبات زیر داریم که دترمینان وارون آن برابر ۱ است:
	     
	     $$det A^{-1} = det \frac {1}{ad-bc} 
	    \begin{bmatrix}
	      d & -b \\
	      -c &  a\\
	      \end{bmatrix} = \frac{ad-bc}{ad-bc} = 1$$
	      \item
	      
	      برای هر ماتریس P که 
	      $P = A (A^TA)^{-1}A^T$ 
	      داریم:
	      $$ |P|  =|A| \frac{1}{|A^T||A|}|A^T| = 1 $$
	      \item
	      اگر دترمینان A برابر صفر باشد، حداقل یکی از cofactor ها باید صفر باشد
	      \item 
	      دترمینان ماتریسی که همه درایه های آن عضو مجموعه
	     $\{-1,0,1\}$
	         باشد، عضو همین مجموعه است.
	      \item 
	      $Adj A^T = (Adj A)^T$
	      \item 
	      اگر A وارون پذیر باشد،
	     $A$ $Adj$ 
	        نیز وارون پذیر است و
	      $ Adj A^{-1}  = (Adj A)^{-1}$
	      \item
	      اگر A قطری باشد
	       $A$ $Adj$ 
	       نیز قطری است
	      \item
	     اگر A یک ماتریس مربعی ۵ در ۵ باشد،
	     $$Adj(2A) = 32 Adj (A)$$
	     
	     \item 
	     اگر
	     $A = (a , r_2 , r_3, r_4)$
	     و
	     $B = (b , r_2, r_3, r_4)$
	     دو ماتریس 
	     $4\times 4 $ 
	     باشند، که
	     $a , b , r_2 , r_3, r_4$
	     بردار های ستونی در 
	     $\mathbb{R}^4$
	     باشند، اگر دترمینان A  برابر ۴ و دترمینان B  برابر ۱ باشد، دترمینان A+B برابر ۴۰ و اگر ماتریس C را برابر 
	     $(r_4, r_3,r_2,a+b) $
	     تعریف کنیم، دترمینان آن برابر ۵ است.
	     \item
	     با توجه به اینکه دترمینان همه ماتریس های پاسکال برابر ۱ است. اگر یک واحد از درایه n و n ام کم کنیم، دترمینان آن صفر می شود.
	     (برای مثال ماتریس پاسکال
	     $4\times 4 $ 
	     به صورت 
	    $\begin{bmatrix}
	    1 & 1 & 1 & 1 \\
	    1 & 2 & 3 & 4 \\
	    1 & 2 &3 & 4\\
	     1 & 4 &10 & 20\\
	    \end{bmatrix}$
	    می باشد.)
	    \item 
	    ماتریس L یک ماتریس پایین مثلثی ۳ در ۳ است، معکوس هر ماتریس پایین مثلثی یک ماتریس پایین مثلثی است، برای همین cofactor های 
	    $C_{21}, C_{31}, C_{32}$
	   برای ماتریس L برابر صفر هستند.
	   \item
	   ماتریس S یک ماتریس متقارن ۳ در ۳ است. معکوس هر ماتریس متقارن یک ماتریس متقارت است برای همین رابطه 
	   $ C_{12}=C_{21} C_{13}=C_{31}, C_{23}=C_{32}$
	   بین cofactor های ماتریس S برقرار است.
	   \item
	   مساحت مثلث ABC که در آن 
	   $A(x_1,y_1)$
	   ,
	   $B(x_2,y_2)$
	   ,
	   $C(x_3,y_3)$
	 را می توان از طریق رابطه 
	   $$Area(ABC) = \frac{1}{2}
	   \begin{bmatrix}
	   	x_1 &y_1& 1\\
	   	x_2 &y_2&1\\
	   	x_3 &y_3&1\\
	   \end{bmatrix}$$
   به دست آورد.
   
   

\مسئله{}
در هر مورد مشخص کنید آیا زیر مجموعه ی داده شده یک زیر فضا از فضای برداری مشخص شده می باشد یا خیر.
\begin{enumerate}
	\item 
	$\{(x,y)\in\mathbb{R}^2|x^2+y^2 \leq4\}$
	در فضای برداری 
	$\mathbb{R}^2$
	\item
	$\{(a,b,c)\in\mathbb{R}^3|a+b+2c=0  \}$
	در فضای برداری 
	$\mathbb{R}^3$.
	\item
	$\{A\in M_n(\mathbb{R})|A^2=A\}$
	در
	$M_n(\mathbb{R})$.
	(منظور از 
	$M_n(\mathbb{R})$
	مجموعه تمام ماتریس های 
	$n\times n$
	با درایه هایی از مجموعه اعداد حقیقی است.)
	 
	\item 
	$\{p(x)|2p(0)=p(1),p(x)\in\mathbb{P} [x]  \}$
	در فضای برداری 
	$\mathbb{P}[x]$
	(تمامی چند جمله های حداکثر از درجه 
	$n$
	با ضرایب حقیقی را با نماد  
	$\mathbb{P}_n[x]$
	نشان می دهیم و همچنین مجموعه تمام چند جمله ها با ضرایب حقیقی را با  
		$\mathbb{P}[x]$
		نشان می دهیم) 
		\item 
		$\{p(x)|p(x)=a+x^2,a\in\mathbb{R}\}$
		در فضای برداری 
		$\mathbb{P}_3[x]$.
	
	\end{enumerate}

\مسئله{}
اگر 
$\mathbb{P}[x],\mathbb{P}_n[x]$
طبق تعریف بالا فضا های برداری با ضرایب حقیقی باشند آنگاه :
\begin{enumerate}
\item 
نشان دهید  اگر 
$\{1,x,x^2,\cdots,x^{n-1}\}$
پایه ای برای 
$\mathbb{P}_n[x]$
باشد آنگاه:
$$\{1,(x-a),(x-a)^2,\cdots,(x-a)^{n-1}\},\qquad a\in \mathbb{R}$$
نیز پایه ای برای 
$\mathbb{P}_n[x]$	
است.
\item
مختصات 
$$f(x)=a_0+a_1x+\cdots+a_{n-1}x^{n-1}\in \mathbb{P}_n[x]$$
را نسبت به پایه 
$$\{1,(x-a),(x-a)^2,\cdots,(x-a)^{n-1}\},\qquad a\in \mathbb{R}$$
بیابید.
\item 
فرض کنید 
$a_1,a_2,\cdots,a_n\in \mathbb{R}$
و متمایز باشند.برای هر 
$i=1,2,\cdots,n$

$$f_i(x)=(x-a_1)\dots(x-a_{i-1})(x-a_{i+1})\dots(x-a_n)$$
را در نظر بگیرید،نشان دهید 
$\{f_1(x),f_2(x),\cdots,f_n(x)\}$
نیز پایه ای برای 
$\mathbb{P}_n[x]$
است.
\end{enumerate}
\مسئله{}
فرض کنید 
$W_1,W_2$
زیر فضا های فضای برداری 
$V$
باشند، تعریف می کنیم :
$$W_1+W_2=\{w_1+w_2|w_1\in W_1,w_2\in W_2\}$$.

\begin{enumerate}
	\item 
	نشان دهید :
	$$W_1+W_2+\cdots+W_n=span(\bigcup_{i=1}^{n}W_i)$$
	\item 
	نشان دهید 
	$W_1\cap W_2,W_1+W_2$
	زیر فضای 
	$V$
	هستند و همچنین نشان دهید:
	$$W_1\cap W_2 \subseteq W_1\cup W_2\subseteq W_1+W_2$$.
	\item 
	نشان دهید :
	$$dim(W_1+W_2)=dim(W_1)+dim(W_2)-dim(W_1\cap W_2)$$.
	\item 
	نتیجه گیری قسمت 
	$2$
	را با استفاده از دو خط که از مبدا مختصات 
	$xy$
	می گذرند توجیه کنید.
	\item 
	درستی یا نادرستی تساوی زیر را بررسی کنید در صورت درست بودن اثبات و در صورت نادرست بودن مثال نقض بزنید:
	$$W_3\cap(W_1+W_2)=(W_3\cap W_1)+(W_3\cap W_2)$$.
	\item 
	اگر 
	$W_1\cap W_2=\{0\}$
	باشد آنگاه به 
	$W_1+W_2$
	جمع مستقیم نیز می گویند و آن را با 
	$W_1\bigoplus W_2$
	نشان می دهند،ثابت کنید اگر 
	$V_1$
	زیر فضایی از فضای برداری 
	$V$
	باشد و اگر زیر فضای برداری یکتای 
	$V_2$
	موجود باشد که 
	$V=V_1\bigoplus V_2$
	آنگاه 
	$V_1=V$.
	
\end{enumerate}
\مسئله{}
فرض کنید 
$T:V\longrightarrow W$
نگاشت خطی باشد،
$Nul\ \ T=\{0\}$
اگر و فقط اگر 
$T$
هر زیر مجموعه مستقل خطی را به زیر مجموعه مستقل خطی نگاشت کند. علاوه بر ویژگی های بالا اگر 
$A$
ماتریس استاندارد تبدیل 
$T$
باشد که به ازای هر 
$b$
که 
$b$
مختصات برداری در 
$W$
است وجود داشته باشد 
$x$
ای که مختصات برداری در 
$V$
 باشد که
 $Ax=b$
 باشد آنگاه 
 $T$
 هر پایه
 $V$
  را به پایه ای در 
  $W$
   می نگارد.
   
   \مسئله {} فرض کنید 
   $T:V\longrightarrow V$
   تبدیل خطی رو فضای متناهی البعد 
   $V$
   باشد و 
   $T^2=0$.
   ثابت کنید 
   \\
$2rank(A)\leq dim(V)$
   (
   $A$
   ماتریس استاندارد تبدیل 
   $T$
   است.
   )
   
   \مسئله{}
   اگر 
   $A$
    یک ماتریس 
    $m\times n$
    باشد که 
    $rank A=r>0$.
 $U$
   شکل سطری پلکانی ماتریس 
   $A$
   است.نشان دهید یک ماتریس وارون پیذیر مانند 
   $E$
   وجود دارد که 
   $A=EU$.
   با استفاده از این موضوع 
   $A$
    را به صورت حاصل جمع 
    $r$
    ماتریس با رنک 
    $1$
    بنویسید. 

    
    \مسئله {} در هر یک از قسمت های زیر ابتدا مختصات بردار داده شده
    $(v)$ 
    را در هریک از پایه ها بیابید سپس ماتریس انتقال از یک پایه
    $(B)$
    به پایه
    $(C)$
     دیگر را محاسبه کنید.
    \begin{enumerate}
    	\item 
    	$$V=\mathbb{P}_3[x]\qquad v=p(x)=8+x+6x^2+9x^3$$
    	$$B=\{−2+3x+4x^2-x^3, 3x+5x^2+2x^3, -5x^2-5x^3,
    		4 + 4x + 4x^2\}$$
    		$$C=\{1 - x^3, 1 + x, x + x^2, x^2 + x^3\}$$
    	\item 
    	$$V=M_2(\mathbb{R})\qquad v=\begin{bmatrix}
    	-3&-2\\
    	-1&2
    	\end{bmatrix}
    	$$
    	$$B=\{
    	\begin{bmatrix}
    	1&0\\
    	-1&-2
    	\end{bmatrix}
    	\begin{bmatrix}
    	0&-1\\
    	3&0
    	\end{bmatrix}
    	\begin{bmatrix}
    	3&5\\
    	0&0
    	\end{bmatrix}
    	\begin{bmatrix}
    	-2&-4\\
    	0&0
    	 \end{bmatrix}
    	\}$$
    		$$C=\{
    	\begin{bmatrix}
    	1&1\\
    	1&1
    	\end{bmatrix}
    	\begin{bmatrix}
    	1&1\\
    	1&0
    	\end{bmatrix}
    	\begin{bmatrix}
    	1&1\\
    	0&0
    	\end{bmatrix}
    	\begin{bmatrix}
    	1&0\\
    	0&0
    	\end{bmatrix}
    	\}$$
    	 \item
    	 $$V=\mathbb{R}^3\qquad v=(1,7,7)$$
    	 $$B=\{(-7,4,4),(4,2,-1),(-7,5,0)\}$$
    	 $$C={(1,1,0),(0,1,1),(3,-1,-1)}$$
    \end{enumerate}
   \مسئله{سوال امتیازی}
   
   بازی دو نفره ی زیر را در نظر بگیرید:
   \begin{enumerate}
   	\item 
   	بازی با یک ماتریس ۱۰ در ۱۰ خالی بازی شروع می شود.
   	\item
   	بازیکن اول و دوم به ترتیب اعداد حقیقی دلخواهی در درایه های این ماتریس قرار می دهند 
   	\item
   	بعد از پر شدن ماتریس، بازیکن اول در صورتی برنده است که دترمینان ماتریس نهایی مخالف صفر باشد و بازیکن دوم در صورتی برنده است که دترمینان صفر شود.
   \end{enumerate}
   کدام یک از بازیکننان یک استراتژی ای برای پیروزی دارد؟ در واقع اگر شما در این بازی حق انتخاب اول یا دوم بودن را داشتید کدام را انتخاب می کردید و استراتژی شما برای پیروزی در این نوبت چیست؟
   
\end{enumerate}
   \مسئله{سوال امتیازی}{برای حل این سوال باید از لم زُرن استفاده کنید که لم زُرن یک لم پر کاربرد در زمینه نظریه مجموعه ها است.}
  
   نشان دهید هر فضای برداری غیر صفر یک پایه دارد؟

   
   
\end{document}


