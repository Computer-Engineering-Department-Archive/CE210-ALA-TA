
\documentclass{article}
\usepackage{enumitem}


\usepackage{graphicx,comment,framed}
\usepackage{roundbox}
\usepackage{fancybox}
\usepackage{tikz}
\usepackage{color}
\usepackage[hidelinks]{hyperref}
\usepackage{framed}
\usepackage{amsthm,amssymb,amsmath}
%\usepackage[colorlinks,linkcolor=blue,citecolor=blue]{hyperref}
\definecolor{shadecolor}{cmyk}{0,0,0,0}
\usepackage{listings}
\usepackage{xepersian}
\usepackage[noend]{algpseudocode}


%--------------------- page settings ----------------------

\settextfont[Scale=1.1]{XB Niloofar}
\setdigitfont[Scale=1.1]{XB Niloofar}
\defpersianfont\sayeh[Scale=1.1]{XB Niloofar}
\addtolength{\textheight}{3.2cm}
\addtolength{\topmargin}{-22mm}
\addtolength{\textwidth}{3cm}
\addtolength{\oddsidemargin}{-1.5cm}


%------------------------ Environments ------------------------------------

\newtheorem{قضیه}{قضیه}
\newtheorem{لم}{لم}
\newtheorem{مشاهده}{مشاهده}
\newtheorem{تعریف}{تعریف}


%-------------------------- Notations ------------------------------------

\newcommand{\IR}{\ensuremath{\mathbb{R}}} 
\newcommand{\IZ}{\ensuremath{\mathbb{Z}}} 
\newcommand{\IN}{\ensuremath{\mathbb{N}}} 
\newcommand{\IS}{\ensuremath{\mathbb{S}}} 
\newcommand{\IC}{\ensuremath{\mathbb{C}}} 
\newcommand{\IB}{\ensuremath{\mathbb{B}}} 

\newcommand{\bR}{\mathbb{R}}
\newcommand{\cB}{\mathcal{B}}
\newcommand{\cO}{\mathcal{O}}
\newcommand{\cG}{\mathcal{G}}
\newcommand{\rM}{\mathrm{M}}
\newcommand{\rC}{\mathrm{C}}
\newcommand{\rV}{\mathrm{V}}

\newcommand{\lee}{\leqslant}
\newcommand{\gee}{\geqslant}
\newcommand{\ceil}[1]{{\left\lceil{#1}\right\rceil}}
\newcommand{\floor}[1]{{\left\lfloor{#1}\right\rfloor}}
\newcommand{\prob}[1]{{\mbox{\tt Pr}[#1]}}
\newcommand{\set}[1]{{\{ #1 \}}}
\newcommand{\seq}[1]{{\left< #1 \right>}}
\newcommand{\provided}{\,|\,}
\newcommand{\poly}{\mbox{\rm poly}}
\newcommand{\polylog}{\mbox{\rm \scriptsize polylog}\,}
\newcommand{\comb}[2] {\left(\!\!\begin{array}{c}{#1}\\{#2}\end{array}\!\!\right)}




\newcounter{probcnt}
\newcommand{\مسئله}[1]{\stepcounter{probcnt}\bigskip\bigskip{
 	\large \bf مسئله‌ی \arabic{probcnt}$\mbox{\bf{.}}$ \ #1} \bigskip}

\newcommand{\fqed}[1]{\leavevmode\unskip\nobreak\quad\hspace*{\fill}{\ensuremath{#1}}}

\newenvironment{اثبات}
	{\begin{trivlist}\item[\hskip\labelsep{\em اثبات.}]}
	{\fqed{\square}\end{trivlist}}

\newenvironment{حل}
	{\begin{trivlist}\item[\hskip\labelsep{\bf حل.}]}
	{\fqed{\blacktriangleright}\end{trivlist}}

\ifdefined\hidesols
	\newsavebox{\trashcan} % uncomment the following line to hide solutions
	\renewenvironment{حل}{\begin{latin}\begin{lrbox}{\trashcan}}{\end{lrbox}\end{latin}}
\fi


%------------------------- Header -----------------------------

%------------------------- Header -----------------------------

\newcommand{\سربرگ}[3]{
	\parindent=0em
	\begin{shaded}
		
		\rightline{ 
			\makebox[6em][c]{
				\includegraphics[height=1.6cm]{aut.png}
		}}
		\vspace{-.2em}
		{\scriptsize\bf دانشکده‌ی مهندسی کامپیوتر}
		\hfill {\small
			مدرس: دکتر امیرمزلقانی  \ 
		}\\[-5em]
		\leftline{\hfill\Large\bf 
			کاربرد های جبر خطی
		}\\[.7em]
		\leftline{\hfill\bf 
			نیم‌سال دوم ۹۶-۹۷
		}\\[1.7em]
		\hrule height .12em
		
		\normalsize
		\vspace{1mm} #1
		\hfill \small  #3
		\vspace{1mm} 
		\hrule height .1em
		
		\vspace{-0.5em} 
		\hfill {\sayeh\large #2} \hfill
	\end{shaded}
	\begin{large}

توجه:
\end{large}
\\
\\
\begin{itemize}
\item
این تمرین از مباحث مربوط به فصل 3 و 4 طراحی شده است که شامل 8 سوال اجباری و 2 سوال امتیازی است که نمره سوال های امتیازی فقط به نمرات تمرین شما کمک می کند.

\item 
اگه سوالی داشتین از طریق
\begin{latin}
\href{mailto:aut.la2018@gmail.com}{\[aut.la2018@gmail.com\]}
\end{latin}
 حتما بپرسید.

\item پاسخ های تمرین را در قالب یک فایل به صورت الگوی زیر آپلود کنید.
\begin{latin}
9531000\_Jonatan\_Vannieuwenhoven\_HW3.pdf
\end{latin}
\item \color{red}
مهلت تحویل جمعه 21 اردیبهشت 1397 ساعت 23:54:59
\end{itemize}

}




%\settextfont{Yas}
\begin{document}
\سربرگ{پاسخ تمرین های فصل3و4}{}{}
\clearpage
\مسئله{}درستی یا نادرستی عبارت های زیر را مشخص کنید و دلیل آن را بیان کنید.
\\
(
از بین 21 مورد  15 مورد را به اختیار انتخاب کرده و مشخص کنید.)

\begin{enumerate}[label=\alph*)]
	
	\item 
	$$det (S^{-1}AS)  = det(S^{-1}) det (A) det(S) = \frac{1}{det(S) }det(A) det(S) = det(A)$$
	\item 
	مثال نقض:
	$$A = \begin{bmatrix}
	1& 0 \\
	0 &  1\\
	\end{bmatrix} \rightarrow det(A) = 1$$
	$$ 4 \times A = \begin{bmatrix}
	4& 0 \\
	0 &  4\\
	\end{bmatrix}  \rightarrow det(4A) = 16 \ne 4$$
	\item 
	$$A = \begin{bmatrix}
	1\\
	0\\
	\end{bmatrix} , B = A^{T}  $$
	$$AB = \begin{bmatrix}
	1 & 0\\
	0 & 0\\
	\end{bmatrix} \rightarrow det(AB) = 0$$
	$$BA = \begin{bmatrix}
	1 \\
	\end{bmatrix} \rightarrow det(BA) = 1$$
	توجه: تنها در صورتی که A,B مربعی باشند، داریم:
	$$ det(AB) = det(A) det(B) = det(B) det(A) = det(BA)$$
	
	\item 
	مثال نقض:
	$$A = \begin{bmatrix}
	1 &2\\
	3 & 4 \\
	\end{bmatrix} , B =\begin{bmatrix}
	1 &0\\
	1 & 1 \\
	\end{bmatrix}  $$
	
	$$AB = \begin{bmatrix}
	3 &2\\
	7 & 4 \\
	\end{bmatrix} , BA =\begin{bmatrix}
	1 &2\\
	4 & 6 \\
	\end{bmatrix}  \rightarrow det(AB-BA) = det(\begin{bmatrix}
	2 &0\\
	3 & -2 \\
	\end{bmatrix}) = -4 $$
	
	\item 
	باتوجه به اینکه معکوس پذیری برای ماتریس های مربعی تعریف می شود:
	ماتریس A معکوس ناپذیر است اگر و تنها اگر دترمینان آن صفر باشد پس:
	$$ det(A) = 0 , det(AB) = det(A) det(B) \rightarrow det(AB)= 0 \times det(B) = 0 $$
	پس ماتریس AB نیز معکوس پذیر نمی باشد.
	
	\item 
	باتوجه به اینکه دترمینان برای ماتریس های مربعی تعریف می شود:
	$$ det(AA^T) = 1 \rightarrow =det(A) det(A^T) = det(A)^2 = 1 \rightarrow det(A) = \pm 1$$
	
	\item 
	
	$$det(A) = det(A^T) = det (-A)= (-1)^ndet(A)$$
	اگر n فرد باشد:
	$$(-1)^ndet(A) = -det(A)$$
	$$ \rightarrow det(A) = -det(A) \rightarrow det(A) = 0$$
	
	اگر n زوج باشد:
	$$(-1)^ndet(A) = det(A)$$
	$$ \rightarrow det(A) = det(A)$$
	و الزامی به صفر بودن دترمینان A وجود ندارد.
	\item
	$$A^{-1} = \frac{1}{det(A)}Adj(A), det(A) = 1 \rightarrow A^{-1}=Adj(A) $$
	با توجه به اینکه داریه های ماتریس A اعداد صحیح هستند، پس دترمینان آن و دترمینان هر زیر ماتریس از آن نیز عددی صحیح خواهد بود پس ماتریس cofactor  ماتریس A نیز از درایه های صحیح تشکیل شده است و در نتیجه عناصر adjugate ماتریس A نیز اعدادی صحیح هستند.
	$$(Adj_{ij} = C_{ji} = (-1)^{i+j}M_{ji})$$
	
	\item 
	$$det A^{-1} = det \frac {1}{ad-bc} 
	\begin{bmatrix}
	d & -b \\
	-c &  a\\
	\end{bmatrix} = \frac{ad-bc}{(ad-bc)^2} =\frac{1}{(ad-bc)} $$
	
	\item 
	اگر ماتریس A مربعی نباشد 
	$det(A^TA) \ne det(A^T) det(A)$ 
	پس رابطه فوق برای هر ماتریس P صدق نمی کند.
	\item 
	مثال نقض:
	$$
	A = \begin{bmatrix}
	1 & 2& 3 \\
	4 & 5& 6 \\
	7 & 8& 9 \\
	\end{bmatrix}
	$$
	$$det(A) = a_{11}C_{11}+a_{12}C_{12}+a_{13}C_{13} = 1 \times (-3) + 2 \times 6  + 3 \times (-3) = 0$$ 
	\item
	مثال نقض:
	$$det(
	\begin{bmatrix}
	1 & 1& 0 \\
	0 & 1& 1 \\
	1 & 0& 1 \\
	\end{bmatrix}) = 2
	$$
	\item 
	
	(I)
	$$
	(A^T)^{-1} = \frac{adj(A^T)}{det(A^T)}  \rightarrow adj(A^T) = (A^T)^{-1} det(A^T) = (A^{-1})^T det(A)  $$
	(II)
	$$A^{-1} = \frac{adj(A)}{det(A)} \rightarrow adj(A) = A^{-1} det(A)  \rightarrow (adj(A))^T = (A^{-1} det(A))^T = det(A)(A^{-1})^T 
	$$
	$$
	(I, II)\rightarrow adj(A^T) =  (adj(A))^T
	$$
	\item 
	اگر A معکوس پذیر باشد:
	$
	A^{-1} = \frac{adj(A)}{det(A)}$
	$$\rightarrow (A^{-1})^{-1} = \frac{1}{det(A)}Adj(A^{-1})
	$$
\end{enumerate}
   

\مسئله{}
در هر مورد مشخص کنید آیا زیر مجموعه ی داده شده یک زیر فضا از فضای برداری مشخص شده می باشد یا خیر.
\begin{حل}
 برای حل این سوال با فرض اینکه  $H$ زیر فضایی از $V$ باشد آنگاه باید در هر مورد ثابت کنیم:

\begin{enumerate}[label=\arabic*),align=left,leftmargin=*]
	\item
	بردار صفر باید عضو $H$ می باشد.
	\item 
	$H$
	باید نسبت به جمع بردار ها  بسته باشد. یعنی برای هر $V$و عضو $H$،
	$v+u$
	هم باید عضو 
	$H$
	باشد.
	\item 
	$H$
باید تحت ضرب عددی (sacalrs) بسته باشد.یعنی برای هر $u$ عضو $H$ و هر اسکالر$c$ بردار $cu$ عضو $H$ باشد.	
	\end{enumerate}

\begin{enumerate}
	\item 
	$\{(x,y)\in\mathbb{R}^2|x^2+y^2 \leq4\}$
	در فضای برداری 
	$\mathbb{R}^2$
	
	
\begin{حل}
	به وضوح 
	$(0,0)$
	عضو این مجموعه است زیرا 
	$0^2+0^2\leq 4$.
	حال باید شرط دوم را ثابت کنیم، برای این فرض کنید 
	$x^2+y^2\leq4$
	و 
	$x'^2+y'^2\leq4$
	آنگاه باید ثابت کنیم :
	$(x+x')^2+(y+y')^2\leq 4$
	
	که این موضوع مثلا برای 
	$(2,2),(1,1)$
	هر دو عضو این زیر فضا هستند اما 
	$(3,3)$
	در شرط صدق نمی کند پس زیر فضا نیست.
	
	\end{حل}
	\item
	$\{(a,b,c)\in\mathbb{R}^3|a+b+2c=0  \}$
		در فضای برداری 
	$\mathbb{R}^3$.
	\begin{حل}
	
	
	به وضوح 
	$(0,0,0)$
	عضو این زیر فضا است زیرا 
	$0+0+2\times0=0$
	حال فرض کنید 
	$(a,b,c),(a',b',c')$
	عضو این زیر فضا باشند آنگاه 
	$(a+a',b+b',c+c')$
	نیز عضو این زیر فضا است زیرا: 
	$$a+a'+b+b'+2(c+c')=a+b+2c+a'+b'+2c'=0+0=0$$
	حال باید ثابت کنیم اگر
	$k$
	اسکالر باشد و 
	$(a,b,c)$
	عضو این زیر فضا آنگاه :
	$k(a,b,c)$
	نیز عضو این زیر فضا هست که این نیز برقرار است زیرا:
	$$ka+kb+2kc=k(a+b+2c)=k\times0=0$$
	\end{حل}

	\item
	$\{A\in M_n(\mathbb{R})|A^2=A\}$
	در
	$M_n(\mathbb{R})$.
	(منظور از 
	$M_n(\mathbb{R})$
	مجموعه تمام ماتریس های 
	$n\times n$
	با درایه هایی از مجموعه اعداد حقیقی است.)
	\begin{حل}
	به وضوح ماتری صفر عضو این زیر فضا است،حال باید ثابت کنیم اگر 
	$A$
	و 
	$B$
	دو ماتریس باشند که عضو این زیر فضا هستند انوقت برای زیر فضا بودن باید نشان دهیم 
	$$(A+B)^2=A+B$$
فرض کنید 
$A=I$
 می دانیم 
 $I^2=I$
 پس 
 $I$
 عضو این زیر فصا است. اما اگر فرض کنیم
 $A=I,B=I$
 آنگاه:
 $$(A+B)^2=(2I)^2=4I\neq2I=I+I=A+B$$
 پس این زیر مجموعه یک زیر فضای برداری نیست.
	\end{حل}
	 
	\item 
	$\{p(x)|2p(0)=p(1),p(x)\in\mathbb{P} [x]  \}$
	در فضای برداری 
	$\mathbb{P}[x]$
	(تمامی چند جمله های حداکثر از درجه 
	$n$
	با ضرایب حقیقی را با نماد  
	$\mathbb{P}_n[x]$
	نشان می دهیم و همچنین مجموعه تمام چند جمله ها با ضرایب حقیقی را با  
		$\mathbb{P}[x]$
		نشان می دهیم) 
		\begin{حل}
		برای حل این سوال 
		$p(x)=0$
		عضو این فضای برداری است زیرا :
		$2p(0)=p(1)=0$
		حال فرض کنید 
		$p_1(x),p_2(x)$
		دو عضو این زیر فضا باشند انگاه:
		$$2(p_1(0)+p_2(0))=2p_1(0)+2p_2(0)=p_1(1)+p_2(1)$$
		پس شرط دوم نیز برقرار است.
		حال برای اثبات شرط سوم فرض کنید 
		$k$
		یک اسکالر باشد آنگاه باید ثابت کنیم 
		$kp_1(x)$
	عضو این زیر فضاس که این موضوع نیز قابل اثبات است زیرا:
	$$2kp_1(0)=k2p_1(0)=kp_1(1)$$
	پس مجموعه فوق یک زیر فضاس.				
		\end{حل}
		\item 
		$\{p(x)|p(x)=a+x^2,a\in\mathbb{R}\}$
		در فضای برداری 
		$\mathbb{P}_3[x]$.
		\begin{حل}
		این مجموعه زیر فضا نمی باشد برای مثال دو بردار
		$1+x^2$
		و
		$2+x^2$
		را در نظر بگیرید آنگاه مجموع این دو بردار به شکل 
		$3+2x^2$
		خواهد بود که به شکل 
		$a+x^2$
		نمی باشد.	
		\end{حل}
		
	\end{enumerate}
\end{حل}
\مسئله{}
اگر 
$\mathbb{P}[x],\mathbb{P}_n[x]$
طبق تعریف بالا فضا های برداری با ضرایب حقیقی باشند آنگاه :
\begin{enumerate}
\item 
نشان دهید  اگر 
$\{1,x,x^2,\cdots,x^{n-1}\}$
پایه ای برای 
$\mathbb{P}_n[x]$
باشد آنگاه:
$$\{1,(x-a),(x-a)^2,\cdots,(x-a)^{n-1}\},\qquad a\in \mathbb{R}$$
نیز پایه ای برای 
$\mathbb{P}_n[x]$	
است.
\begin{حل}
	از انجاییکه 
	$\{1,x,x^2,\cdots,x^n\}$
	یک پایه برای 
	$\mathbb{P}_n[x]$
	است کافی است ثابت کنیم :
	$$\{1,(x-a),(x-a)^2,\cdots,(x-a)^{n-1}\}$$
	این مجموعه مستقل خطی است زیرا در صورت مستقل خطی بودن چون تعداد اعضای دو مجموعه برابر است پس 
	$\{1,(x-a),(x-a)^2,\cdots,(x-a)^{n-1}\}$
	یک پایه است.برای اثبات مستقل خطی بودن ضرایب 
	$\{\lambda_0,\lambda_1,\lambda_2,\cdots,\lambda_{n-1}\}$
	در نظر می گیریم باید ثابت کنیم اگر:
	$$\lambda_0+\lambda_1(x-a)+\lambda_2(x-a)^2+\cdots+\lambda_{n-1}(x-a)^{n-1}=0$$
آنگاه:
$$\lambda_0=\lambda_1=\lambda_2=\cdots=\lambda_{n-1}=0$$
برای اثبات این موضوع یکبار 
$x=a$
در نظر می گیریم آنگاه 
$\lambda_0=0$
می شود،در مرحله بعد از 
$x-a$
فاکتور می گیریم و نتیجه می گیریم 
$\lambda_1=0$
و همینطور الی آخر،پس استقلال خطی ثابت شد 
 $$\{1,(x-a),(x-a)^2,\cdots,(x-a)^{n-1}\}$$
پایه است.


\end{حل}
\item
مختصات 
$$f(x)=a_0+a_1x+\cdots+a_{n-1}x^{n-1}\in \mathbb{P}_n[x]$$
را نسبت به پایه 
$$\{1,(x-a),(x-a)^2,\cdots,(x-a)^{n-1}\},\qquad a\in \mathbb{R}$$
بیابید.
\begin{حل}
برای پیدا کردن ضرایب فرض کنیم 
$\lambda_0,\lambda_1,\lambda_2,\cdots,\lambda_{n-1}$	
ضرایب مورد نظر ما بر اساس پایه 
\\
$\{1,(x-a),(x-a)^2,\cdots,(x-a)^{n-1}\}$
باشند آنگاه داریم:
$$f(x)=a_0+a_1x+\cdots+a_{n-1}x^{n-1}=\lambda_0+\lambda_1(x-a)+\cdots+\lambda_{n-1}(x-a)^{n-1}$$
حال 
$x=a$
قرار می دهیم آنگاه 
$f(a)=\lambda_0$
حال از دو طرف تساوی مشتق می گیریم:
$$f'(x)=a_1+2a_2x+\cdots+n-1a_{n-1}x^{n-2}=\lambda_1+2\lambda_2(x-a)+\cdots+(n-1)\lambda_{n-1}(x-a)^{n-2}$$
پس نتیجه می گیریم 
$f'(a)=\lambda_1$
در حالت کلی نتیجه می شود ضرایب 
$\lambda$
به صورت 
$(f(a),\frac{f'(a)}{1!},\cdots,\frac{f^{(n-1)}(a)}{(n-1)!})$
خواهد بود که 
$f^{(i)}$
مشتق مرتبه 
$i$
ام می باشد.
\end{حل}
\item 
فرض کنید 
$a_1,a_2,\cdots,a_n\in \mathbb{R}$
و متمایز باشند.برای هر 
$i=1,2,\cdots,n$

$$f_i(x)=(x-a_1)\dots(x-a_{i-1})(x-a_{i+1})\dots(x-a_n)$$
را در نظر بگیرید،نشان دهید 
$\{f_1(x),f_2(x),\cdots,f_n(x)\}$
نیز پایه ای برای 
$\mathbb{P}_n[x]$
است.
\begin{حل}
از انجاییکه تعداد اعضای 	$\{f_1(x),f_2(x),\cdots,f_n(x)\}$ برابر تعداد اعضای پایه است برای اثبات پایه بودن کافی است ثابت کنیم این مجموعه مستقل خطی است یعنی به ازای هر 
$\alpha_1,\alpha_2,\cdots,\alpha_n$
اگر 
$$\alpha_1f_1(x)+\alpha_2f_2(x)+\cdots+\alpha_nf_n(x)=0$$
باشد انگاه 
$\alpha_i$
ها مساوی صفر هستند برای اثبات این موضوع 
$a_i$
را در این عبارت جایگذاری می کنیم آنگاه داریم :
$$\alpha_1f_1(a_i)+\cdots+\alpha_if(a_i)+\cdots+\alpha_nf_n(a_i)=0$$
در این صورت به ازای هر 
$f_j$
که
$i\neq j$
،
$f_j(a_i)=0$
 پس از اینجا نتیجه می شود تمامی 
 $\alpha_i$
 ها مساوی صفر هستند و استقلال خطی ثابت می شوند که پایه بودن را نتیجه می دهد.
\end{حل}
\end{enumerate}
\مسئله{}
فرض کنید 
$W_1,W_2$
زیر فضا های فضای برداری 
$V$
باشند، تعریف می کنیم :
$$W_1+W_2=\{w_1+w_2|w_1\in W_1,w_2\in W_2\}$$.

\begin{enumerate}
	\item 
	نشان دهید :
	$$W_1+W_2+\cdots+W_n=span(\bigcup_{i=1}^{n}W_i)$$
	\begin{حل}
	$$v\in W_1+W_2+\cdots+W_n\longleftrightarrow \exists \ \ w_1,w_2,\cdots,w_n \quad w_1\in  W_1,w_2\in  W_2,\cdots w_n\in W_n \quad v=w_1+w_2+\cdots+w_n$$
	$$\longleftrightarrow w_1,w_2,\cdots,w_n\in\bigcup_{i=1}^{n}W_i\longrightarrow w_1+w_2+\cdots+w_n\in span(\bigcup_{i=1}^{n}W_i)\longleftrightarrow v\in span(\bigcup_{i=1}^{n}W_i) $$	
	\end{حل}
	\item 
	نشان دهید 
	$W_1\cap W_2,W_1+W_2$
	زیر فضای 
	$V$
	هستند و همچنین نشان دهید:
	$$W_1\cap W_2 \subseteq W_1\cup W_2\subseteq W_1+W_2$$.
	\begin{حل}
		می دانیم 
		$0\in W_1,0\in W_2$
		پس 0 عضو 
		$W_1+W_2$
		هست از سوی دیگر 
		\\
		اگر 
		$v_1\in W_1+W_2,v_2\in W_1+W_2$
		باشد،آنگاه طبق تعریف داریم :
		$$\exists w_1\in W_1,w_2\in W_2 \ \ v_1=w_1+w_2\quad,\quad \exists w'_1\in W_1,w'_2\in W_2 \ \ v_2=w'_1+w'_2$$
		در نتجه:
		$$v_1+v_2=w_1+w_2+w'_1+w'_2=\underbrace{w_1+w'_1}_{\in W_1}+\underbrace{w_2+w'_2}_{\in W_2}\longrightarrow v_1+v_2\in W_1+W_2$$
		همچنین باید ثابت کنیم اگر 
		$v\in W_1+W_2$
		باشد آنگاه 
		$kv$
		هم چنین است که 
		$k$
		یک اسکالر است.
		$$\exists w_1\in W_1,w_2\in W_2 \ \ v=w_1+w_2\longrightarrow kv=\underbrace{kw_1}_{\in W_1}+\underbrace{kw_2}_{\in W_2}\longrightarrow kv\in W_1+W_2$$
		پس 
		$W_1+W_2$
		یک زیر فضای 
		$V$
		است.
		
		حال باید ثابت کنیم 
		$W_1\cap W_2 $
		زیر فضای 
		$V$
		است. می دانیم صفر عضو هر دو زیر فضا است پس صفر در اشتراک آن ها نیز وجود دارد،حال باید ثابت می کنیم که :
		$$v_1\in W_1\cap W_2, v_2\in W_1\cap W_2\longrightarrow v_1\in W_1\wedge v_1\in W_2, v_2\in W_1,v_2\in W_2$$$$\longrightarrow v_1+v_2\in W_1\wedge v_1+v_2\in W_2\longrightarrow v_1+v_2\in W_1\cap W_2 $$
		به همین شکل ثابت می شود ضرب یک اسکالر در اعصای 
		$W_1\cap W_2$
		عضو 
		$W_1\cap W_2$
		است پس زیر فضا بودن اشتراک دو زیر فضا نیز ثابت می شود.
		
		اکنون باید ثابت کنیم رابطه بالا برقرار است بدیهی است که اشتراک دو زیر فضا زیر مجموعه اجتماع آن است (این رابطه برای هر دو مجموعه ای فارغ از زیر فضا بودن یا نبودن صادق است) کافی است ثابت کنیم: 
		$$W_1\cup W_2\subseteq W_1+W_2$$
		برای اثبات این موضوع از رابطه قسمت 1 استفاده می کنیم، طبق قسمت 1 می دانیم: 
		$$W_1+W_2=span(W_1\cup W_2)$$
		واضح است که اگر مجموعه ای به شکل 
		$A$
		داشته باشیم آنگاه:
		$$A\subseteq span(A)$$
		زیرا :
		$$span(A)=\lambda_1a_1+\lambda_2 a_2+\cdots+\lambda_n a_n\qquad \lambda_i\in \mathbb{R},a_i\in A$$
		و فرض کنید در هر مرحله 
		$(\lambda_i=1)$
		و 
		$(\lambda_j=0 , j\neq i)$
		در این صورت 
		$A\subseteq span(A)$.
	\end{حل}
	\item 
	نشان دهید :
	$$dim(W_1+W_2)=dim(W_1)+dim(W_2)-dim(W_1\cap W_2)$$.
	\begin{حل}
	فرض کنیم :
	$$diam W_1=n,dim W_2=m,dim(W_1\cap W_2)=t$$
	همچنین فرض کنید:
	$\{u_1,u_2,\cdots,u_t\}$
یک پایه برای 
$W_1\cap W_2$
باشد،پس می توان آنرا به یک پایه 
\\
$B_1=\{u_1,u_2,\cdots,u_t,v_1,v_2,\cdots,v_{n-t}\}$
از 
$W_1$
و همچنین  
$B_2=\{u_1,u_2,\cdots,u_t,w_1,w_2,\cdots,w_{m-t}\}$
از 
$W_2$
توسعه داد.ثابت می کنیم:
$$B=\{u_1,u_2,\cdots,u_t,v_1,v_2,\cdots,v_{n-t},w_1,w_2,\cdots,w_{m-t}\}$$
یک پایه برای 
$W_1+W_2$
است.که رد این صورت حکم مسئله نیز ثابت می شود ،برای اثبات پایه بودن باید استقلال خطی و مولد بودن را ثابت می کنیم(مولد بودن یعنی ترکیب های خطی یک مجموعه تمامی اعضای آن مجموعه را تولید می کنند که در واقع بیانگر این است که اگر 
$if\ \ A\subseteq B\to span(A)=B$
در بحث ما می گوییم بردار هایی مولد یک فضا یا زیر فضا هستند که تمامی اعضای آن ها را بتوان با ترکیب خطی این  بردار ها تولید کرد.
)
	\\
	{\bf  استقلال خطی :}
	$$\sum_{i=1}^{t}\alpha_iu_i+\sum_{i=1}^{n-t}\beta_i v_i+\sum_{i=1}^{m-t}\gamma_i w_i=0(\star)\longrightarrow \underbrace{\sum_{i=1}^{t}\alpha_iu_i+\sum_{i=1}^{n-t}\beta_i v_i}_{\in W_1}=\underbrace{\sum_{i=1}^{m-t}-\gamma_i w_i}_{\in W_2} $$
	$$\longrightarrow \sum_{i=1}^{m-t}-\gamma_i w_i\in W_1\cap W_2 $$
	پس وجود دارد 
	$\mu_1,\mu_2,\cdots,\mu_t$
	به طوری که:
	$$\sum_{i=1}^{m-t}-\gamma_i w_i=\sum_{i=1}^{t}\mu_iu_i$$
	$$\longrightarrow \sum_{i=1}^{m-t}\gamma_i w_i+\sum_{i=1}^{t}\mu_iu_i=0$$
	چون ترکیب خطی فوق صفر ،
	$\mu_i$
	ها ،
	$w_i$
	ها یک پایه برای 
	$w_2$
	و ذا مستقل خطی هستند 
	
	پس :
	$\forall i\ \ \gamma_i=0,\forall i\ \ \mu_i=0$
		با جایگذاری در 
		$\star$
		داریم :
		$$\sum_{i=1}^{t}u_i+\sum_{i=1}^{n-t}=0$$
		یعنی ترکیب خطی از اعضای پایه 
		$W_1$
		صفر شده است ،پس :
		$$\forall i \alpha_i=0,\forall i \beta_i=0$$
		پس 
		$B$
		مستقل خطی است.
		
		{\bf  مولد بودن:}
		باید ثابت کنیم هر 
		$w\in W_1+W_2$
		را می توان به صورت ترکیب خطی 
		$B$
		نوشت.
		
		می دانیم طبق تعریف :
		$$\exists w'_1\in W_1,w'_2\in W_2\quad w=w'_1+w'_2$$
		$$\longrightarrow w'_1=\alpha_1u_1+\alpha_2u_2+\cdots+\alpha_tu_t+\alpha_{t+1}v_1+\alpha_{t+2}v_2+\cdots+\alpha_{n}v_{n-t}$$
		$$\longrightarrow w'_2=\beta_1u_1+\beta_2u_2+\cdots+\beta_tu_t+\beta_{t+1}w_1+\beta_{t+2}w_2+\cdots+\beta_{m}w_{m-t}$$
		\begin{align*}\longrightarrow w=w'_1+w'_2=&(\alpha_1+\beta_1)u_1+(\alpha_2+\beta_2)u_2+\cdots+(\alpha_t+\beta_t)u_t+\\&\alpha_{t+1}v_1+\alpha_{t+2}v_2+\cdots+\alpha_{n}v_{n-t}+\beta_{t+1}w_1+\beta_{t+2}w_2+\cdots+\beta_{m}w_{m-t}
		\end{align*}
		پس توانستیم 
		$w$
		را برحسب 
		$B$
		بنویسیم و در نتیجه 
		$B$
		مولد و مستقل خطی است و پایه است و حکم ثابت می شود.
\end{حل}
	\item 
	نتیجه گیری قسمت 
	$2$
	را با استفاده از دو خط که از مبدا مختصات 
	$xy$
	می گذرند توجیه کنید.
	\begin{حل}
		فرض کنید دو خط متقاطع داریم که در یک نقطه مشترکند،خط اول را با 
		$W_1$
		و خط دوم را با 
		$W_2$
		نشان می دهیم.
		آنگاه :
		$W_1\cap W_2$
		یک نقطه خواهد بود،و 
		$W_1\cup W_2$
		از خود این دو خط متقاطع تشکیل خواهد شد.در این صورت 
		$W_1+W_2$
		صفحه گذرنده از این دو خط خواهد بود.که رابطه 2 به وضوح با این فرضیات مشخص است.
		\end{حل}
	\item 
	درستی یا نادرستی تساوی زیر را بررسی کنید در صورت درست بودن اثبات و در صورت نادرست بودن مثال نقض بزنید:
	$$W_3\cap(W_1+W_2)=(W_3\cap W_1)+(W_3\cap W_2)$$.
	\begin{حل}
		برای اینکه نشان دهیم این تساوی برقرار نیست،فرض می کنیم 
		$W_1,W_2,W_3$
		سه خط هستند که در مبدا مختصات مشترکند.مثلا فرض کنید 
		$W_1$
		محور 
		$x$
		ها ،
		$W_2$
		محور 
		$y$
		ها و 
		$W_3$
		نیمساز ناحیه اول و سوم است،
		در این صورت 
		$W_3\cap(W_1+W_2)$
		یک خط خواهد بود ولی 
		$(W_3\cap W_1)+(W_3\cap W_2)$
		همان نقطه صفر خواهد بود پس مشاهده می کنید که تساوی برقرار نیست.
		
		
	\end{حل}
	\item 
	اگر 
	$W_1\cap W_2=\{0\}$
	باشد آنگاه به 
	$W_1+W_2$
	جمع مستقیم نیز می گویند و آن را با 
	$W_1\bigoplus W_2$
	نشان می دهند،ثابت کنید اگر 
	$V_1$
	زیر فضایی از فضای برداری 
	$V$
	باشد و اگر زیر فضای برداری یکتای 
	$V_2$
	موجود باشد که 
	$V=V_1\bigoplus V_2$
	آنگاه 
	$V_1=V$.
	\begin{حل}
	برای اثبات این سوال به برهان خلف فرض کنید 
	$V_1\neq V$
	در این صورت 
	$dim V_1< dim V$
	فرض کنید
	$$\{\alpha_1,\alpha_2,\cdots,\alpha_m,\alpha_{m+1},\cdots,\alpha_n\}$$
	یک پایه برای 
	$V$
	باشد،که 	
	$\alpha_1,\alpha_2,\cdots,\alpha_m\in V_1$
	(این موضوع ممکن است زیرا در واقع می توانیم یک پایه برای 
	$V_1$
	در نظر بگیریم و آن را به پایه ای از 
	$V$
	گسترش دهیم.)
	فرض کینم 
	$V_2=span(\alpha_{m+1},\cdots,\alpha_n)$
	و 
	$V_3=span(\alpha_{m+1}+\alpha_1,\cdots,\alpha_n)$
	در این صورت 
	$V_2,V_3$
	با
	$V_1$
	عضو مشتر ک ندارند که در این صورت
	$V=V_1\bigoplus V_2$
	و
	$V=V_1\bigoplus V_3$
	که 
	$V_2\neq V_3$
	و این با یکتایی وجود عضوی مانند 
	$V_2$
	در تناقض است پس فرض خلف باطل و حکم درست است.
	
	\end{حل}
	
\end{enumerate}
\مسئله{}
فرض کنید 
$T:V\longrightarrow W$
نگاشت خطی باشد،
$Nul\ \ T=\{0\}$
اگر و فقط اگر 
$T$
هر زیر مجموعه مستقل خطی را به زیر مجموعه مستقل خطی نگاشت کند. علاوه بر ویژگی های بالا اگر 
$A$
ماتریس استاندارد تبدیل 
$T$
باشد که به ازای هر 
$b$
که 
$b$
مختصات برداری در 
$W$
است وجود داشته باشد 
$x$
ای که مختصات برداری در 
$V$
 باشد که
 $Ax=b$
 باشد آنگاه 
 $T$
 هر پایه
 $V$
  را به پایه ای در 
  $W$
   می نگارد.
   \begin{حل}
   	فرض کنید مجموعه مستقل خطی 
   	$\{v_1,v_2,\cdots,v_n\}$
   	را تحت 
   	$T$
   	نگاشت کنیم می خواهیم ثابت کنیم بردار های نگاشت شده مستقل خطی هستند پس داریم:
   	$$\alpha_1T(v_1)+\alpha_2T(v_2)+\cdots+\alpha_nT(v_n)=0\longrightarrow T(\alpha_1v_1+\alpha_2v_2+\cdots+\alpha_nv_n)=0\longrightarrow \alpha_1v_1+\alpha_2v_2+\cdots+\alpha_nv_n=0  $$
   	چون 
   	$v_i$
   	ها مستقل خطی هستند پس:
   	$$\forall i\quad\alpha_i=0$$
در نتیجه استقلال خطی نگاشت یک مجموعه مستقل خطی ثابت می شود.برای اثبا عکس قضیه  به برهان خلف فرض کنید وجود داشته با شد برداری مثل 
$v\neq0$
که 
$T(v)=0$
در این صورت مجموعه 
$\{v\}$
مستقل خطی است در حالی که تصویر آن صفر می شود که وابسته خطی است و این با فرض تناقش دارد پس فرض خلف باطل و حکم درست است.
برای اثبات قسمت دوم ابتدا ثابت می کنیم اگر 
$T:V\longrightarrow W$
تبدیلی پوشا است اگر و فقط اگر هر هر مولد 
$V$
به مولدی از 
$W$
بنگارد.ابتدا این لم را ثبات می کنیم، فرض کنیم 
$T$
مولدی از 
$V$
را به مولدی در 
$W$
ننگارد و فرض کنید 
$C=\{v_1,v_2,\cdots,v_n\}$
مجموعه مولد مورد نظر ما باشد.از انجاییکه 
$C$
مولد است پس :
$$\forall v\in V \quad v=\alpha_1v_1+\alpha_2v_2+\cdots+\alpha_nv_n\longrightarrow T(v)=\alpha_1T(v_1)+\alpha_2T(v_2)+\cdots+\alpha_nT(v_n)\qquad \star$$
پس تصویر هر بردار در 
$V$
را می توان به صورت ترکیب خطی بردار های 
$$C'=\{T(v_1),T(v_2),T(v_3),\cdots,T(v_n)\}$$ 
نوشت اما از انجاییکه 
$C'$
مولد نیست وجود دارد برداری مثل 
$w$
که نمی توان آن را به صورت ترکیب خطی بردار های 
$C'$
نوشت پس هیچکدام از برداری های 
$V$
به 
$w$
نگاشت نمی شود و این با پوشا بودن 
$T$
در تناقض است. برای اثبات عکس لم از 
$\star$
استفاده می کنیم و ثابت می کنیم که پوشاست. حال برای اثبات قسمت دوم سوال چون تساوی 
$Ax=b$
به ازای هر 
$b$
جواب دارد پس پوشاست و از پوشایی نتیجه می شود که تبدیل خطی 
$T$
هر مولد را به مولد می نگارد و از قسمت قبل دیدیم هر پایه را به پایه می نگارد پس در نتیجه 
$T$
هر پایه را به پایه می نگارد.
   	
   	
   	\end{حل}
   
   \مسئله {} فرض کنید 
   $T:V\longrightarrow V$
   تبدیل خطی رو فضای متناهی البعد 
   $V$
   باشد و 
   $T^2=0$.
   ثابت کنید 
   \\
$2rank(A)\leq dim(V)$
   (
   $A$
   ماتریس استاندارد تبدیل 
   $T$
   است.
   )
   
   \begin{حل}
برای اثبا این موضوع می دانیم برای هر تبدیل خطی مانند 
$T:V\longrightarrow W$
داریم :
$$dim V=dim(null(T))+dim(range\ \ T) \qquad \star$$
از سوی دیگر می دانیم :
$$rank\ \ A=diam(range \ \ T)$$
حال در مسئله داریم :
$$T(T(v))=0$$
از این موضوع نتیجه می گیریم:
$$range\ \ T\subseteq null\ \ T\longrightarrow diam(range\ \ T)\leq diam(null\ \ T)$$
حال با توجه به 
$\star$
داریم 
$$dim V-dim(range\ \ T)=dim(null(T))$$
و این را در نامساوی به دست آمده جایگذاری می کنیم:
$$diam(range\ \ T)\leq dim V-dim(range\ \ T)\longrightarrow2dima(range \ \ T)\leq dim(V)\longrightarrow2rank(A)\leq dim(V)$$

   	\end{حل}
   \مسئله{}
   اگر 
   $A$
    یک ماتریس 
    $m\times n$
    باشد که 
    $rank A=r>0$.
 $U$
   شکل سطری پلکانی ماتریس 
   $A$
   است.نشان دهید یک ماتریس وارون پیذیر مانند 
   $E$
   وجود دارد که 
   $A=EU$.
   با استفاده از این موضوع 
   $A$
    را به صورت حاصل جمع 
    $r$
    ماتریس با رنک 
    $1$
    بنویسید. 
    \begin{حل}
    می دانیم برای اینکه 
    $A$
    را سطری پلکانی کنیم باید یک سری ماتریس مقدماتی را باید در آن  ضرب کنیم به شکل زیر :
    $$E_1E_2E_3\cdots E_nA=U$$
از تمرین سری قبل می دانیم ماتریس های مقدماتی معکوس پذیرند و معکوس آن ها نیز ماتریس مقدماتی است 
پس داریم: 
$$A=E^{-1}_nE^{-1}_{n-1}\cdots E^{-1}_1U$$    
از انجاییکه ضرب چنر ماتریس معکوس پذیر معکوس پذیر است پس 
$$E=E^{-1}_nE^{-1}_{n-1}\cdots E^{-1}_1$$
    	حال برای اثبات قسمت دوم از فصل های قبل می دانیم: 
    	$$AB=\begin{bmatrix}col_1(A)&clo_2(A)&\cdots&col_n(A)   \end{bmatrix}\begin{bmatrix}
    	row_1(A)\\
    	row_2(A)\\
    	\vdots\\
    	row_n(A)
    	\end{bmatrix}=col_1(A)row_1(B)+col_2(A)row_2(B)+\cdots+col_n(A)row_n(B) \qquad \star$$
    	چون 
    	$rank(A)=r$
    	هست پس شکل سطری پلکانی آن دارای 
    	$r$
    	سطر مستقل خطی است و بقیه سر ها صفر هستند.پس با توجه 
    	$\star$،
    	$r$
    	سطر غیر صفر داریم و همچنین 
    	$col_i(A)row_i(B)$
    	یک ماتریس با رنک 1 است زیرا همه سطر های 
    	$col_i(A)row_i(B)$
    	مضربی از 
    	$row_i(B)$
    	هستند که در این صورت فقط یک بردار مستقل خطی داریم و بدین ترتیب حکم ثابت می شود.
    	
    
    \end{حل}

    
    \مسئله {} در هر یک از قسمت های زیر ابتدا مختصات بردار داده شده
    $(v)$ 
    را در هریک از پایه ها بیابید سپس ماتریس انتقال از یک پایه
    $(B)$
    به پایه
    $(C)$
     دیگر را محاسبه کنید.
    \begin{enumerate}
    	\item 
    	$$V=\mathbb{P}_3[x]\qquad v=p(x)=8+x+6x^2+9x^3$$
    	$$B=\{−2+3x+4x^2-x^3, 3x+5x^2+2x^3, -5x^2-5x^3,
    		4 + 4x + 4x^2\}$$
    		$$C=\{1 - x^3, 1 + x, x + x^2, x^2 + x^3\}$$
    	
    	\item 
    	$$V=M_2(\mathbb{R})\qquad v=\begin{bmatrix}
    	-3&-2\\
    	-1&2
    	\end{bmatrix}
    	$$
    	$$B=\{
    	\begin{bmatrix}
    	1&0\\
    	-1&-2
    	\end{bmatrix}
    	\begin{bmatrix}
    	0&-1\\
    	3&0
    	\end{bmatrix}
    	\begin{bmatrix}
    	3&5\\
    	0&0
    	\end{bmatrix}
    	\begin{bmatrix}
    	-2&-4\\
    	0&0
    	 \end{bmatrix}
    	\}$$
    		$$C=\{
    	\begin{bmatrix}
    	1&1\\
    	1&1
    	\end{bmatrix}
    	\begin{bmatrix}
    	1&1\\
    	1&0
    	\end{bmatrix}
    	\begin{bmatrix}
    	1&1\\
    	0&0
    	\end{bmatrix}
    	\begin{bmatrix}
    	1&0\\
    	0&0
    	\end{bmatrix}
    	\}$$
    	 \item
    	 $$V=\mathbb{R}^3\qquad v=(1,7,7)$$
    	 $$B=\{(-7,4,4),(4,2,-1),(-7,5,0)\}$$
    	 $$C={(1,1,0),(0,1,1),(3,-1,-1)}$$
    	 
    	 \begin{حل}
    	 	برای حل این سوال قسمت دوم برای نمونه حل می شود حل دو قسمت دیگر نیز مشابه قسمت دوم می باشد که حل آن ها بر عهده خود شما دانشجویان گذاشته می شود.
    	 	ابتدا مختصات 
    	 	$v$
    	 	را نسبت به پایه های 
    	 	$B$
    	 	و 
    	 	$C$
    	 	می یابیم.
    	 	پس مختصات 
    	 	$v$
    	 	برحسب دو پایه برابراست با:
    	 	$$[v]_B=(-1,-\frac{2}{3},-\frac{4}{3},-1)\qquad \qquad [v]_C=(2,-3,-1,-1)$$
    	 	حال می خواهیم 
    	 	$\underset{C\gets B}{P}$
    	 	برای این کار باید ماتریس زیرا را تشکیل می دهیم:
    	 $$	\left(\begin{array}{cccc|cccc}
    	 	1 & 1 & 1 & 1 & 1 & 0 & 3 & -2 \\ 
    	 	1 & 1 & 1 & 0 & 0 & -1 & 5 & -4 \\
    	 	1 & 1 & 0 & 0 & -1 & 3 & 0 & 0 \\
    	 	1 & 0 & 0 & 0 & -2& 0 & 0 & 0$$ 
    	 	
    	 	\end{array} \right)	$$
     	
     	حال ماتریس سمت چپ به شکل کاهش یافته سطری پلکانی در می اوریم و اعمال سطری پلکانی مشابه را بر روی ماتریس سمت راست نیز اعمال می کنیم در نهایت ماتریس سمت راست همان 
     		$\underset{C\gets B}{P}$
     		 خواهد بود.
    	 
    	 \end{حل}
    \end{enumerate}
   \مسئله{سوال امتیازی}
   
   بازی دو نفره ی زیر را در نظر بگیرید:
   \begin{enumerate}
   	\item 
   	بازی با یک ماتریس ۱۰ در ۱۰ خالی بازی شروع می شود.
   	\item
   	بازیکن اول و دوم به ترتیب اعداد حقیقی دلخواهی در درایه های این ماتریس قرار می دهند 
   	\item
   	بعد از پر شدن ماتریس، بازیکن اول در صورتی برنده است که دترمینان ماتریس نهایی مخالف صفر باشد و بازیکن دوم در صورتی برنده است که دترمینان صفر شود.
   \end{enumerate}
   کدام یک از بازیکننان یک استراتژی ای برای پیروزی دارد؟ در واقع اگر شما در این بازی حق انتخاب اول یا دوم بودن را داشتید کدام را انتخاب می کردید و استراتژی شما برای پیروزی در این نوبت چیست؟
   

   \مسئله{سوال امتیازی}{برای حل این سوال باید از لم زُرن استفاده کنید که لم زُرن یک لم پر کاربرد در زمینه نظریه مجموعه ها است.}
  
   نشان دهید هر فضای برداری غیر صفر یک پایه دارد؟

   
   
\end{document}


