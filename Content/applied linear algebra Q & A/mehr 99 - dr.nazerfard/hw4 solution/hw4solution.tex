\documentclass{article}
\usepackage{enumitem}

\input{Style1}

%\settextfont{Yas}
\begin{document}
	\سربرگ{پاسخ تمرین سری 4}{}{}
	\clearpage
	\مسئله{} یکی از ماتریس های زیر  را به اختیار انتخاب کنید ابتدا چند جمله ای سرشت نما را برای آن بیابید سپس مقدار ویژه و بردار های ویژه را برای آن  مشخص کنید  در نهایت صورت قطری شدن آن را قطری کنید.
	
	$$\begin{bmatrix}
	-1&4&-2\\
	-3&4&0\\
	-3&1&3
	\end{bmatrix} \qquad
	\begin{bmatrix}
	2&2&-1\\
	1&3&-1\\
	-1&-2&2
	\end{bmatrix}
	$$
	
	\مسئله{}
	5 مورد از گزاره های زیر را به اختیار ثابت کنید:
	\begin{enumerate}
		\item 
		نشان دهید اگر 
		$\lambda$
		مقدار ویژه ماتریس واون پذیر 
		$A$
		باشد آنگاه 
		$\lambda^{-1}$
		مقدار ویژه ماتریس 
		$A^{-1}$
		است.
		\begin{حل}
			قسمت 1.5 سوال 25.
		\end{حل}
		\item 
		نشان دهید اگر 
		$A^2=0$
		انگاه تنها  مقدار ویژه 
		$A$
		صفر است.
		\begin{حل}
			قسمت 1.5 سوال 26.
		\end{حل}
		\item 
		$\lambda$
		مقدار ویژه از 
		$A$
		است اگر و فقط اگر 
		مقدار ویژه ای از 
		$A^{T}$
		باشد.
		\begin{حل}
			قسمت 1.5 سوال 27.
		\end{حل}
		\item
		
		نشان دهید 
		$A$
		و 
		$A^{T}$
		جند جمله ای سرشت نمای مشابه ای دارند.
		\begin{حل}
			قسمت 2.5 سوال 20.
		\end{حل}
		\item 
		با توجه به الگوریتم 
		$QR$
		ثابت گنید اگر 
		$A=QR$
		باشد که 
		$Q$
		معکوس پذیر است آنگاه 
		$A$
		با 
		$A_1=RQ$
		متشابه است.
		\begin{حل}
			قسمت 2.5 سوال 23.
		\end{حل}
		\item 
		فرض کنید 
		$A$
		یک ماتریس حقیقی 
		$n\times n$
		باشد که 
		$A^T=A$.
		نشان دهید اگر برای 
		$x$
		های غیر صفری در 
		$\mathbb{C}^n$،
		$Ax=\lambda x$
		باشد آنگاه 
		$\lambda$
		حقیقی است و در واقع قسمت حقیقی 
		$x$
		بردار ویژه 
		$A$
		است.
		\begin{حل}
			قسمت 5.5 سوال 24.
		\end{حل}
		\item 
		برای بردار های 
		$u,v$
		در 
		$\mathbb{R}^n$
		ثابت کنید:
		$$\parallel u+v\parallel ^2 +\parallel u-v \parallel=2\parallel u\parallel^2+2\parallel v\parallel^2$$
		\begin{حل}
			قسمت 1.6 سوال 24.
		\end{حل}
		\item
		اگر 
		$U,V$
		دو ماتریس 
		$n\times n$
		متعامد باشند،نشان دهید 
		$UV$
		نیز یک ماتریس متعامد است.
		\begin{حل}
			قسمت 2.6 سوال 29.
		\end{حل}
	\end{enumerate}
	
	
	\مسئله{}
	فرض کنید 
	$A$
	ماتریس 
	$n\times n$
	باشد که مجموع درایه های تمام سطر های آن
	$s$
	باشد ثابت کنید 
	$s$
	مقدار ویژه ای از 
	$A$
	است.
	\begin{حل}
		قسمت 1.5 سوال 29
	\end{حل}
	
	\مسئله{}
	({\bf سوال امتیازی })
	برای هر اسکالر 
	$a,b,c$
	نشان دهید:
	$$A=\begin{bmatrix}
	b&c&a\\
	c&a&b\\
	a&b&c
	
	\end{bmatrix},
	B=\begin{bmatrix}
	c&a&b\\
	a&b&c\\
	b&c&a
	
	\end{bmatrix}
	,C=
	\begin{bmatrix}
	a&b&c\\
	b&c&a\\
	c&a&b
	\end{bmatrix}
	$$
	همگی متشابهند و اگر
	$BC=CB$
	باشند آنگاه 
	$A$
	دو مقدار ویژه صفر دارد.
	
	
	
	\مسئله{}({\bf سوال امتیازی })
	اگر 
	$$A=
	\begin{bmatrix}
	-1&1&1&-1\\
	1&-1&-1&1\\
	1&-1&-1&1\\
	-1&1&1&-1
	\end{bmatrix}$$
	باشد،
	$A^2,A^6$
	را محاسبه کنید.
	
	
	\مسئله{}فرض کنید
	$\varepsilon=\{e_1,e_2,e_3\}$
	پایه استاندارد برای 
	$\mathbb{R}^3$
	و 
	$\mathcal{B}=\{b_1,b_2,b_3\}$
	پایه ای برای فضای برداری 
	$V$
	باشد و 
	$T:\mathbb{R}^3\longrightarrow V$
	یک تبدیل خطی باشد که:
	$$T(x_1,x_2,x_3)=(x_3-x_1)b_1-(x_1+x_2)b_2+(x_1-x_2)b_3$$
	\begin{enumerate}
		\item 
		$T(e_1),T(e_2)$
		و 
		$T(e_3)$
		را محاسبه کنید.
		\item 
		$[T(e_1)]_{\mathcal{B}},[T(e_2)]_{\mathcal{B}}$
		و 
		$[T(e_3)]_{\mathcal{B}}$
		را محاسبه کنید.
		\item 
		ماتریس تبدیل 
		$T$
		را تحت پایه های 
		$\varepsilon,\mathcal{B}$
		بیابید.
		
		
		
	\end{enumerate}
	\begin{حل}
		قسمت4.5 سوال 3.
	\end{حل}
	
	
	
	\مسئله{}ثابت کنید مجموع درایه های روی قطر اصلی هر ماتریس قطری شدنی برابر است با مجموع مقادیر ویژه آن ماتریس.
	\begin{حل}
		قسمت 4.5 سوال 25و 26.
	\end{حل}
	
	
	
	\مسئله{} یک دیگر از روش هایی  زمانی که تقریبی از بردار ویژه در دسترس باشد می شود با آن مقادیر ویژه را یافت 
	روش خارج قسمت ریلی 
	({\lr rayleigh quotient })
	است. 
	
	مشاهده کردیم اگر 
	$Ax=\lambda x$
	آنگاه 
	$x^TAx=x^T(\lambda x)=\lambda(xTx)$
	و در این صورت خارج قسمت ریلی 
	$$R(x)=\frac{x^TAx}{x^Tx}$$
	ّبرابر
	$\lambda$
	خواهد بود.اگر 
	$x$
	به حد کافی به به یک بردار ویژه 
	$\lambda$
	نزدیک باشد آنگاه این خارج قسمت به 
	$\lambda$
	نزدیک خواهد شد.زمانی که 
	$A$
	متقارن باشد خارج قشمت ریلی 
	$R(x_k)=(x^T_kAx_k)/(x^T_kx_k)$
	با دقتی دو برابر نسبت 
	$\mu_k$
	در روش توانی عمل خواهد کرد این موضوع را برای ماتریس و بردار اولیه زیر نشان دهید:
	$$A=\begin{bmatrix}
	5&2\\
	2&2
	\end{bmatrix},x_0=\begin{bmatrix}
	1\\
	0
	\end{bmatrix}$$
	\begin{حل}
		قسمت 8.5 سوال 11.
	\end{حل}
	
	
	\مسئله{}({\bf سوال امتیازی })
	اگر 
	$W$
	یک زیر فضا از فضای ضرب داخلی 
	$V$
	باشد،
	$u\in V$
	را در نظر بگیرید نشان دهید
	$v\in W$
	تصویری از 
	$u$
	بر روی 
	$W$
	است به طوری که   
	$$u=v+v'\qquad for\  \ some\ \ v' \in W^{\perp}$$
	اگر و فقط اگر 
	$$\parallel u-v\parallel \leq \parallel u-w\parallel\quad,\quad for \ \ every \ \ w\in W $$
	
	
	\مسئله{}({\bf سوال امتیازی }) فرض کنید 
	$W_1,W_2$
	زیر فضایی از فضای ضرب داخلی 
	$V$
	باشد آنگاه نشان دهید:
	\begin{enumerate}
		\item $(W_1+W_2)^{\perp}=W_1^{\perp}\cap W_2^{\perp}$
		\item $(W_1\cap W_2)^{\perp}=W_1^{\perp}+W_2^{\perp}$
	\end{enumerate}
	
	
	
	\مسئله{} فرض کنید 
	$y=\begin{bmatrix}
	7\\
	9
	\end{bmatrix}$
	و 
	$u_1=\begin{bmatrix}
	\frac{1}{\sqrt{10}}\\
	-\frac{3}{\sqrt{10}}
	\end{bmatrix}$
	و 
	$W=span\{u_1\}$.
	
	
	$proj_W \boldsymbol{y},(UU^T)y$
	را حساب کنید.
	\begin{حل}
		قسمت 3.6 سوال 11.
	\end{حل}
	
	\مسئله {} فرض کنید 
	$W$
	زیر فضایی از 
	$\mathbb{R}^n$
	با پایه متعامد 
	$\{w_1,w_2,\cdots,w_p\}$
	و همچنین فرض کنید 
	$\{v_1,v_2,\cdots,v_q\}$
	پایه ای متعامد برای 
	$W^{\perp}$
	باشد.
	\begin{enumerate}
		\item 
		چرا 
		$\{w_1,w_2,\cdots,w_p,v_1,v_2,\cdots,v_q\}$
		یک پایه متعامد است ؟
		\item 
		چرا 
		$span$
		مجموعه قسمت 1
		$\mathbb{R}^n$
		را تولید می کند؟
		\item 
		نشان دهید
		$dim W+dim W^{\perp}=n$.
		
		
	\end{enumerate}
	\begin{حل}
		قسمت 3.6 سوال 24.
	\end{حل}
	
	\مسئله {}یکی از ماتریس های زیر را به اختیار انتخاب و برای فضایی ستونی آن یک پایه متعامد پیدا کنید:
	$$\begin{bmatrix}
	3&-5&1\\
	1&1&1\\
	-1&5&-2\\
	3&-7&8
	\end{bmatrix} \qquad
	\begin{bmatrix}
	1&2&-1\\
	-1&1&-4\\
	-1&4&-3\\
	1&-4&7\\
	1&2&1
	\end{bmatrix}
	$$
	\begin{حل}
		قسمت4.6 سوال 9و12. 
	\end{حل}
	
	
	
	\مسئله{} تمام جواب های کوچکترین مربعات را برای تساوی 
	$Ax=b$
	بیابید.
	$$A= \begin{bmatrix}
	1&1&0\\
	1&1&0\\
	1&0&1\\
	1&0&1
	\end{bmatrix},
	b=\begin{bmatrix}
	1\\
	3\\
	8\\
	2
	\end{bmatrix}
	$$
	\begin{حل}
		قسمت 5.6 سوال 5.
	\end{حل}
	
	\مسئله{}فرض کنید 
	$A$
	یک ماتریس 
	$m\times n$
	باشد که ستون هایش مستقل خطی هستند و
	$b\in \mathbb{R}^n$.
	
	با استفاده  از
	روش نرمال یک فرمول برای 
	$\hat{b}$
	که تصویر 
	$b$
	بر روی 
	$Col \ \ A$
	هست بیابید.
	\begin{حل}
		قسمت 5.6 سوال 23.
	\end{حل}
	
	\مسئله{}نشان دهید اگر 
	$A$
	یک ماتریس
	$n\times n$
	مثبت معین باشد،آنگاه یک ماتریس مثبت معین 
	$n\times n$
	مانند 
	$B$
	وجود دارد که 
	$A=BB^T$.
	
	\begin{حل}
		قسمت 2.7 سوال 25.
	\end{حل}
	\مسئله{} ماتریس 
	$A$
	را در نظر بگیرید،
	$\lambda_1$
	بزرگترین مقدار ویژه آن و
	$u_1$
	بردار ویژه یکه متناظر با 
	$\lambda_1$
	است،ثابت کنید بزرگترین مقدار 
	$x^TAx$
	با توجه به قیود:
	$$x^Tx=0\qquad x^Tu_1=0$$
	برابر 
	$\lambda_2$
	است که 
	$\lambda_2$
	دومین مقدار ویژه بزرگ 
	$A$
	است. همچنین این بزرگترین مقدار زمانی اتفاق می افتد که 
	$x$
	برابر 
	$u_2$
	که بردار ویژه یکه متناظر با 
	$\lambda_2$
	است،باشد.
	
	\مسئله{} تجزیه 
	$SVD$
	ماتریس زیر را به دست آورید.(راهنمایی:
	ماتریس
	$\begin{bmatrix}
	-\frac{1}{3}&\frac{2}{3}&\frac{2}{3}\\
	\frac{2}{3}&-\frac{1}{3}&\frac{2}{3}\\
	\frac{2}{3}&\frac{2}{3}&-\frac{1}{3}
	\end{bmatrix}$
	می تواند به عنوان یک انتخاب برای 
	$U$
	در نظر گرفته شود.
	)
	$$\begin{bmatrix}
	-3&1\\
	6&-2\\
	6&-2
	
	\end{bmatrix}$$
	
	\begin{حل}
		قسمت 4.7 سوال 11.
	\end{حل}
	
	\مسئله{} نشان دهید در یک ماتریس مربعی قدر مطلق دترمینان برابر حاصلضرب مقادیر تکین ماتریس است. 
	\begin{حل}
		$$det(A)=det(U\sum V^T)=det(U)det(\sum)det(V^T)$$
		می دانیم 
		$det(U),det(V^T)$
		برابر 1 یا -1 است و همچنین چون 
		$\sum$
		ماتریس قطری است که بر روی قطر آن مقدار ویژه منفرد هستند پس 
		$$|det(A)|=|det(\sum)|=\prod_{i}^{}\sigma_i$$
	\end{حل}
	
	\\
	\\
	$\star \star$
	{\bf  سوالات زیر برای تمرین بیشتر در نظر گرفته شده است و به آن ها نمره ای تعلق نمی گیرد:}
	
	\مسئله{} فرض کنید:
	$$A=\begin{bmatrix}
	0.5&0.2&0.3\\
	0.3&0.8&0.3\\
	0.2&0&0.4
	\end{bmatrix}
	,v_1=
	\begin{bmatrix}
	0.3\\
	0.6\\
	0.1
	\end{bmatrix}
	,v_2=
	\begin{bmatrix}
	1\\
	-3\\
	2
	\end{bmatrix}
	,
	v_3=
	\begin{bmatrix}
	-1\\
	0\\
	1
	\end{bmatrix}
	,
	w=
	\begin{bmatrix}
	1\\
	1\\
	1
	\end{bmatrix}
	$$
	\begin{enumerate}
		\item 
		نشان دهید 
		$v_1,v_2$
		و 
		$v_3$
		بردار ویژه های 
		$A$
		هستند.
		\item
		فرض کنید 
		$x_0$
		بردار برداری در 
		$\mathbb{R}^3$
		باشد که درایه های آن نامنفی باشند و مجموعشان 1 باشد. ثابت کنید وجود دارد ثابت هایی مثل 
		$c_1,c_2,c_3$
		که 
		$x_0=c_1v_1+c_2v_2+c_3v_3$
		باشد و همچنین 
		$w^{T}x_0$
		را محاسبه کنید و نتیجه بگیرید 
		$c_1=1$.
		
		\item 
		برای 
		$k=1,2,\cdots$
		تعریف می کنیم 
		$x_k=A^kx_0$
		که 
		$x_0$
		در قسمت (2) معرفی شده است .نشان دهید 
		$x_k\to v_1$
		زمانی که 
		$k$
		افزایش می یابد.
	\end{enumerate}
	\begin{حل}
		قسمت 2.5 سوال 27.
	\end{حل}
	
	
	\مسئله{}فرض کنید 
	$A$
	یک ماتریس 
	$n\times n$
	متقارن باشد،فرض 
	$x$
	هر برداری در 
	$\mathbb{C}^n$
	باشد و در نظر بگیرید 
	$q=\bar{x}^TAx$.
	تساوی های زیر نشان می دهند که 
	$q=\bar{q}$.
	هرکدام از تساوی ها را با ادله کافی توجیه کنید.
	$$\bar{q}=\overline{\bar{x}^TAx}=x^T\overline{Ax}=x^TA\bar{x}=(x^TA\bar{x})^T=\bar{x}^TA^Tx=q$$
	
	\begin{حل}
		قسمت 5.5 سوال 23.
	\end{حل}
	
	\مسئله{}
	$u\neq 0$
	را در 
	$\mathbb{R}^n$
	در نظر بگیرید،فرض کنید 
	$L=span\{u\}$
	برای هر 
	$y$
	در 
	$\mathbb{R}^n$
	تصویر 
	$y$
	نسبت به 
	$L$
	را اینگونه تعریف می کنیم:
	$$refl_L \boldsymbol{y}=2.proj_L\boldsymbol{y}-\boldsymbol{y}$$
	نشان دهید 
	$\boldsymbol{y}\mapsto refl_L\boldsymbol{y}$
	یک تبدیل خطی است.
	\begin{حل}
		قسمت 2.6 سوال 34.
	\end{حل}
	
	
	\مسئله{} فرض کنید 
	$A=QR$
	یک تقسیم بندی 
	$QR$
	برای ماتریس 
	$A$ای
	باشد که ستون های آن مستقل خطی هستند.
	$A$
	را به شکل 
	$[A_1 \ \ A_2]$
	می نویسیم که 
	$A_1$
	p ستون دارد.
	چگونه می توان یک تقسیم بندی 
	$QR$
	برای 
	$A_1$
	یافت؟توضیح دهید تقسیم بندی شما چگونه شرایط یک تقسیم بندی 
	$QR$
	را حفظ می کند.
	\begin{حل}
		قسمت 4.6 سوال 23.
	\end{حل}
	
	
\end{document}


