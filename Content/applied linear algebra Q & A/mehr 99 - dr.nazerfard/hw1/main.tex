
\documentclass{article}



\usepackage{graphicx,comment,framed}
\usepackage{roundbox}
\usepackage{fancybox}
\usepackage{tikz}
\usepackage{color}
\usepackage[hidelinks]{hyperref}
\usepackage{framed}
\usepackage{amsthm,amssymb,amsmath}
%\usepackage[colorlinks,linkcolor=blue,citecolor=blue]{hyperref}
\definecolor{shadecolor}{cmyk}{0,0,0,0}
\usepackage{listings}
\usepackage{xepersian}
\usepackage[noend]{algpseudocode}


%--------------------- page settings ----------------------

\settextfont[Scale=1.1]{XB Niloofar}
\setdigitfont[Scale=1.1]{XB Niloofar}
\defpersianfont\sayeh[Scale=1.1]{XB Niloofar}
\addtolength{\textheight}{3.2cm}
\addtolength{\topmargin}{-22mm}
\addtolength{\textwidth}{3cm}
\addtolength{\oddsidemargin}{-1.5cm}


%------------------------ Environments ------------------------------------

\newtheorem{قضیه}{قضیه}
\newtheorem{لم}{لم}
\newtheorem{مشاهده}{مشاهده}
\newtheorem{تعریف}{تعریف}


%-------------------------- Notations ------------------------------------
\renewcommand{\labelitemi}{$\bullet$}
\newcommand{\IR}{\ensuremath{\mathbb{R}}} 
\newcommand{\IZ}{\ensuremath{\mathbb{Z}}} 
\newcommand{\IN}{\ensuremath{\mathbb{N}}} 
\newcommand{\IS}{\ensuremath{\mathbb{S}}} 
\newcommand{\IC}{\ensuremath{\mathbb{C}}} 
\newcommand{\IB}{\ensuremath{\mathbb{B}}} 

\newcommand{\bR}{\mathbb{R}}
\newcommand{\cB}{\mathcal{B}}
\newcommand{\cO}{\mathcal{O}}
\newcommand{\cG}{\mathcal{G}}
\newcommand{\rM}{\mathrm{M}}
\newcommand{\rC}{\mathrm{C}}
\newcommand{\rV}{\mathrm{V}}

\newcommand{\lee}{\leqslant}
\newcommand{\gee}{\geqslant}
\newcommand{\ceil}[1]{{\left\lceil{#1}\right\rceil}}
\newcommand{\floor}[1]{{\left\lfloor{#1}\right\rfloor}}
\newcommand{\prob}[1]{{\mbox{\tt Pr}[#1]}}
\newcommand{\set}[1]{{\{ #1 \}}}
\newcommand{\seq}[1]{{\left< #1 \right>}}
\newcommand{\provided}{\,|\,}
\newcommand{\poly}{\mbox{\rm poly}}
\newcommand{\polylog}{\mbox{\rm \scriptsize polylog}\,}
\newcommand{\comb}[2] {\left(\!\!\begin{array}{c}{#1}\\{#2}\end{array}\!\!\right)}




\newcounter{probcnt}
\newcommand{\مسئله}[1]{\stepcounter{probcnt}{
 	\bf \arabic{probcnt}$\mbox{\bf{.}}$ \ #1}}

\newcommand{\fqed}[1]{\leavevmode\unskip\nobreak\quad\hspace*{\fill}{\ensuremath{#1}}}

\newenvironment{اثبات}
	{\begin{trivlist}\item[\hskip\labelsep{\em اثبات.}]}
	{\fqed{\square}\end{trivlist}}

\newenvironment{حل}
	{\begin{trivlist}\item[\hskip\labelsep{\bf حل.}]}
	{\fqed{\blacktriangleright}\end{trivlist}}

\ifdefined\hidesols
	\newsavebox{\trashcan} % uncomment the following line to hide solutions
	\renewenvironment{حل}{\begin{latin}\begin{lrbox}{\trashcan}}{\end{lrbox}\end{latin}}
\fi


%------------------------- Header -----------------------------

%------------------------- Header -----------------------------

\newcommand{\سربرگ}[3]{
	\parindent=0em
	
	
	\begin{shaded}
		
		\rightline{ 
			\makebox[8em][c]{
				\includegraphics[height=3cm]{aut.png}
		}} \ \
	\\[-3em] 
	\centerline{ مدرس :دکتر ناظر فرد } 
	\\[-6.5em]
	\centerline{\large \bf جبرخطی کاربردی }
	\\[0.05em]
	\centerline {\bf نیمسال دوم 96-97}	
	\\[-6.4em]
		\leftline{ 
		\makebox[8em][c]{
			\includegraphics[height=3.2cm]{ceit.jpg}
	}}
	
	
		

		\hrule height .12em
		
		\normalsize
		\vspace{1mm} #1
		\hfill \small  #3
		\vspace{1mm} 
		\hrule height .1em
		
		\vspace{-0.5em} 
		\hfill {\sayeh\large #2} \hfill
	\end{shaded}
	\begin{large}
	

\bf{توجه!!! :}
\end{large}
\\
\begin{itemize}
\item
سری دوم تمرینات با موضوع جبر ماتریسی و دترمینان را در زیر مشاهده می کنید. 
\item
این سری تمرین شامل 11 سوال نظری  است که سوالات شبیه سازی نیز به زودی در اختیار شما قرار خواهد گرفت.


\item 
پس از حل مسائل آن ها را به صورت یک فایل 
\lr{pdf}
 در قسمت مورد نظر آپلود کنید همچنین تمرینات عملی و شبیه سازی را نیز در یک پوشه قرار دهید و در قسمت در نظر گرفته شده با توجه به اصول ارسال تمارین که در کانال و مودل قرار گرفته است ارسال کنید. 
\item
تمرینات نظری را به شکل:\\
\lr{\[9531000\_T\_Giovanni\ \  van Bronckhorst\_HW2.pdf\]}
\\
و تمرینات عملی و شبیه سازی را به شکل:\\
\lr{\[9531000\_S\_Giovanni\ \  van Bronckhorst\_HW2.pdf\]}
ارسال فرمایید.
\item
\color{red}
مهلت تحویل تمارین ساعت 23:55 روز جمعه 97/2/28 خواهد بود.
\end{itemize}
{\bf تمارین:}
\\
\\
}





\begin{document}
\سربرگ{مجموعه سوالات فصل 1 (معادلات خطی در جبر خطی)}{}{}


\مسئله{} 
در زیر دو دستگاه معادلات مشاهده می کنید،برای این دستگاه ها ابتدا ماتریس افزوده را تشکیل دهید سپس ماتریش افزوده آن ها را به شکل کاهش یافته سطری پلکانی در بیاورید و در مورد تعداد جواب های این دستگاه ها بحث کنید و آن ها را به شکل پارامتریک برداری بیان کنید،در نهایت یک توصیف هندسی از این جواب ها ارائه دهید.
\begin{equation*}
\left\{
\begin{array}{rl}
 x_1+3x_2+x_3&=1\\
-4x_1-9x_2+2x_3&=-1\\
-3x_2-6x_3&=-3
\end{array} \right.\qquad
\left\{
\begin{array}{rl}
 x_1+3x_2-5x_3&=4\\
x_1+4x_2+-8x_3&=7\\
-3x_1-7x_2+9x_3&=-6
\end{array} \right.
\end{equation*}
.



\مسئله{}
در دستگاه معادلات زیر 
$h$
و 
$k$
را به گونه ای انتخاب کنید که :
\begin{enumerate}
\item
معادلات جواب نداشته باشند.
\item 
معادلات جواب یکتا داشته باشند.
\item 
بیش از یک جواب داشته باشند.

\end{enumerate}
به هر قسمت به طور جداگانه پاسخ دهید.
\begin{equation*}
\left\{
\begin{array}{rl}
x_1+hx_2&=2\\
4x_1+8x_2&=k
\end{array} \right.\qquad
\left\{
\begin{array}{rl}
x_1+3x_2&=2\\
3x_1+hx_2&=k
\end{array} \right.
\end{equation*}
\مسئله{}
تمام جواب های ممکن برای 
$x_1,x_2,x_3,x_4,x_5$
از دستگاه معادلات زیر بیابید.

$$\begin{array}{ccccc}
x_5&+&x_2&=&yx_1\\
x_1&+&x_3&=&yx_2\\
x_2&+&x_4&=&yx_3\\
x_3&+&x_5&=&yx_4\\
x_4&+&x_1&=&yx_5
\end{array}$$
$y$
یک پارامتر است.

\مسئله{}
در مورد تعداد جواب های دستگاه معادلات زیر را برای مقادیر مختلف 
$a,b$
مشخص کنید.
$$
\begin{array}{ccccccc}
ax_1&+&bx_2&+&2x_3&=&1\\
ax_1&+&(2b-1)x_2&+&3x_3&=&1\\
ax_1&+&bx_2&+&(b+3)x_3&=&2b-1

\end{array}$$

\مسئله{}
خطوط راست در  صفحه 
$xy$
را در نظر بگیرید نشان دهید سه خط
$$
\begin{array}{cccccccc}
l_1&:&ax&+&by&+&c=&0\\
l_2&:&bx&+&cy&+&a=&0\\
l_3&:&cx&+&ay&+&b=&0
\end{array}$$
در یک نقطه متقاطعند اگر و فقط اگر 
$a+b+c=0$
باشند.

\مسئله{}درستی و نادرستی گزاره های زیر را مشخص کنید در صورت درست بودن آن را ثابت کنید و در صورت نادرست بودن برای آن ها مثال نقض بزنید.

\begin{enumerate}
\item
اگر 
$v_1,v_2,v_3$
مستقل خطی باشند و 
$v_2,v_3,v_4$
وابسته خطی باشند آنگاه 
$v_1$
ترکیب خطی از 
$v_3,v_3$
است و 
$v_4$
ترکیب خطی از 
$v_1,v_2,v_3$
است.
\item 
اگر 
$A=\{v_1,v_2,\ldots,v_n\}$
یک مجموعه از بردار ها عضو 
$\mathbb{R}^n$
 که مستقل خطی باشند و 
 $span(A)=\mathbb{R}^n$
 آنگاه
 $B=\{v_1+v_2,v_2+v_3,\ldots,v_{n-1}+v_n,v_n+v_1\}$
 نیز مجموعه مستقل خطی است که 
  $span(B)=\mathbb{R}^n$
  .
  \item 
  اگر 
  $v_1,v_2,\ldots,v_n$
  بردار هایی مستقل خطی هستند اگر و فقط اگر هیچکدام از 
  $v_i$
  ها را نتوان به شکل ترکیب خطی بقیه بردار ها نوشت.
  \item 
  اگر هر 
  $r-1$
  بردار از مجموعه بردار های 
  $v_1,v_2,\ldots,v_r$
  مستقل خطی باشند آنگاه 
   $v_1,v_2,\ldots,v_r$
   مستقل خطی است.
   \item 
   یک سیستم معادلات خطی کاهش یافته(دستگاه معادلاتی که تعداد معادلات کمتر از متغیر ها باشد) با توجه به نوع ضرایب می تواند  فقط یک جواب داشته باشد یا جواب نداشته باشد.
   \item 
   شکل اکولون (\lr{echelon}) یک ماتریس یکتاست.
   \item 
   اگر 
   $A=\begin{bmatrix}
   a&b\\
   c&d
   \end{bmatrix}$
   و
   $ad-bc\neq0$
   آنگاه 
   $Ax=0$
   فقط جواب بدیهی دارد.
\end{enumerate}
\clearpage
\مسئله {}
مربع های جادویی (\lr {magic squre})یکی از ساختار های جالب ترکیبیاتی در ریاضی هستند که در بخش ها گوناگون ریاضی کاربرد دارند و ارتباطات جالبی بین مربع جادویی و ساختار های گرافی و ... وجود دارد،حتی این ساختار ها در علوم دیگر از جمله مکانیک و کامپیوتر نیز کاربرد دارند و هنوز تعداد زیادی مسئله حل نشده در این زمینه وجود دارد.
مربع جادویی یا وفقی جدولی است 
$n\times n$
که خانه های آن با اعداد مثبت 
$1$
تا 
$n^2$
پر شده است به نحوی که مجموع اعداد هر ستون عمودی و هر سطر افقی و قطر آن عدد ثابتی را نشان می دهد برای مثال شکل زیر یک مربع جادویی 
$3\times 3$
است:
\begin{table}[!hpb]
	\centering
	\begin{tabular}{|c|c|c|}
		\hline
		2&7&6\\
		\hline 
		9&5&1\\
		\hline
		4&3&8\\
		\hline
		
	\end{tabular}		
\end{table}
\\
$\S$
اگر تعداد مربع های جادویی 
$6\times 6$
را بیابید نمره درس جبر خطی کاربردی شما 20 منظور می شود :) 
$\S$


اگر 
$M_i$
یک مارتیس 
$i\times i$
باشد که درایه های آن اعداد متناظر بر روی یک مربع جادویی 
$i\times i$
باشد آنگاه حاصل ضرب های ماتریسی زیر را بیابید:
$$M_1 \times
\begin{bmatrix}
1
\end{bmatrix}
\qquad M_2 \times
\begin{bmatrix}
1\\
1
\end{bmatrix}
\qquad M_3 \times
\begin{bmatrix}
1\\
1\\
1
\end{bmatrix}
\qquad M_n \times
\begin{bmatrix}
	1\\
	1\\
	\vdots\\
	1
\end{bmatrix}_{n\times1}
$$
همچینین تعیین کنید یک ماتریس 
$M_i$
با کدامیک از اعمال سطری پلکانی همچنان به شکل جادویی می ماند.


\مسئله{} 
گزاره های زیر ثابت کنید:
\begin{enumerate}
\item 
	اگر معادله 
	$Ax=b$
	به ازای هر 
	$b\in \mathbb{R}^n$
	جواب داشته باشد و به ازای 
	$b=0$
	فقط جواب بدیهی داشته باشد آنگاه  ماتریس 
	$B$
	 را بدین شکل از ماتریس 
	 $A$ 
	 می سازیم که تمامی ستون های کمتر از 
	 $n$ 
	ام را با ستون 
	 $n$
	 ام جمع می کنیم و در همون ستون 
	 $n $
	 قرار می دهیم ثابت کنید معادله 
	 $Bx=b$
	 به ازای هر 
	 $b\in \mathbb{R}^n$
	 جواب دارد و به ازای 
	 $b=0$
	 فقط جواب بدیهی دارد.
	 \item 
	 نشان دهید اگر معادله 
	 $Ax=0$
	 فقط جواب بدیهی داشته باشد و 
	 $A$
	 به شکل 
	 $[a_1 a_2 \ldots a_n]$
	باشد که 
	 $a_i$
	 ها ستون های ماتریس 
	 $A$
	 هستند آنگاه وجود دارد عدد صحیح مانند 
	 $k$
	 ای که 
	 $1<k\le n$
	 و 
	 $Bx=a_k$
	 سازگار باشد
	 که 
	 $B=[a_1,a_2,\ldots,a_{n-1}]$
	 .
	 \item 
	 فرض کنید 
	 $w$
	حوابی از 
	$Ax=b$ 
	باشد و تعریف می کنیم 
	$v_h=w-p$
	.نشان دهید 
	$v_h$
	جوابی از 
	$Ax=0$
	است.این نشان می دهد که هر جوابی از 
	$Ax=b$
	به شکل 
	$w=p+v_h$
	است که 
	$p$
	یک جواب خاص از 
	$Ax=b$
	 است و 
	 $v_h$
	 جوابی از
	 $Ax=0$
	 .
	 
\end{enumerate}

\مسئله{}
$u,v$
 را دو بردار مستقل خطی عضو 
 $\mathbb{R}^3$
 در نظر بگیرید و 
 $P$
 را  صفحه ای در نظر بگیرید که از این دو بردار و نقطه 
 $0$
 می گذرد. نمایش پارامتریک 
 $P$
 به شکل 
 $x=su+tv(s,t \in \mathbb{R})$
 است .نشان دهید که یک تبدیل خطی 
 $T:\mathbb{R}^3\longrightarrow \mathbb{R}^3$
 صفحه 
 $P$
 را به صفحه ای که از 
 $0$
 می گذرد یا به خطی که از 
 $0$
 می گذرد و یا به مبدا مختصات در 
 $\mathbb{R}^3$
 نگاشت می کند و همچنین چه چیزی باید در مورد 
 $T(u),T(v)$
 صدق کند که تصویر صفحه 
 $P$
 یک صفحه باشد.
 
 \مسئله{}
 فرض کنید که 
 $span\{v_1,v_2,\ldots,v_p\}=\mathbb{R}^n$
 و 
 $T:\mathbb{R}^n\longrightarrow \mathbb{R}^n$
یک ترکیب خطی باشد که 
$$\forall i \in\{1,\ldots,p \}\ \ T(v_i)=0 $$
آنگاه نشان دهید که
$T$
 یک تبدیل صفر است.(به تبدیلی تبدیل صفر گویند که 
$\forall x\in \mathbb{R}^n \ \ T(x)=0$)


\مسئله{}
فرض کنید 
$T:\mathbb{R}^n\longrightarrow \mathbb{R}^n$
یک تبدیل خطی باشد نشان دهید اگر 
$T$	
دو بردار مستقل خطی را به یک مجموعه وابسته خطی نگاشت کند آنگاه 
$T(x)=0$
جواب غیر بدیهی دارد.

\مسئله{}
در هر کدام از تبدیل های زیر مشخص کنید تبدیل خطی هست یا نه و در صورت خطی بودن ماتریس استاندارد آن را مشخص کنید.
\begin{enumerate}
	\item
	{\setlength\arraycolsep{0.1em}
	\begin{eqnarray*}
	T:\mathbb{R}^2&\longrightarrow&\mathbb{R}^2\\
	(x_1,x_2)&\longrightarrow&(4x_1-2x_2,3|x_2|)
	\end{eqnarray*}}
\item 
	{\setlength\arraycolsep{0.1em}
	\begin{eqnarray*}
		T:\mathbb{R}^2&\longrightarrow&\mathbb{R}^2\\
		(x_1,x_2)&\longrightarrow&(sin(x_1),x_2)
\end{eqnarray*}}
\item
{\setlength\arraycolsep{0.1em}
	\begin{eqnarray*}
		T:\mathbb{R}^3&\longrightarrow&\mathbb{R}^3\\
		(x_1,x_2,x_3)&\longrightarrow&(3x_1,x_1-x_2,2x_1+x_2+x_3)
\end{eqnarray*}}

\end{enumerate}	
\end{document}