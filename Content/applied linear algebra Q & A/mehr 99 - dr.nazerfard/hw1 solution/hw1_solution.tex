
\documentclass{article}



\usepackage{graphicx,comment,framed}
\usepackage{roundbox}
\usepackage{fancybox}
\usepackage{tikz}
\usepackage{color}
\usepackage[hidelinks]{hyperref}
\usepackage{framed}
\usepackage{amsthm,amssymb,amsmath}
%\usepackage[colorlinks,linkcolor=blue,citecolor=blue]{hyperref}
\definecolor{shadecolor}{cmyk}{0,0,0,0}
\usepackage{listings}
\usepackage{xepersian}
\usepackage[noend]{algpseudocode}


%--------------------- page settings ----------------------

\settextfont[Scale=1.1]{XB Niloofar}
\setdigitfont[Scale=1.1]{XB Niloofar}
\defpersianfont\sayeh[Scale=1.1]{XB Niloofar}
\addtolength{\textheight}{3.2cm}
\addtolength{\topmargin}{-22mm}
\addtolength{\textwidth}{3cm}
\addtolength{\oddsidemargin}{-1.5cm}


%------------------------ Environments ------------------------------------

\newtheorem{قضیه}{قضیه}
\newtheorem{لم}{لم}
\newtheorem{مشاهده}{مشاهده}
\newtheorem{تعریف}{تعریف}


%-------------------------- Notations ------------------------------------
\renewcommand{\labelitemi}{$\bullet$}
\newcommand{\IR}{\ensuremath{\mathbb{R}}} 
\newcommand{\IZ}{\ensuremath{\mathbb{Z}}} 
\newcommand{\IN}{\ensuremath{\mathbb{N}}} 
\newcommand{\IS}{\ensuremath{\mathbb{S}}} 
\newcommand{\IC}{\ensuremath{\mathbb{C}}} 
\newcommand{\IB}{\ensuremath{\mathbb{B}}} 

\newcommand{\bR}{\mathbb{R}}
\newcommand{\cB}{\mathcal{B}}
\newcommand{\cO}{\mathcal{O}}
\newcommand{\cG}{\mathcal{G}}
\newcommand{\rM}{\mathrm{M}}
\newcommand{\rC}{\mathrm{C}}
\newcommand{\rV}{\mathrm{V}}

\newcommand{\lee}{\leqslant}
\newcommand{\gee}{\geqslant}
\newcommand{\ceil}[1]{{\left\lceil{#1}\right\rceil}}
\newcommand{\floor}[1]{{\left\lfloor{#1}\right\rfloor}}
\newcommand{\prob}[1]{{\mbox{\tt Pr}[#1]}}
\newcommand{\set}[1]{{\{ #1 \}}}
\newcommand{\seq}[1]{{\left< #1 \right>}}
\newcommand{\provided}{\,|\,}
\newcommand{\poly}{\mbox{\rm poly}}
\newcommand{\polylog}{\mbox{\rm \scriptsize polylog}\,}
\newcommand{\comb}[2] {\left(\!\!\begin{array}{c}{#1}\\{#2}\end{array}\!\!\right)}




\newcounter{probcnt}
\newcommand{\مسئله}[1]{\stepcounter{probcnt}{
 	\bf \arabic{probcnt}$\mbox{\bf{.}}$ \ #1}}

\newcommand{\fqed}[1]{\leavevmode\unskip\nobreak\quad\hspace*{\fill}{\ensuremath{#1}}}

\newenvironment{اثبات}
	{\begin{trivlist}\item[\hskip\labelsep{\em اثبات.}]}
	{\fqed{\square}\end{trivlist}}

\newenvironment{حل}
	{\begin{trivlist}\item[\hskip\labelsep{\bf حل.}]}
	{\fqed{\blacktriangleright}\end{trivlist}}

\ifdefined\hidesols
	\newsavebox{\trashcan} % uncomment the following line to hide solutions
	\renewenvironment{حل}{\begin{latin}\begin{lrbox}{\trashcan}}{\end{lrbox}\end{latin}}
\fi


%------------------------- Header -----------------------------

%------------------------- Header -----------------------------

\newcommand{\سربرگ}[3]{
	\parindent=0em
	
	
	\begin{shaded}
		
		\rightline{ 
			\makebox[8em][c]{
				\includegraphics[height=3cm]{aut.png}
		}} \ \
	\\[-3em] 
	\centerline{ مدرس :دکتر ناظر فرد } 
	\\[-6.5em]
	\centerline{\large \bf جبرخطی کاربردی }
	\\[0.05em]
	\centerline {\bf نیمسال دوم 96-97}	
	\\[-6.4em]
		\leftline{ 
		\makebox[8em][c]{
			\includegraphics[height=3.2cm]{ceit.jpg}
	}}
	
	
		

		\hrule height .12em
		
		\normalsize
		\vspace{1mm} #1
		\hfill \small  #3
		\vspace{1mm} 
		\hrule height .1em
		
		\vspace{-0.5em} 
		\hfill {\sayeh\large #2} \hfill
	\end{shaded}
	\begin{large}
	

\bf{توجه!!! :}
\end{large}
\\
\begin{itemize}
\item
سری دوم تمرینات با موضوع جبر ماتریسی و دترمینان را در زیر مشاهده می کنید. 
\item
این سری تمرین شامل 11 سوال نظری  است که سوالات شبیه سازی نیز به زودی در اختیار شما قرار خواهد گرفت.


\item 
پس از حل مسائل آن ها را به صورت یک فایل 
\lr{pdf}
 در قسمت مورد نظر آپلود کنید همچنین تمرینات عملی و شبیه سازی را نیز در یک پوشه قرار دهید و در قسمت در نظر گرفته شده با توجه به اصول ارسال تمارین که در کانال و مودل قرار گرفته است ارسال کنید. 
\item
تمرینات نظری را به شکل:\\
\lr{\[9531000\_T\_Giovanni\ \  van Bronckhorst\_HW2.pdf\]}
\\
و تمرینات عملی و شبیه سازی را به شکل:\\
\lr{\[9531000\_S\_Giovanni\ \  van Bronckhorst\_HW2.pdf\]}
ارسال فرمایید.
\item
\color{red}
مهلت تحویل تمارین ساعت 23:55 روز جمعه 97/2/28 خواهد بود.
\end{itemize}
{\bf تمارین:}
\\
\\
}





\begin{document}
\سربرگ{پاسخ سوالات سری اول }{}{}


\مسئله{} 
در زیر دو دستگاه معادلات مشاهده می کنید،برای این دستگاه ها ابتدا ماتریس افزوده را تشکیل دهید سپس ماتریس افزوده آن ها را به شکل کاهش یافته سطری پلکانی در بیاورید و در مورد تعداد جواب های این دستگاه ها بحث کنید و آن ها را به شکل پارامتریک برداری بیان کنید،در نهایت یک توصیف هندسی از این جواب ها ارائه دهید.
\begin{equation*}
\left\{
\begin{array}{rl}
 x_1+3x_2+x_3&=1\\
-4x_1-9x_2+2x_3&=-1\\
-3x_2-6x_3&=-3
\end{array} \right.\qquad
\left\{
\begin{array}{rl}
 x_1+3x_2-5x_3&=4\\
x_1+4x_2+-8x_3&=7\\
-3x_1-7x_2+9x_3&=-6
\end{array} \right.
\end{equation*}
\begin{حل}
	ابتدا ماتریس افزوده را برای هر دو دستگاه معادلات تشکیل می دهیم سپس با اعمال سطری آن ها را به شکل کاهش یافته سطری پلکانی در می آوریم و درنهایت جواب دستگاه به شکل پارامتری تعیین می کنیم و تعداد جواب ها را ذکر می کنیم:
	\begin{itemize}
		\item
$$	\begin{bmatrix}
		1&3&1&1\\
		-4&-9&2&-1\\
		0&-3&-6&-3
	\end{bmatrix}\xrightarrow{R_2+4R_1\to R_2}
	\begin{bmatrix}
		1&3&1&1\\
	0&3&6&3\\
	0&-3&-6&-3
	\end{bmatrix}\xrightarrow[R_2+R_3\to R_3]{R_1-R_2\to R_1}	
	\begin{bmatrix}
	1&0&-5&-2\\
	0&3&6&3\\
	0&0&0&0
	\end{bmatrix}
	$$
	$$\xrightarrow{\frac{1}{3}R_2\to R_2}
	\begin{bmatrix}
	1&0&0&-2\\
	0&1&2&1\\
	0&0&0&0
	\end{bmatrix}
	$$
	این دستگاه بی شمار دارد جواب آن به شکل پارامتری به صورت زیر است:
	$$x=\begin{bmatrix}
	x_1\\
	x_2\\
	x_3
	\end{bmatrix}
	={\begin{bmatrix}
		-2\\
		1-2x_3\\
		x_3
		\end{bmatrix}}
	=
		\underbrace{\begin{bmatrix}
		-2\\
		1\\
		0
		\end{bmatrix}}_\text{\lr{p}}+
	\underbrace{\begin{bmatrix}
		0\\
		-2\\
		1
		\end{bmatrix}}_\text{\lr{v}}x_3$$
	\item 
	$$\begin{bmatrix}
		1&3&-4&4\\
		1&4&-8&7\\
		-3&-7&9&-6
		\end{bmatrix}\xrightarrow[R_2-R_1\to R_2]{R_3+3R_1\to R_3}
		\begin{bmatrix}
		1&3&-4&4\\
		0&1&-3&3\\
		0&2&-6&6
		\end{bmatrix}\xrightarrow[R_1-3R_2\to R_1]{R_3-2R_2\to R_3}
		\begin{bmatrix}
		1&0&4&-5\\
		0&4&-3&3\\
		0&0&0&0
		\end{bmatrix}$$
		$$\left\{
		\begin{array}{rl}
		x_1+4x_3&=-5\\
		4x_2-3x_3&=3\\
		x_3 \ \ is\ \ free
		\end{array} \right.$$
		جواب دستگاه به صورت پارامتریک به شکل زیر است:
		$$x=\begin{bmatrix}
		x_1\\
		x_2\\
		x_3
			\end{bmatrix}=
	\begin{bmatrix}
			-4x_3-5\\
			3x_3+3\\
			x_3
			\end{bmatrix}	=
				\underbrace{\begin{bmatrix}
			-5\\
			3\\
			0\\
			\end{bmatrix}}_\text{\lr{p}}+
			\underbrace{\begin{bmatrix}
			-4\\
			3\\
			1
			\end{bmatrix}}_\text{\lr{v}}x_3$$
	\end{itemize}
\end{حل}


\مسئله{}
در دستگاه معادلات زیر 
$h$
و 
$k$
را به گونه ای انتخاب کنید که :
\begin{enumerate}
\item
معادلات جواب نداشته باشند.
\item 
معادلات جواب یکتا داشته باشند.
\item 
بیش از یک جواب داشته باشند.

\end{enumerate}
به هر قسمت به طور جداگانه پاسخ دهید.
\begin{equation*}
\left\{
\begin{array}{rl}
x_1+hx_2&=2\\
4x_1+8x_2&=k
\end{array} \right.\qquad
\left\{
\begin{array}{rl}
x_1+3x_2&=2\\
3x_1+hx_2&=k
\end{array} \right.
\end{equation*}
\begin{حل}
	\begin{itemize}
		\item
	$$	\begin{bmatrix}
		1&h&2\\
		4&8&k	
		\end{bmatrix}\xrightarrow[R_2-4R_1\to R_2]{R_1-\frac{h}{8-4h}R_2\to R_2}
		\begin{bmatrix}
		1&0&2-\frac{(k-8)h}{8-4h}\\
		0&8-4h&k-8	
		\end{bmatrix}$$
		$$\left\{
		\begin{array}{rl}
		h=2,k=8 &\text{بی نهایت جواب}\\
		h\neq2 & \text{یک جواب} \\
		h=2,k\neq8 &\text{بدون جواب} 
		\end{array} \right.$$
		
		\item
		$$	\begin{bmatrix}
		1&3&2\\
		3&h&k	
		\end{bmatrix}\xrightarrow[R_2-3R_1\to R_2]{R_1-\frac{3}{h-9}R_2\to R_2}
		\begin{bmatrix}
		1&0&2-3\frac{k-6}{h-9}\\
		0&1&\frac{k-6}{h-9}	
		\end{bmatrix}$$
		$$\left\{
		\begin{array}{rl}
		h=9,k=6 &\text{بی نهایت جواب}\\
		h\neq 9 & \text{یک جواب} \\
		h=9,k\neq6 &\text{بدون جواب} 
		\end{array} \right.$$
	\end{itemize}
\end{حل}
\مسئله{}
تمام جواب های ممکن برای 
$x_1,x_2,x_3,x_4,x_5$
از دستگاه معادلات زیر بیابید.

$$\begin{array}{ccccc}
x_5&+&x_2&=&yx_1\\
x_1&+&x_3&=&yx_2\\
x_2&+&x_4&=&yx_3\\
x_3&+&x_5&=&yx_4\\
x_4&+&x_1&=&yx_5
\end{array}$$
$y$
یک پارامتر است.
\begin{حل}
	دو طرف تساوی ها را با هم جمع می کنیم جواب به شکل زیر در می آید:
	$$2(x_1+x_2+x_3+x_4+x_5)=y(x_1+x_2+x_3+x_4+x_5)$$
	اگر 
	$x_1+x_2+x_3+x_4+x_5\neq 0$
	باشد آنگاه به ازای 
	$y=2$
	دستگاه یک جواب دارد
	وجواب آن به شکل 
	$$x_1=x_2=x_3=x_4=x_5=c$$
	که 
	$c$
	هر عددی می تواند باشد(این موضوع از حل معادله به ازای 
	$y=2$
	به دست می آید.)
	 و اگر 
	$y\neq 2$
	باشد آنگاه با حذف 
	$x_5,x_4$
	و 
	$x_3$
(به روش جایگذاری مقادیر معادل یک مغیر در معادله)به معادلات: 
$$(y^2+y-1)(x_2-x_1)=0\qquad (y^2+y-1)[x_2-(y-1)x_1]=0$$
می رسیم،اگر 
$y\neq2$
و 
$y^2+y-1\neq0$
آنگاه :
$$x_1=x_2=x_3=x_4=x_5=0$$
اگر 
$y\neq2$
و 
$y^2+y-1=0$
پاسخ مسئله به شکل :
$$\begin{array}{ccccccc}
	x_1&=&s\\
	x_2&=&t\\
	x_3&=&kt-s\\
	x_4&=&(K^2-1)t-ks\\
	x_5&=&ks-t
	
\end{array}$$
که 
$s,t$
مقادیری دلخواه 
و 
$k$
جواب معادله 
$y^2+y-1=0$
است.
	
\end{حل}	
\مسئله{}
در مورد تعداد جواب های دستگاه معادلات زیر را برای مقادیر مختلف 
$a,b$
مشخص کنید.
$$
\begin{array}{ccccccc}
ax_1&+&bx_2&+&2x_3&=&1\\
ax_1&+&(2b-1)x_2&+&3x_3&=&1\\
ax_1&+&bx_2&+&(b+3)x_3&=&2b-1

\end{array}$$
\begin{حل}
	ابتدا ماتریس افزوده را تشکیل می دهیم سپس ماتریس را به شکل کاهش یافته سطری پلکانی در می آوریم و جواب های معادله را می یابیم.
$$	\begin{bmatrix}
	a&b&2&1\\
	a&2b-1&3&1\\
	a&b&b+3&2b-1	
	\end{bmatrix}\xrightarrow[R_2-R_1\to R_2]{R_3-R_1\to R_3}
	\begin{bmatrix}
	a&b&2&1\\
	0&b-1&1&0\\
	0&0&b+1&2b-2	
	\end{bmatrix}\xrightarrow{R_1-\frac{b}{b-1}R_2\to R_2}
	\begin{bmatrix}
	a&0&\frac{b-2}{b-1}&1\\
	0&b-1&1&0\\
	0&0&b+1&2b-2	
	\end{bmatrix}
	$$
	$$\xrightarrow[R_2-\frac{R_3}{b+1}\to R_2]{R_1+\frac{2}{b^2-1}R_3\to R_1}
	\begin{bmatrix}
	a&0&0&\frac{(5-b)(b-1)}{b^2-1}\\
	0&b-1&0&\frac{2(1-b)}{b+1}\\
	0&0&b+1&2b-2	
	\end{bmatrix}\xrightarrow[R_1/a\to R_1]{R_2/b-1 \to R_2,R_3/b+1\to R_3}
	\begin{bmatrix}
	1&0&0&\frac{(5-b)(b-1)}{(b^2-1)a}\\
	0&1&0&\frac{2(1-b)}{b^2-1}\\
	0&0&1&\frac{2(b-1)}{b+1}	
	\end{bmatrix}
	$$
	$$\left\{
	\begin{array}{rl}
		b=1 &\text{بی شمار جواب} \\
		b=5,a=0&\text{بی شمار جواب}\\
		b=5,a\neq0 &\text{یک جواب}\\
		b=-1&\text{بدون جواب} \\
		b\neq\pm1,5,a\neq0&\text{یک جواب}\\
		b\neq1,5,a=0&\text{بدون جواب} 
	\end{array} \right.$$
	
\end{حل}

\مسئله{}
خطوط راست در  صفحه 
$xy$
را در نظر بگیرید نشان دهید سه خط
$$
\begin{array}{cccccccc}
l_1&:&ax&+&by&+&c=&0\\
l_2&:&bx&+&cy&+&a=&0\\
l_3&:&cx&+&ay&+&b=&0
\end{array}$$
در یک نقطه متقاطعند اگر و فقط اگر 
$a+b+c=0$
باشند.


\begin{حل}
	خطوط را به شکل یک سر معادله در نظر می گیریم و دستگاه معادلات را برای آن ها تشکیل می دهیم :
	$$
\begin{bmatrix}
	a&b&-c\\
	b&c&-a\\
	c&a&-b
\end{bmatrix}\xrightarrow[R_3-cR_1]{\frac{1}{a}R_1,R_2-bR_1}
\begin{bmatrix}
1&\frac{b}{a}&-\frac{c}{a}\\
0&c-\frac{b^2}{a}&\frac{bc}{a}-a\\
0&a-\frac{bc}{a}&\frac{c^2}{a}-b
\end{bmatrix}\xrightarrow{\frac{1}{c-\frac{b^2}{a}}R_2}
\begin{bmatrix}
1&\frac{b}{a}&-\frac{c}{a}\\
0&1&\frac{\frac{bc}{a}-a}{c-\frac{b^2}{a}}\\
0&a-\frac{bc}{a}&\frac{c^2}{a}-b
\end{bmatrix}
$$
$$\xrightarrow[R1-\frac{b}{a}\to R_1]{R_3-a-\frac{bc}{a}R_2}
\begin{bmatrix}
1&0&-\frac{c}{a}-\frac{\frac{bc}{a}-a}{c-\frac{b^2}{a}}\cdot\frac{b}{a}\\
0&1&\frac{\frac{bc}{a}-a}{c-\frac{b^2}{a}}\\
0&0&\frac{c^2}{a}-b-\frac{\frac{bc}{a}-a}{c-\frac{b^2}{a}}\cdot a-\frac{bc}{a}
\end{bmatrix}
$$
برای اینکه این دستگاه یک جواب داشته باشد باید درایه 
$3,3$
مساوی 
$0$
شود تا سطر آخر مساوی صفر شود در غیر اینصورت دستگاه جواب ندارد.پس داریم:
$$\frac{c^2}{a}-b-\frac{\frac{bc}{a}-a}{c-\frac{b^2}{a}}\cdot a-\frac{bc}{a}=\frac{c^2-ab}{a}-\frac{\frac{bc-a^2}{a}}{\frac{ac-b^2}{a}}\cdot \frac{a^2-bc}{a}=\frac{(ac-b^2)(c^2-ab)+(a^2-bc)^2}{a(ac-b^2)}=0$$
$$\longrightarrow (ac-b^2)(c^2-ab)+(a^2-bc)^2=ac^3+ab^3+a^4-3a^2bc=0 $$
$$\longrightarrow a(a^3+b^3+c^3-3abc)=a((a+b+c)(a^2+b^2+c^2-ab-bc-ac))=0$$
$$\longrightarrow a+b+c=0$$
همانطور که مشاهده کردید برای اینکه این دستگاه یک جواب داشته باشد باید 
$a+b+c=0$
باشد و از سوی دیگر اگر 
$a+b+c=0$
باشد آنگاه درایه 
$3,3$
صفر خواهد شد و در نتیجه دستگاه یک جواب خواهد داشت.
\end{حل}
\مسئله{}درستی و نادرستی گزاره های زیر را مشخص کنید در صورت درست بودن آن را ثابت کنید و در صورت نادرست بودن برای آن ها مثال نقض بزنید.

\begin{enumerate}
\item
اگر 
$v_1,v_2,v_3$
وابسته خطی باشند و 
$v_2,v_3,v_4$
وابسته خطی باشند آنگاه 
$v_1$
ترکیب خطی از 
$v_2,v_3$
است و 
$v_4$
ترکیب خطی از 
$v_1,v_2,v_3$
است.
\begin{حل}
	لازم است در حل تمامی مسائل بعد از این قسمت این نماد گذاری معرفی شود:
	$$\begin{bmatrix}
	v_1\\
	v_2\\
	\vdots\\
	v_n
		\end{bmatrix}=(v_1,v_2,\cdots,v_n)$$
	این گزاره نادرست است زیرا فرض کنید:
	$$v_1=(1,0,0),v_2=(0,2,0),v_3=(0,4,0)$$
	در این صورت 
	$v_1$
	ترکیب خطی از 
	$v_2,v_3$
	نیست.و همچینین فرض کنید 
	$v_4=(0,0,1)$
	در این صورت حکم دوم نیز برقرار نیست.
\end{حل}
\item 
اگر 
$A=\{v_1,v_2,\ldots,v_n\}$
یک مجموعه از بردار ها عضو 
$\mathbb{R}^n$
 که مستقل خطی باشند و 
 $span(A)=\mathbb{R}^n$
 آنگاه
 $B=\{v_1+v_2,v_2+v_3,\ldots,v_{n-1}+v_n,v_n+v_1\}$
 نیز مجموعه مستقل خطی است که 
  $span(B)=\mathbb{R}^n$
  .
  \begin{حل}
  	این گزاره به طور کلی نادرست است اما برای 
  	$n$
  	 هایی که زوج باشند درست است و برای بررسی بیشتر درست بودن گزاره برای 
  	 $n$
  	 های زوج را ثابت می کنیم،ابتدا ثابت می کنیم 
  	$B$
  	مستقل خطی است پس باید نشان دهیم اگر:
  	$$\beta_1(v_1+v_2)+\beta_2(v_2+v_3)+\cdots+\beta_n(v_n+v_1)=0$$
  	آنگاه
  	$\forall i \ \ \beta_i=0$
  	حال می توانیم عبارت بالا را به شکل زیر بنویسیم:
  	$$(\beta_1+\beta_n)v_1+(\beta_1+\beta_2)v_2+\cdot+(\beta_{n-1}+\beta_n)v_n=0$$
  	چون 
  	$v_i$
  	ها مستقل خطی هستند پس باید:
  	$$\forall\ \ i<n\ \ \beta_i+\beta_{i+1}=0,\beta_1+\beta_n=0 $$
  	از گزاره بالا نتیجه می شود
  	$\beta_i=-\beta_i$
  	که در نتیجه 
  	$\beta_i=0$.
  	\\
  	برای اثبات اینکه 
  	$span(B)=\mathbb{R}^n$
  	می دانیم 
  	$span(A)=\mathbb{R}^n$
  	پس :
  	$$\forall \ \ v\in\mathbb{R}^n\ \  \exists\alpha_1+\cdots+\alpha_n \ \ v=\alpha_1v_1,\cdots\alpha_nv_n$$
  	حال 
  	$v$
  	را بر حسب بردار های 
  	$B$
  	اینگونه می نویسیم ،ابتدا فرض می کنیم 
  	$u_i=v_i+v_{i+1},u_n=v_n+v_1$
  	همچنین در نظر می گیریم:
  	$$r=v_1=\frac{(u_1-(u_2-(\cdots-(u_n))))}{2}$$
  	پس می توانیم بنویسیم :
  	$$v=\alpha_1(r)+\alpha_2(u_2-r)+\alpha_3(u_3-(u_2-r))+\cdots+\alpha_n(u_n-(u_{n-1}(\cdots-(u_2-r))))$$
  	که این نمایش در واقع نمایشی از 
  	$v$
  	برحسب 
  	$B$
  	است.
  	\end{حل}
  \item 
  $v_1,v_2,\ldots,v_n$
  بردار هایی مستقل خطی هستند اگر و فقط اگر هیچکدام از 
  $v_i$
  ها را نتوان به شکل ترکیب خطی بقیه بردار ها نوشت.
 \begin{حل}
 	این گزاره درست است  ،به برهان خلف فرض کنیم یکی از بردار ها را توانسته ایم به شکل ترکیب خطی بقیه بردار ها نوشت در واقع :
 	$$\exists i,\exists \alpha_j\neq0\ \ j\neq i \ \ v_i=\alpha_1v_1+\cdots+v_{i-1}+v_{i+1}+\cdots+\alpha_nv_n$$
 	$$\longrightarrow \alpha_1v_1+\cdots+v_{i-1}+v_i+v_{i+1}+\cdots+\alpha_nv_n=0$$
 	این خلاف فرض است که 
 	$v_i$
 	ها مستقل خطی باشند.
 	برای عکس گزاره نیز فرض می کنیم 
 	$v_i$
 ها مشتقل خطی نباشند آنگاه وجود دارد
 	$v_i$
 	که صورت ترکیب خطی بقیه بردار ها نوشت و این خلاف فرض و دراین صورت نیز فرض خلف باطل و حکم درست است.
 \end{حل}
  \item 
  اگر هر 
  $r-1$
  بردار از مجموعه بردار های 
  $v_1,v_2,\ldots,v_r$
  مستقل خطی باشند آنگاه 
   $v_1,v_2,\ldots,v_r$
   مستقل خطی است.
   \begin{حل}
   		این گزاره نادرست است ،سه بردار 
   	$$v_1=(1,2,3),v_2=(1,2,4),v_3=(3,4,7)$$
   	هر دو بردار از این بردار ها مستقل خطی هستند اما هرسه بردار وابسته خطی هستند.
   %	این گزاره نیز درست است ،برای اثبات به برهان خلف فرض می کنیم 
   	%$v_i$
   	%ها مستقل خطی نباشند آنگاه :
   	%$$\exists\ \ \alpha_i\neq 0 \alpha_1v_1+\cdots+\alpha_rv_r=0$$
   	%$$\to v_i=\frac{\alpha_1v_1+\cdots+\alpha_{i-1}v_{i-1}+\alpha_{i+1}v_{i+1}+\cdots+\alpha_rv_r}{\alpha_i}=\beta_1v_1+\cdots+\beta_{i-1}v_{i-1}+\beta_{i+1}v_{i+1}+\cdots+\beta_rv_r$$
   	%از فرض مسئله می دانیم به ازای هر 
   	%$r_1$
   	%تا از بردار ها مستقل خطی هستند پس:
	%$$if \ \  \alpha_1v_1+\cdots+\alpha_{j-1}v_{j-1}+\alpha_{j+1}v_{j+1}+\alpha_rv_r=0\longrightarrow \forall\ \ j v_j=0 \ \ j\neq i$$
	%حال 
	%$v_i$
	%را براساس تساوی بالا جایگذاری می کنیم :
	%$$(\alpha_1+\beta_1)v_1+\cdots+(\alpha_{j-1}+\beta_{j-1})v_{j-1}+\beta_jv_j+(\alpha_{j+1}+\beta_{j+1})v_{j+1}+\cdots+$$$$(\alpha_{i-1}+\beta_{i-1})v_{i-1}+\alpha_iv_i+(\alpha_{i+1}+\beta_{i+1})v_{i+1}+\cdots+\alpha_rv_r=0$$
	%حال 
	%$r-1$
	%تا از بردار ها را به شکل ترکیب خطی نوشتیم که حاصل آن 
	%$0$
	%است اما ضرایب آن ها صفر نیست(چرا؟)،و خلاف فرض مستقل خطی بودن هر 
	%$r-1$
	%بردار است ،پس فرض خلف باطل و حکم درست است.
   	\end{حل}
   \item 
   یک سیستم معادلات خطی کاهش یافته(دستگاه معادلاتی که تعداد معادلات کمتر از متغیر ها باشد) با توجه به نوع ضرایب می تواند  فقط یک جواب داشته باشد یا جواب نداشته باشد.
   \begin{حل}
   	این گزاره نادرست است و دستگاه زیر را به عنوان نمونه در نظر بگیرید که تعداد جواب های بی شمار است.
   	$$\begin{bmatrix}
   	1&6&2&-5&-2&-4\\
   	0&0&2&-8&-1&3\\
   	0&0&0&0&1&7
   	\end{bmatrix}$$
   	حل دستگاه و نتیجه گیری برعهده خودتان!!!
   	\end{حل}
   \item 
   شکل اکولون (\lr{echelon}) یک ماتریس یکتاست.
   \begin{حل}
   	این گزاره نادرست است زیرا فقط شکل اکولون کاهش یافته یکتاست،مثال نقض:
   	$$\begin{bmatrix}
   	2&1\\
   	0&3
   	\end{bmatrix}\qquad \qquad \begin{bmatrix}
   	6&3\\
   	0&12
   	\end{bmatrix}   	$$
   	\end{حل}
   \item 
   اگر 
   $A=\begin{bmatrix}
   a&b\\
   c&d
   \end{bmatrix}$
   و
   $ad-bc\neq0$
   آنگاه 
   $Ax=0$
   فقط جواب بدیهی دارد.
   \begin{حل}
   $$\begin{bmatrix}
   a&b&0\\
   c&d&0
   \end{bmatrix}\xrightarrow{\frac{1}{a}R_1}
   \begin{bmatrix}
   1&\frac{b}{a}&0\\
   c&d&0
   \end{bmatrix}\xrightarrow{R_2-cR_1\to R_2}
   \begin{bmatrix}
   1&\frac{b}{a}&0\\
   0&d-\frac{cb}{a}&0
   \end{bmatrix}\xrightarrow{R_1-\frac{\frac{b}{a}}{d-\frac{cb}{a}}R_2\to R_1}
   \begin{bmatrix}
   1&0&0\\
   0&\frac{ad-bc}{a}&0
   \end{bmatrix}$$
   
   برای اینکه دستگاه فقط یک جواب داشته باشد که آن جواب جواب بدیهی باشد باید 
   درایه
 $2,2$
   نباید صفر شود پس  $ad-bc\neq0$.
\end{حل}   
\end{enumerate}

\مسئله {}
مربع های جادویی (\lr {magic squre})یکی از ساختار های جالب ترکیبیاتی در ریاضی هستند که در بخش ها گوناگون ریاضی کاربرد دارند و ارتباطات جالبی بین مربع جادویی و ساختار های گرافی و ... وجود دارد،حتی این ساختار ها در علوم دیگر از جمله مکانیک و کامپیوتر نیز کاربرد دارند و هنوز تعداد زیادی مسئله حل نشده در این زمینه وجود دارد.
مربع جادویی یا وفقی جدولی است 
$n\times n$
که خانه های آن با اعداد مثبت 
$1$
تا 
$n^2$
پر شده است به نحوی که مجموع اعداد هر ستون عمودی و هر سطر افقی و قطر آن عدد ثابتی را نشان می دهد برای مثال شکل زیر یک مربع جادویی 
$3\times 3$
است:
\begin{center}

	\begin{tabular}{|c|c|c|}
		\hline
		2&7&6\\
		\hline 
		9&5&1\\
		\hline
		4&3&8\\
		\hline
	\end{tabular}		

\end{center}
$\S$
اگر تعداد مربع های جادویی 
$6\times 6$
را بیابید نمره درس جبر خطی کاربردی شما 20 منظور می شود :) 
$\S$


اگر 
$M_i$
یک مارتیس 
$i\times i$
باشد که درایه های آن اعداد متناظر بر روی یک مربع جادویی 
$i\times i$
باشد آنگاه حاصل ضرب های ماتریسی زیر را بیابید:
$$M_1 \times
\begin{bmatrix}
1
\end{bmatrix}
\qquad M_2 \times
\begin{bmatrix}
1\\
1
\end{bmatrix}
\qquad M_3 \times
\begin{bmatrix}
1\\
1\\
1
\end{bmatrix}
\qquad M_n \times
\begin{bmatrix}
	1\\
	1\\
	\vdots\\
	1
\end{bmatrix}_{n\times1}
$$
همچینین تعیین کنید یک ماتریس 
$M_i$
با کدامیک از اعمال سطری پلکانی همچنان به شکل جادویی می ماند.
\begin{حل}
	می دانیم 
	$M_1=1$
	پس 
	$M_1 \times
	\begin{bmatrix}
	1
	\end{bmatrix}
	=1$
	 همچینین لازم است اشاره شود 
	 $M_2$
	 وجود ندارد ،اما به طور کلی برای 
	 $M_n$ 
	 می دانیم که مربع جادویی شامل تمامی اعداد 
	 $1$
	 تا 
	 $n^2$
	 هست پس مجموع تمام اعداد بر روی آن برابر است با:
	 $\frac{n^2(n^2+1)}{2}$ 
	 از انجاییکه مجموع تمامی اعداد واقع بر سطر ها برابر است با پس مجموع اعداد واقع بر یک سطر برابر است با:
	 $\frac{n^2(n^2+1)}{2}\cdot \frac{1}{n}=\frac{n(n^2+1)}{2}$
	 از این نتیجه می گیریم :
	 $$M_n\times\begin{bmatrix}
	 1\\
	 1\\
	 \vdots\\
	 1
	 \end{bmatrix}_{n\times1}=\begin{bmatrix}
	 \frac{n(n^2+1)}{2}\\
	 \frac{n(n^2+1)}{2}\\
	 \vdots\\
	 \frac{n(n^2+1)}{2}
	 \end{bmatrix}_{n\times1}$$
	 همچنین نتیحه می شود:
	 $$M_3\times\begin{bmatrix}
	 1\\
	 1\\
	
	 1
	 \end{bmatrix}=\begin{bmatrix}
	 \frac{3(3^2+1)}{2}=15\\
	 \frac{3(3^2+1)}{2}=15\\
	 \frac{3(3^2+1)}{2}=15
	 \end{bmatrix}$$
	 هیچکدام یک ااز اعمال سطری پلکانی باعث نمی شود ماتریس جادویی بماند.
\end{حل}

\مسئله{} 
گزاره های زیر ثابت کنید:
\begin{enumerate}
\item 
	اگر معادله 
	$Ax=b$
	به ازای هر 
	$b\in \mathbb{R}^n$
	جواب داشته باشد و به ازای 
	$b=0$
	فقط جواب بدیهی داشته باشد آنگاه  ماتریس 
	$B$
	 را بدین شکل از ماتریس 
	 $A$ 
	 می سازیم که تمامی ستون های کمتر از 
	 $n$ 
	ام ماتریس 
	$A$
	 را با ستون 
	 $n$
	 ام جمع می کنیم و در  ستون 
	 $n $
	 ام ماتریس 
	 $B$
	 قرار می دهیم ثابت کنید معادله 
	 $Bx=b$
	 به ازای هر 
	 $b\in \mathbb{R}^n$
	 جواب دارد و به ازای 
	 $b=0$
	 فقط جواب بدیهی دارد.
	 \begin{حل}
	 	فرض کنیم 
	 	$A$
	 	ماتریسی به شکل 
	 	$\begin{bmatrix}
	 	v_1&v_2&\cdots&v_n
	 	\end{bmatrix}$
	 	ماتریس $B$  بنابر صورت به شکل:
	 	$$\begin{bmatrix}
	 	v_1&v_1+v_2&v_1+v_2+v_3&\cdots&v_1+v_2+\cdots v_n
	 	\end{bmatrix}$$
	 	خواهد بود می دانیم اگر یک معادله به شکل 
	 	$Mx=0$
	 	فقط جواب صفر داشته باشد آنگاه ستون های مستقل خطی هستند،پس در واقع 
	 	$v_1,v_2,v_3,\cdots,v_n$
	 	بردار هایی مستقل خطی هستند و باید ثابت کنیم که:
	 	$$	v_1,v_1+v_2,v_1+v_2+v_3,\cdots,v_1+v_2+\cdots v_n$$
	 	نیز برداری هایی مستقل خطی هستند. پس باید ثابت کنیم اگر:
	 	$$\alpha_1v_1+\alpha_2(v_1+v_2)+\alpha_3(v_1+v_2+v_3)+\cdots+\alpha_n(v_1+v_2+\cdots v_n)=0$$
	 	باشد آنگاه
	 	$\alpha_i$
	 	ها همگی صفر هستند.عبارت بالا را می توانیم اینگونه باز نویسی کنیم:
	 	$$(\alpha_1+\alpha_2+\cdots+\alpha_n)v_1+(\alpha_1+\alpha_2+\cdots+\alpha_{n-1})+\cdots+\alpha_nv_n=0$$
	 	می دانیم که $v_i$ ها مستقل خطی هستند پس اگر حاصل ترکیب خطی آن ها صفر شود باید ضرایب آن ها نیز صفر شود،پس می توان گفت :
	 	$$\alpha_1+\alpha_2+\cdots+\alpha_n=0,\alpha_1+\alpha_2+\cdots+\alpha_{n-1}=0,\cdots,\alpha_{n-1}+\alpha_n=0,\alpha_n=0$$
 	 	از صفر بودن $\alpha_n$ نتیجه می گیریم 
 	 	$\alpha_{n-1}=0$
 	 	نتیجه می شود و همینطور الی آخر،پس استقلال خطی ثابت شد، ثابت کنیم 
 	 	به ازای هر 
 	 	$b$
 	 	وجود دارد 
 	 	$y_1,y_2,\cdots ,y_n$
 	 	که 
 	 	$$v_1y_1+(v_1+v_2)y_2+\cdots+(v_1+v_2+\cdots+v_n)y_n=b\qquad \star$$
 	 	می دانیم برای هر 
 	 	$b$
 	 	وجود دارد 
 	 		$x_1,x_2,\cdots ,x_n$
 	 	که 
 	 	$$v_1x_1+v_2x_2+\cdots+v_nx_n=b$$
 	 	حال تساوی $\star$ را ساده تر می کنیم:
 	 	$$(y_1+y_2+\cdots+y_n)v_1+(y_1+y_2+\cdots+y_{n-1})v_2+\cdots+y_nv_n=b$$
 	 	در نتیجه می توانیم ضرایب $y_i$ را به شکل زیر بیابیم:
 	 	$$y_1+y_2+\cdots+y_n=x_1,y_1+y_2+\cdots+y_{n-1}=x_2,\cdots,y_n=x_n$$
 	 	پس از عبارت بالا این نتیجه می شود
 	 	$y_{n-1}=x_{n-1}-x_{n}$
 	 	و همینطور الی آخر پس در واقع توانستیم $y_n$  هایی را بیابیم که به ازای هر $b$ معادله $Bx=b$ جواب داشته باشد.
 	 	
	 	\end{حل}
	 \item 
	 نشان دهید اگر معادله 
	 $Ax=0$
	بیش از یک جواب داشته باشد و 
	 $A$
	 به شکل 
	 $[a_1 a_2 \ldots a_n]$
	باشد که 
	 $a_i$
	 ها ستون های ماتریس 
	 $A$
	 هستند آنگاه وجود دارد عدد صحیح مانند 
	 $k$
	 ای که 
	 $1<k\le n$
	 و 
	 $Bx=a_k$
	 سازگار باشد
	 ($B=[a_1\ \ a_2 \ \ \ldots \ \ a_{k-1}]$).
	 \begin{حل}
	 	می دانیم اگر 
	 	$Ax=0$
	 	باشد آنگاه ستون های 
	 	$A$
	 	تشکیل بردار هایی می دهند که وابسته خطی هستند فرض کنیم 
	 	$A$
	 	به شکل:
	 	$$A=\begin{bmatrix}
	 	v_1&v_2&v_3&\cdots&v_n
	 	\end{bmatrix}$$
	 	از انجاییکه بردار های $v_1,v_2,\cdots,v_n$ وابسته خطی هستند پس:
	 	$$\exists \alpha_i\neq0 \ \ \alpha_1v_1+\alpha_2v_2+\cdots+\alpha_nv_n=0$$
	 	
	 	بزرگترین $i$ که $\alpha_i\neq0$ را در نظر می گیریم و 
	 	$i=k$
	 	قرار می دهیم آنگاه می توانیم بنویسیم:
	 	$$\alpha_1v_1+\alpha_2v_2+\cdots+\alpha_kv_k=0$$
	 	$$\longrightarrow \frac{\alpha_1v_1+\alpha_2v_2+\cdots+\alpha_{k-1}v_{k-1}}{\alpha_k}=v_k$$
	 	پس می توانیم به عبارت بالا را به شکل ماتریسی نیز بنویسیم آنگاه:
	 	$$[v_1\ \ v_2 \ \ \ldots \ \ v_{k-1}]x=v_k$$
	 	که همان حکم مسئله است.
	 	
	 \end{حل}
	 \item 
	 فرض کنید 
	 $w$
	حوابی از 
	$Ax=b$ 
	باشد و تعریف می کنیم 
	$v_h=w-p$
	.نشان دهید 
	$v_h$
	جوابی از 
	$Ax=0$
	است.این نشان می دهد که هر جوابی از 
	$Ax=b$
	به شکل 
	$w=p+v_h$
	است که 
	$p$
	یک جواب خاص از 
	$Ax=b$
	 است و 
	 $v_h$
	 جوابی از
	 $Ax=0$
	 .
	 \begin{حل}
	 	می دانیم 
	 	$v_h=w-p$
	 	ماتریس 
	 	$A$
	 	را سمت چپ در دو طرف تساوی ضرب می کنیم داریم:
	 	$$Av_h=A(w-p)=Aw_Ap$$
	 	می دانیم 
	 	$w,p$
	 	جواب های 
	 	$Ax=b$
	 	هستند پس:
	 	$Av_h=b-b=0$
	 	درنتیحه 
	 	$v_h$
	 	یک جواب از 
	 	$Ax=0$
	 	است.
	 \end{حل}
	 
\end{enumerate}

\مسئله{}
$u,v$
 را دو بردار مستقل خطی عضو 
 $\mathbb{R}^3$
 در نظر بگیرید و 
 $P$
 را  صفحه ای در نظر بگیرید که از این دو بردار و نقطه 
 $0$
 می گذرد. نمایش پارامتریک 
 $P$
 به شکل 
 $x=su+tv(s,t \in \mathbb{R})$
 است .نشان دهید که یک تبدیل خطی 
 $T:\mathbb{R}^3\longrightarrow \mathbb{R}^3$
 صفحه 
 $P$
 را به صفحه ای که از 
 $0$
 می گذرد یا به خطی که از 
 $0$
 می گذرد و یا به مبدا مختصات در 
 $\mathbb{R}^3$
 نگاشت می کند و همچنین چه چیزی باید در مورد 
 $T(u),T(v)$
 صدق کند که تصویر صفحه 
 $P$
 یک صفحه باشد.
 \begin{حل}
 	اگر تبدیل خطی 
 	$T$
 	رو نقاط صفحه اعمال شود داریم:
 	$$T(x)=T(su+tv)\xrightarrow{\text{$T$خطی است}}=T(su)+T(tv)=sT(u)+tT(v)$$
 	حال با توجه به اینکه
 	$T(0)=0$
 	پس این صفحه تحت این نگاشت از نقطه $0$ می گذرد حال اگر 
 	$T(u),T(v)$
 	دو بردار غیر هم راستا باشند از جواب به شکل ترکیب بردار هایی است که از صفر می گذرند و صفحه ای را تشکیل می دهند اگر یکی از 
 	$T(u),T(v)$
 	به صفر نگاشت شود آنگاه خطی داریم که از صفر می گذرد و اگر هر دو به صفر نگاشت شوند صفحه به یک نقطه صفر نگاشته خواهد شد،قسمت دوم سوال نیز در خلال قسمت اول توضیح داده شد.
 	\end{حل}
 \مسئله{}
 فرض کنید که 
 $span\{v_1,v_2,\ldots,v_p\}=\mathbb{R}^n$
 و 
 $T:\mathbb{R}^n\longrightarrow \mathbb{R}^n$
یک تبدیل خطی باشد که 
$$\forall i \in\{1,\ldots,p \}\ \ T(v_i)=0 $$
آنگاه نشان دهید که
$T$
 یک تبدیل صفر است.(به تبدیلی تبدیل صفر گویند که 
$\forall x\in \mathbb{R}^n \ \ T(x)=0$)
\begin{حل}
$x\in \mathbb{R}^n$
را در نظر بگیرید چون 
$span\{v_1,v_2,v_2,\cdots,v_n\}=\mathbb{R}^n$
آنگاه:
$$\exists \alpha_i \ \ x=\alpha_1v_1+\alpha_2v_2+\cdots+\alpha_nv_n$$
$$\longrightarrow T(x)=T(\alpha_1v_1+\alpha_2v_2+\cdots+\alpha_nv_n)\xrightarrow{\text{$T$خطی است}}=\alpha_1T(v_1)+\alpha_2T(v_2)+\cdots+\alpha_nT(v_n)\xrightarrow{\text{$T(v_i)=0$}}T(x)=0$$	
\end{حل}


\مسئله{}
فرض کنید 
$T:\mathbb{R}^n\longrightarrow \mathbb{R}^n$
یک تبدیل خطی باشد نشان دهید اگر 
$T$	
دو بردار مستقل خطی را به یک مجموعه وابسته خطی نگاشت کند آنگاه 
$T(x)=0$
جواب غیر بدیهی دارد.
\begin{حل}
	فرص کنیم 
	$v_1,v_2$
	دو بردار مستقل خطی باشند و
	$$T(v_1)=u_1\qquad T(v_2)=u_2$$
	$u_1,u_2$
	وابسته خطی هستند پس می توان گفت 
	$u_1=ku_2\quad k\neq0$
	پس می توانیم جواب بردار 
	$v_1-kv_2$
	را تحت نگاشت بیابیم از آنجا که 
	$k\neq0$
	و 
	$v_1,v_2$
	مستقل خطی هستند پس :
	$v_1-kv_2\neq0$
	از نکات بالا می توانیم نتیجه بگیریم :
	$$T(v_1-kv_2)=T(v_1)-kT(v_2)=u_1-ku_2=0$$
	پس یک جواب غیر بدیهی برای مسئله یافتیم و حکم اینگونه ثابت می شود.
	
	
\end{حل}

\مسئله{}
در هر کدام از تبدیل های زیر مشخص کنید تبدیل خطی هست یا نه و در صورت خطی بودن ماتریس استاندارد آن را مشخص کنید.
\begin{حل}
در هر کدام تبدیلات زیر ابتدا خطی بودن را بررسی می کنیم،برای این موضوع باید دو شرط:
\begin{enumerate}
	\item $T(0)=0$
	\item $T(c{\bf u}+d{\bf v})=cT({\bf u})+dT({\bf v})$
\end{enumerate}

برقرار باشند ،و در صورت خطی بودن بایستی ماتریس استاندارد تبدیل خطی را بیابیم.
برای یا فتن ماتریس استاندارد تبدیل خطی، ماتریس
$$A= [T({\bf e_1}) \ldots T({\bf e_n})]$$
را می یابیم که 
$e_j$ 
،
$j$امین ستون ماتریس همانی است.
	
\end{حل}
\begin{enumerate}
	\item
	{\setlength\arraycolsep{0.1em}
	\begin{eqnarray*}
	T:\mathbb{R}^2&\longrightarrow&\mathbb{R}^2\\
	(x_1,x_2)&\longrightarrow&(4x_1-2x_2,3|x_2|)
	\end{eqnarray*}}
\begin{حل}
	ابتدا خطی بودن را بررسی می کنیم 
	$$T(0)=T((0,0))=(4(0)-2(0),3|0|)=(0,0)$$
	$$T(c(x,y)+d(u,v))=T(cx+du,cy+dv)=(4(cx+du)-2(cy+dv),3|cy+dv|)$$
	$$\neq cT(x,y)+dT(u,v)=c(4x-2y,3|y|)+d(4u-2v,3|v|)=(4(cx+du)-2(cy+dv),3c|y|+3d|v|)$$
	پس این تبدیل یک تبدیل خطی نیست.
\end{حل}
\item 
	{\setlength\arraycolsep{0.1em}
	\begin{eqnarray*}
		T:\mathbb{R}^2&\longrightarrow&\mathbb{R}^2\\
		(x_1,x_2)&\longrightarrow&(sin(x_1),x_2)
\end{eqnarray*}}
\begin{حل}
	ابتدا خطی بودن را بررسی می کنیم 
	$$T(0)=T((0,0))=(sin(0),0)=(0,0)$$
	$$T(c(x,y)+d(u,v))=T(cx+du,cy+dv)=(sin(cx+du),cy+dv)=(sin(cx)cos(du)+cos(cx)+sin(du),cy+dv)$$
	$$\neq cT(x,y)+dT(u,v)=c(sin(x),y)+d(sin(u),v)=(csin(x)+dsin(y),cy+dv)$$
	پس این تبدیل یک تبدیل خطی نیست.
\end{حل}

\item
{\setlength\arraycolsep{0.1em}
	\begin{eqnarray*}
		T:\mathbb{R}^3&\longrightarrow&\mathbb{R}^3\\
		(x_1,x_2,x_3)&\longrightarrow&(3x_1,x_1-x_2,2x_1+x_2+x_3)
\end{eqnarray*}}
\begin{حل}
	$$T(0)=T(0,0,0)=(0,0,0)$$
	$$T(c(x_1,x_2,x_3)+d(v_1,v_2,v_3))=T(cx_1+dv_1,cx_2+dv_2,cx_3+dv_3)$$$$=(3cx_1+3dv_1,cx_1+dv_1-cx_2-dv_2,2cx_1+2dv_1+cx_2+dv_2+cx_3+dv_3)$$
	$$=(3cx_1,cx_1-cx_2,2cx_1+cx_2++cx_3)+(3dv_1,dv_1-dv_2,2dv_1+dv_2,+dv_3)$$
	$$=cT(u)+dT(v)$$
	بنابراین این تبدیل خطی است،پس ماتریس استاندارد آن را می یابیم 
	$$T(e_1)=(3,1,2),T(e_2)=(0,-1,1),T(e_3)=(0,0,1)$$
	$$A=[T(e_1)\ \ T(e_2)\ \ T(e_3)]=\begin{bmatrix}
	3&0&0\\
	1&-1&0\\
	2&1&1
	\end{bmatrix}$$
\end{حل}
\end{enumerate}	
\end{document}